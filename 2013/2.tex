\textbf{Text 1}

In the 2006 film version of The Devil Wears Prada, Miranda Priestly, played by Meryl Streep, scolds her unattractive assistant for imagining that high fashion doesn't affect her, Priestly explains how the deep blue color of the assistant's sweater descended over the years from fashion shows to departments stores and to the bargain bin in which the poor girl doubtless found her garment.

This top-down conception of the fashion business couldn't be more out of date or at odds with the feverish would described in Overdressed, Elizabeth Cline's three-year indictment of ``fast fashion''. In the last decade or so, advances in technology have allowed mass-market labels such as Zara, H\&M, and Uniqlo to react to trends more quickly and anticipate demand more precisely. Quicker turn arounds mean less wasted inventory, more frequent release, and more profit. These labels encourage style-conscious consumers to see clothes as disposable--meant to last only a wash or two, although they don't advertise that--and to renew their wardrobe every few weeks. By offering on-trend items at dirt-cheap prices, Cline argues, these brands have hijacked fashion cycles, shaking an industry long accustomed to a seasonal pace.

The victims of this revolution, of course, are not limited to designers. For H\&M to offer a \$5.95 knit miniskirt in all its 2,300-pius stores around the world, it must rely on low-wage overseas labor, order in volumes that strain natural resources, and use massive amounts of harmful chemicals.

Overdressed is the fashion world's answer to consumer-activist bestsellers like Michael Pollan's The Omnivore’s Dilemma. ``Mass-produced clothing, like fast food, fills a hunger and need, yet is non-durable and wasteful,'' Cline argues. Americans, she finds, buy roughly 20 billion garments a year--about 64 items per person--and no matter how much they give away, this excess leads to waste.

Towards the end of Overdressed, Cline introduced her ideal, a Brooklyn woman named Sarah Kate Beaumont, who since 2008 has made all of her own clothes--and beautifully. But as Cline is the first to note, it took Beaumont decades to perfect her craft; her example can't be knocked off.

Though several fast-fashion companies have made efforts to curb their impact on labor and the environment--including H\&M, with its green Conscious Collection line--Cline believes lasting change can only be effected by the customer. She exhibits the idealism common to many advocates of sustainability, be it in food or in energy. Vanity is a constant; people will only start shopping more sustainably when they can't afford not to.

\begin{tabbing}
21. Priestly criticizes her assistant for her\\
\hspace{0cm}\ \=[A] poor bargaining skill. \quad\quad\quad\quad\= [B] insensitivity to fashion.\\
\> [C] obsession with high fashion. \> [D] lack of imagination.\\

22. According to Cline, mass-market labels urge consumers to\\
\> [A] combat unnecessary waste.\\
\> [B] shut out the feverish fashion world.\\
\> [C] resist the influence of advertisements.\\
\> [D] shop for their garments more frequently.\\

23. The word “indictment” (Line 3, Para.2) is closest in meaning to\\
\> [A] accusation.
\> [B] enthusiasm.\\
\> [C] indifference.
\> [D] tolerance.\\

24. Which of the following can be inferred from the last paragraph?\\
\> [A] Vanity has more often been found in idealists.\\
\> [B] The fast-fashion industry ignores sustainability.\\
\> [C] People are more interested in unaffordable garments.\\
\> [D] Pricing is vital to environment-friendly purchasing.\\

25. What is the subject of the text?\\
\> [A] Satire on an extravagant lifestyle.\\
\> [B] Challenge to a high-fashion myth.\\
\> [C] Criticism of the fast-fashion industry.\\
\> [D] Exposure of a mass-market secret.\\
\end{tabbing}