\textbf{Text 4}

If the trade unionist Jimmy Hoffa were alive today, he would probably represent civil servant.
When Hoffa’s Teamsters were in their prime in 1960, only one in ten American government workers
belonged to a union; now 36\% do. In 2009 the number of unionists in America’s public sector 
passed that of their fellow members in the private sector. In Britain, more than half of public-sector 
workers but only about 15\% of private-sector ones are unionized.

There are three reasons for the public-sector unions’ thriving. First, they can shut things down without suffering much in the way of consequences. Second, they are mostly bright and well-educated. A quarter of America’s public-sector workers have a university degree. Third, they now dominate left-of-centre politics. Some of their ties go back a long way. Britain’s Labor Party, as its name implies, has long been associated with trade unionism. Its current leader, Ed Miliband, owes his position to votes from public-sector unions.

At the state level their influence can be even more fearsome. Mark Baldassare of the Public Policy Institute of California points out that much of the state’s budget is patrolled by unions. The teachers’ unions keep an eye on schools, the CCPOA on prisons and a variety of labor groups on health care.

In many rich countries average wages in the state sector are higher than in the private one. But the real gains come in benefits and work practices. Politicians have repeatedly “backloaded” public-sector pay deals, keeping the pay increases modest but adding to holidays and especially pensions that are already generous.

Reform has been vigorously opposed, perhaps most egregiously in education, where charter schools, academies and merit pay all faced drawn-out battles. Even though there is plenty of evidence that the quality of the teachers is the most important variable, teachers’ unions have fought against getting rid of bad ones and promoting good ones.

As the cost to everyone else has become clearer, politicians have begun to clamp down. In Wisconsin the unions have rallied thousands of supporters against Scott Walker, the hardline Republican governor. But many within the public sector suffer under the current system, too.

John Donahue at Harvard’s Kennedy School points out that the norms of culture in Western 
civil services suit those who want to stay put but is bad for high achievers. The only 
American public-sector workers who earn well above \$250,000 a year are university sports 
coaches and the president of the United States. Bankers’ fat pay packets have attracted 
much criticism, but a public-sector system that does not reward high achievers may be a 
much bigger problem for America.

\begin{tabbing}
36. It can be learned from the first paragraph that\\
\ [A] Teamsters still have a large body of members.\\
\ [B] Jimmy Hoffa used to work as a civil servant.\\
\ [C] unions have enlarged their public-sector membership.\\
\ [D] the government has improved its relationship with unionists.\\
37. Which of the following is true of Paragraph 2?\\
\ [A] Public-sector unions are prudent in taking actions.\\
\ [B] Education is required for public-sector union membership.\\
\ [C] Labor Party has long been fighting against public-sector unions.\\
\ [D] Public-sector unions seldom get in trouble for their actions.\\
38. It can be learned from Paragraph 4 that the income in the state sector is\\
\ \= [A] illegally secured. \quad\quad\quad\quad\quad\quad\quad\quad\= [B] indirectly augmented.\\
\ \> [C] excessively increased. \> [D] fairly adjusted.\\
39. The example of the unions in Wisconsin shows that unions\\
\ [A]often run against the current political system.\\
\ [B]can change people’s political attitudes.\\
\ [C]may be a barrier to public-sector reforms.\\
\ [D] are dominant in the government.\\
40. John Donahue’s attitude towards the public-sector system is one of\\
\ \> [A] disapproval. \quad [B] appreciation. \> [C] tolerance. \quad [D] indifference. 
\end{tabbing}
