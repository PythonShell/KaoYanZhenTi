17. 1971年,迪斯尼乐园的路径设计获得了“世界最佳设计”奖,设计师格罗培斯却说,“其实那不是我的设计”。原因是在迪斯尼乐园主体工程完工后,格罗培斯暂停修筑乐园里的道路,并在空地上撒上草种。五个月后,乐园里绿草茵茵,草地上被游客踏出了不少宽窄不一的小路。格罗培斯根据这些行人踏出来的小路铺设了人行道,成了“优雅自然、简捷便利、个性突出”的优秀设计。格罗培斯的设计智慧对我们认识和实践活动的启示是
\begin{choices}
	\choice0 要从生活实践中获取灵感
	\choice0 要尊重群众的实际需求
	\choice0 不要对自然事物做任何改变
	\choice0 要对事物本来面目做直观反映
\end{choices}
18. 在资本主义社会里,银行垄断资本和工业垄断资本密切地融合在一起,产生了一种新型的垄断资本,即金融资本。在金融资本形成的基础上,产生了金融寡头。
金融寡头操作、控制社会的主要方式有
\begin{choices}
	\choice0 通过“参与制”实现其在经济领域中的统治
	\choice0 通过同政府的 “个人联合”实现其对国家机器的控制
	\choice0 通过政策咨询机构影响和左右内外政策
	\choice0 通过新闻媒体实现国民思想意识的一元化
\end{choices}
19. 2008年由美国次贷危机引发了全球性的经济危机,很多西方人感叹这一次经济危机从根本上仍未超出一百多年前马克思在《资本论》中对资本主义经济危机的理论判断和精辟分析。马克思对资本主义经济危机科学分析的深刻性主要表现为
\begin{choices}
	\choice0 指明经济危机的实质是生产相对过剩
	\choice0 揭示造成相对过剩的制度原因是生产资料的资本主义私有制
	\choice0 指出经济危机的深层根源是人性的贪婪
	\choice0 强调政府对经济的干预是摆脱经济危机的根本出路
\end{choices}
20. 19世纪中叶,马克思恩格斯把社会主义由空想变为科学,奠定这一飞跃的理论基石是
\begin{choices}
	\choice0 阶级斗争学说
	\choice0 劳动价值论
	\choice0 唯物史观
	\choice0 剩余价值理论
\end{choices}
21. 在马克思主义中国化的过程中,产生了毛泽东思想和中国特色社会主义理论体系,这两大理论成果的一脉相承性主要体现在,二者具有共同的
\begin{choices}
	\choice0 马克思主义的理论基础
	\choice0 革命和建设的根本任务
	\choice0 实事求是的理论精髓
	\choice0 和平与发展的时代背景
\end{choices}
22. 社会主义市场经济体制是社会主义基本制度与市场经济的结合。这一结合既体现社会主义的制度特征,又具有市场经济的一般特征。社会主义市场经济体制体现社会主义制度特征的方面主要表现在
\begin{choices}
	\choice0 在所有制结构上,以公有制为主体、多种所有制经济共同发展
	\choice0 在分配制度上,以按劳分配为主体、多种分配方式并存
	\choice0 在宏观调控上,以实现最广大劳动人民利益为出发点和归宿
	\choice0 在资源配置上,以市场为手段,发挥市场的基础性作用
\end{choices}
23. 毛泽东指出:“国家的统一,人民的团结,国内各民族的团结,这是我们的事业必定要胜利的基本保证。”这一思想对于我们在新的历史条件下处理民族关系的现实意义有
\begin{choices}
	\choice0 民族关系始终是我们这个多民族国家至关重要的政治和社会关系
	\choice0 民族问题始终是建设中国特色社会主义必须认真解决的一个重大问题
	\choice0 巩固和发展各民族的团结关系到国家的统一和边疆巩固
	\choice0 加强和巩固各民族的团结是实现中华民族伟大复兴的必然要求
\end{choices}
24. 改革、发展、稳定好比现代化建设棋盘上的三着紧密关联的战略性棋子,每一着棋都下好了,相互促进,就会全局皆活;如果有一着下不好,其他两着也会陷入困境,就可能全局受挫。改革开放以来,党在处理改革、发展、稳定关系方面积累的经验和主要原则包括
\begin{choices}
	\choice0 保持改革、发展、稳定在动态中的相互协调和相互促进
	\choice0 把实现社会稳定作为促进改革、发展的根本出发点
	\choice0 把改革的力度、发展的速度和社会可以承受的程度统一起来
	\choice0 把不断改善人民生活作为处理改革、发展、稳定关系的重要结合点
\end{choices}
25. 做大分好社会财富这个“蛋糕”始终是我国政府面临的重大任务。做大“蛋糕”是政府的责任,分好“蛋糕”是政府的良知。合理调整收入分配关系,分好社会财富这个“蛋糕”是
\begin{choices}
	\choice0 实现社会公平的重要体现
	\choice0 解决当前收入分配领域突出问题的需要
	\choice0 实现共同富裕的内在要求
	\choice0 为了使人民共享改革发展的成果
\end{choices}
26. 辛亥革命是我国近代史上一次比较完全意义上的资产阶级民主革命。这是因为辛亥革命
\begin{choices}
	\choice0 提出了平均地权,耕者有其田的重要原则
	\choice0 建立了中国近代史上第一个资产阶级政党
	\choice0 制定了比较完整的资产阶级民主革命纲领
	\choice0 结束了封建君主专制制度,建立了资产阶级共和国
\end{choices}
27. 第二次世界大战期间,明确规定将台湾、澎湖列岛归还中国的有关国际条约是
\begin{choices}
	\choice0 《德黑兰宣言》
	\choice0 《开罗宣言》
	\choice0 《雅尔塔协定》
	\choice0 《波茨坦公告》
\end{choices}
28. 延安整风运动是一场非常伟大的思想解放运动。这一运动最主要的任务是反对主观主义。主观主义的主要表现形式为
\begin{choices}
	\choice0 教条主义
	\choice0 形式主义
	\choice0 经验主义
	\choice0 宗派主义
\end{choices}
29. 中国共产党根据马克思列宁主义关于农业社会主义改造的思想,从我国的实际出发,开创了一条有中国特点的农业合作化道路,成功地实现了对个体农业的社会主义改造。其历史经验主要有
\begin{choices}
	\choice0 国家用先进的技术和装备发展农业经济
	\choice0 遵循自愿互利、典型示范和国家帮助的原则
	\choice0 在土地改革后不失时机地引导个体农民走互助合作道路
	\choice0 采取从互助组到初级社再到高级社的逐步过渡形式
\end{choices}
30. 刑法的基本原则是指刑法特有的在刑法的立法、解释和适用过程中所必须普遍遵循的具有全局性、根本性的准则。我国刑法明文规定的基本原则有
\begin{choices}
	\choice0 罪刑法定原则
	\choice0 疑罪从无原则
	\choice0 罪刑相当原则
	\choice0 适用刑法一律平等原则
\end{choices}
31. 有位法学家曾经说过:“法律必须被信仰,否则等于形同虚设。”这句话表明,一个人只有从内心深处真正认同、信任和信仰法律,才会自觉维护法律的权威。由此可见
\begin{choices}
	\choice0 法律的内在说服力是法律权威的内在基础
	\choice0 法律权威不可能完全建立在外在强制力的基础之上
	\choice0 法律信仰与宗教信仰没有本质的区别
	\choice0 法律信仰是法律制定和执行的根本依据
\end{choices}
32. 2010年10月15日至18日, 中国共产党第十七届中央委员会第五次全体会议在北京举行。全会审议通过的《中共中央关于制定国民经济和社会发展第十二个五年规划的建议》强调,在当代中国, 坚持发展是硬道理的本质要求,就是坚持科学发展,更加注重
\begin{choices}
	\choice0 以人为本
	\choice0 全面协调可持续发展
	\choice0 统筹兼顾
	\choice0 保障和改善民生,促进社会公平正义
\end{choices}
33. 2010年4月13日,胡锦涛主席在核安全峰会上发表题为《携手应对核安全挑战,共同促进和平与发展》的讲话,强调中国本着负责任的态度,高度重视核安全,坚决反对核扩散和核恐怖主义,为此作出了一系列积极努力。其中包括
\begin{choices}
	\choice0 全面加强核安全能力
	\choice0 严格履行核安全国际义务
	\choice0 重视并积极参与国际核安全合作
	\choice0 积极向发展中国家提供核安全援助
\end{choices}
\vspace{6pt}
