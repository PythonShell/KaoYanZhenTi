1. 我国数学家华罗庚在一次报告中以“一支粉笔多长为好”为例来讲解他所倡导的优选法。对此,他解释道:“每只粉笔都要丢掉一段一定短的粉笔头,但就这一点来说,愈长愈好。但太长了,使用起来很不方便,而且容易折断。每断一次,必然多浪费一个粉笔头,反而不合适。因而就出现了粉笔多长合适的问题——这就是一个优选问题。”所谓优选问题,从辩证法的角度看,就是要
\begin{choices}
	\choice0 注重量的积累
	\choice0 保持事物质的稳定性
	\choice0 坚持适度原则
	\choice0 全面考虑事物属性的多样性
\end{choices}
2. 社会存在是指社会的物质生活条件,它有多方面的内容,其中最能集中体现人类社会物质性的是
\begin{choices}
	\choice0 社会形态
	\choice0 地理环境
	\choice0 人口因素
	\choice0 生产方式
\end{choices}

3. 马克思把商品转换成货币称为“商品的惊险的跳跃”,“这个跳跃如果不成功,摔坏的不是商品,但一定是商品的占有者。”这是因为只有商品变为货币
\begin{choices}
	\choice0 货币才能转化为资本
	\choice0 价值才能转化为使用价值
	\choice0 抽象劳动才能转化为具体劳动
	\choice0 私人劳动才能转化为社会劳动
\end{choices}
4. 邓小平指出:“社会主义究竟是个什么样子,苏联搞了很多年,也并没有完全搞清楚。可能列宁的思路比较好,搞了个新经济政策,但是后来苏联的模式僵化了。”列宁新经济政策关于社会主义的思路之所以“比较好”是因为
\begin{choices}
	\choice0 提出了比较系统的社会主义建设纲领
	\choice0 根据俄国的实际情况来探索社会主义建设的道路
	\choice0 为俄国找到了一种比较成熟的社会发展模式
	\choice0 按照马克思主义关于未来社会主义的设想来建设社会主义
\end{choices}
5. 1927年大革命失败后,党的工作重心开始转向农村,在农村建立革命根据地。农村革命根据地能够在中国长期存在和发展的根本原因是
\begin{choices}
	\choice0 中国是一个政治、经济、文化发展极不平衡的半殖民地半封建的大国。
	\choice0 良好的群众基础和革命形势的继续向前发展
	\choice0 相当力量正式红军的存在
	\choice0 党的领导及正确的政策
\end{choices}
6. 社会主义初级阶段基本经济制度,既包括公有制经济,也包括非公有制经济。把非公有制经济纳入社会主义初级阶段基本经济制度中,是因为非公有制经济
\begin{choices}
	\choice0 是社会主义性质的经济成分
	\choice0 是社会主义经济的重要组成部分
	\choice0 是为社会主义服务的经济成分
	\choice0 在社会主义初级阶段不占主体地位
\end{choices}
7. 党的十七大通过的党章把“和谐”与“富强、民主、文明”一起作为社会主义现代化建设的目标写入社会主义初级阶段的基本路线。其原因在于社会和谐是
\begin{choices}
	\choice0 中国特色社会主义的本质属性
	\choice0 中国传统文化的价值取向
	\choice0 社会建设的内在要求
	\choice0 解决收入分配差距的重要途径
\end{choices}
8. 深化文化体制改革,要坚持公益性文化事业和经营性文化产业协调发展。发展经营性文化产业的根本任务是
\begin{choices}
	\choice0 繁荣文化市场,满足人民群众多方面、多层次、多样化的文化需求
	\choice0 保障人民群众基本的文化权益
	\choice0 构建覆盖全社会的比较完备的公共文化服务体系
	\choice0 加快文化产业基地和区域性特色文化产业群建设
\end{choices}
9. 从1840年至1919年的80年间,中国人民对外来侵略进行了英勇顽强的反抗。但历史的反侵略战争,都是以中国失败、中国政府被迫签订丧权辱国的条约而告结束的。从中国内部因素来分析,其根本原因是
\begin{choices}
	\choice0 军事战略错误
	\choice0 社会制度腐败
	\choice0 经济技术落后
	\choice0 思想观念保守
\end{choices}
10. 1953年9月,彭德怀在一份报告中说,抗美援朝战争的胜利雄辩地证明:“西方侵略者几百年来只要在东方一个海岸上架起几尊大炮就可霸占一个国家的时代一去不复返了。”这场战争的胜利
\begin{choices}
	\choice0 结束了西方列强霸权主义的历史
	\choice0 打破了美国军队不可战胜的神话
	\choice0 奠定了民族独立人民解放的基础
	\choice0 赢得了近代以来中华民族反抗对敌入侵的第一次完全胜利
\end{choices}
11. 社会主义法律在国家和社会生活中的权威和尊严是建设社会主义法治国家的前提条件。法律权威是就国家和社会管理过程中法律的地位和作用而言的,是指
\begin{choices}
	\choice0 法的强制性
	\choice0 法的不可违抗性
	\choice0 法的合理性
	\choice0 法的规范性
\end{choices}
12. 法律的指引作用主要是通过授权性规范、禁止性规范和义务性规范三种形式来实现的。其中义务性规范是告诉人们
\begin{choices}
	\choice0 不得或者不准做什么
	\choice0 可以或者有权做什么
	\choice0 应当或者必须做什么
	\choice0 能够或者不能做什么
\end{choices}
13. 道德的功能是指道德作为社会意识的特殊形式对于社会发展所具有的功效与能力。其中最突出的也是重要的社会功能是
\begin{choices}
	\choice0 辩护功能
	\choice0 沟通功能
	\choice0 调节功能
	\choice0 激励功能
\end{choices}
14. 理想作为一种精神现象,是人类社会实践的产物。理想源于现实,又超越现实,在现实中有多种类型。从层次上划分,理想有
\begin{choices}
	\choice0 个人理想和社会理想
	\choice0 道德理想和政治理想
	\choice0 生活理想和职业理想
	\choice0 崇高理想和一般理想
\end{choices}
15. 2010年10月1日,“嫦娥二号”卫星在西昌卫星发射中心发射升空并成功“奔月”,实现了我国
\begin{choices}
	\choice0 深空探测“零的突破”
	\choice0 首次月球软着陆和自动巡视勘测
	\choice0 首次月球样品自动取样返回探测
	\choice0 运载火箭直接将卫星发射至地月转移轨道等多项技术突破
\end{choices}
16. 2003年3月,美国率其盟友发动了长达7年之久的对伊拉克战争,给伊拉克人民造成了深重灾难。2010年8月19日,美军最后一批作战部队从伊拉克撤离。这表明,美国在多重压力下
\begin{choices}
	\choice0 调整军事部署
	\choice0 改变先发制人战略
	\choice0 转向本土反恐为主
	\choice0 放弃单边名义
\end{choices}
\vspace{6pt}
