17. 关于龙的形象,自古以来就有“角似鹿、头似驼、眼似兔、项似蛇、腹似蜃、鳞似鱼、爪似鹰、掌似虎、耳似牛”的说法。这表明
\begin{choices}
	\choice0 观念的东西是移入人脑并在人脑中改造过的物质的东西
	\choice0 一切观念都是现实的模仿
	\choice0 虚幻的观念也是对事物本质的反映
	\choice0 任何观念都可以从现实世界中找到其物质“原型”
\end{choices}
18. 某地乡村公路边有很多柿子园,金秋时节农民采摘柿子时,最后总要在树上留一些熟透的柿子。果农们说,这是留给喜鹊的食物。每到冬天,喜鹊都在果树上筑巢过冬,到春天也不飞走,整天忙着捕捉果树上的虫子,从而保证了来年柿子的丰收。从这个事例中我们受到的启示是
\begin{choices}
	\choice0 事物之间有其固有的客观联系
	\choice0 人们可以发现并利用规律来实现自己的目的
	\choice0 人与自然的关系是相互利用的关系
	\choice0 保持生态系统的平衡是人类生存发展的必要条件
\end{choices}
19. 2006年7月12日凌晨,刘翔在瑞士洛桑国际田联超级大奖赛男子110米栏比赛中,以12秒88勇夺冠军,打破了由英国名将科林·杰克逊保持了13年之久的12秒91的世界记录。科林·杰克逊在谈起自己已被打破的记录时,没有一丝沮丧:“我一点也不失望。正相反,我感到非常兴奋。”他说:“记录本来就是用来被打破的。”这在哲学上的启示是
\begin{choices}
	\choice0 创新是永无止境的
	\choice0 不断超越前人是历史发展的规律
	\choice0 凡是在历史上产生的都要在历史上灭亡
	\choice0 一切事物都是作为过程而存在,作为过程而发展
\end{choices}
20. 以人为本是科学发展观的本质和核心。以人为本中的“人”是指
\begin{choices}
	\choice0 具体的、现实的人
	\choice0 广大的人民群众
	\choice0 作为个体的个人
	\choice0 社会全体成员
\end{choices}
21. 商品的市场价格发生变化
\begin{choices}
	\choice0 与货币的价值量变化无关
	\choice0 与商品的价值量变化有关
	\choice0 与商品的生产价格变化无关
	\choice0 与商品的供求变化有关
\end{choices}
22. 利润率表示全部预付资本的增殖程度,提高利润率的途径有
\begin{choices}
	\choice0 提高剩余价值率
	\choice0 提高资本有机构成
	\choice0 加快资本周转速度
	\choice0 节省不变成本
\end{choices}
23. 生产要素市场包括
\begin{choices}
	\choice0 土地市场
	\choice0 商品市场
	\choice0 资本市场
	\choice0 劳动力市场
\end{choices}
24. $G-W-G'$是
\begin{choices}
	\choice0 货币资本的循环公式
	\choice0 生产资本的循环公式
	\choice0 商品资本的循环公式
	\choice0 资本总公式
\end{choices}
25. 新民主主义革命时期,以国共合作为基础所建立的统一战线有
\begin{choices}
	\choice0 国民革命联合战线
	\choice0 工农民主统一战线
	\choice0 抗日民族统一战线
	\choice0 人民民主统一战线
\end{choices}
26. 1942年,毛泽东在《整顿党的作风》中指出,我们要的是马克思列宁主义的学风。学风问题主要是指
\begin{choices}
	\choice0 对待知识分子的态度问题
	\choice0 领导机关、全体干部、全体党员的思想方法问题
	\choice0 我们对待马克思列宁主义的态度问题
	\choice0 全党同志的工作态度问题
\end{choices}
27. 20世纪50年代中期,邓小平多次强调,执政的中国共产党必须接受来自几个方面的监督,具体包括
\begin{choices}
	\choice0 党内的监督
	\choice0 人民群众的监督
	\choice0 海外人士的监督
	\choice0 民主党派和无党派民主人士的监督
\end{choices}
28. 坚持和完善社会主义初级阶段个人收入分配制度,就要规范收入分配秩序,其中包括
\begin{choices}
	\choice0 着力提高低收入者收入水平
	\choice0 逐步扩大中等收入者比重
	\choice0 有效调节过高收入
	\choice0 坚决取缔非法收入
\end{choices}
29. 依法治国是中国共产党领导人民治理国家的基本方略,其基本要点有
\begin{choices}
	\choice0 中国共产党领导人民实行依法治国
	\choice0 形成一套比较完备的法律制度
	\choice0 对国家事务、经济文化事业和社会事务的管理工作都要依法进行
	\choice0 依法治国的最重要依据是宪法和法律
\end{choices}
30. 建设社会主义新农村的一项重要任务是培养新型农民,具体措施有
\begin{choices}
	\choice0 加快发展农村义务教育
	\choice0 加强劳动力技能培训
	\choice0 发展农村文化事业
	\choice0 加速农村剩余劳动力的转移
\end{choices}
31. 社会主义核心价值体系是建设和谐文化的根本,它的基本内容包括
\begin{choices}
	\choice0 马克思主义指导思想
	\choice0 中国特色社会主义共同理想
	\choice0 以爱国主义为核心的民族精神和以改革创新为核心的时代精神
	\choice0 社会主义荣辱观
\end{choices}
32. 从2006年1月1日起,我国废止《农业税条例》,这是具有划时代意义的历史事件,它有利于
\begin{choices}
	\choice0 促进城乡税制的统一
	\choice0 推进工业反哺农业、城市支持农村
	\choice0 逐步消除城乡差别,推进城乡统筹发展
	\choice0 增加农民收入,提高消费水平
\end{choices}
33. 中非合作论坛是首脑外交的新形势。中国国家主席、副主席和总理及非洲4国的总统和非洲统一组织秘书长参加了第一届部长级会议并发表讲话;14个非洲国家的领导人及44个国家的88位部长参加了2003年第二届部长级会议;2006年的中非合作论坛北京峰会更是吸引了非洲40多个国家的元首和政府首脑参加。首脑外交对中非关系的重要意义主要表现在
\begin{choices}
	\choice0 推动了和平、稳定、公正、合理的国际新秩序的建立
	\choice0 增进了友谊,促进了贸易往来
	\choice0 体现了平等观念
	\choice0 开辟了“南南合作”的新路
\end{choices}
\vspace{6pt}
