1. “风定花犹落,鸟鸣山更幽”形象地表达了动和静的辩证关系是
\begin{choices}
	\choice0 静不是动,动不是静
	\choice0 静中有动,动中有静
	\choice0 动是必然的,静是偶然的
	\choice0 动是静的原因,静是动的结果
\end{choices}
2. “挟泰山以超北海,语人曰吾不能,是诚不能也。为长者折枝,语人曰吾不能,是不为也,非不能也。”《孟子》中的这段话启示我们,做事情要区分可能性和不可能性,二者的区别在于
\begin{choices}
	\choice0 人的主观努力程度
	\choice0 对人是否有利
	\choice0 现实中有无根据和条件
	\choice0 现实中的根据和条件是否充分
\end{choices}
3. 马克思根据人的发展状况把人类历史划分为三大形态。它们是
\begin{choices}
	\choice0 自然经济社会、商品经济社会、时间经济社会
	\choice0 原始公有制社会、私有制社会、共产主义公有制社会
	\choice0 农业社会、工业社会、信息社会
	\choice0 人的依赖性社会、物的依赖性社会、人的自由全面发展社会
\end{choices}
4. 列宁说:“意识到自己的奴隶地位而与之作斗争的奴隶,是革命家;没有意识到自己的奴隶地位而过着默默无言、浑浑噩噩、忍气吞声的奴隶生活的奴隶,是十足的奴隶。对奴隶生活的各种好处津津乐道并对和善的好主人感激不尽以至垂涎欲滴的奴隶是奴才,无耻之徒。”这三种奴隶的思想意识之所有有如此巨大的差异,是由于
\begin{choices}
	\choice0 人的社会意识并不都是社会存在的反映
	\choice0 人的社会意识与社会存在具有不一致性
	\choice0 人的社会意识中的各种形式之间相互作用
	\choice0 人的社会意识具有历史继承性
\end{choices}
5. 货币的本质是
\begin{choices}
	\choice0 商品交换的媒介物
	\choice0 商品价值的一般等价物
	\choice0 商品的等价物
	\choice0 商品的相对价值形式
\end{choices}
6. 在资本主义社会,农业资本家和土地所有者之间争夺的是
\begin{choices}
	\choice0 形成级差地租I的超额利润
	\choice0 形成级差地租II的超额利润
	\choice0 形成绝对地租的超额利润
	\choice0 形成垄断地租的超额利润
\end{choices}
7. 作为商品的资本是
\begin{choices}
	\choice0 商业资本
	\choice0 借贷资本
	\choice0 产业资本
	\choice0 流通资本
\end{choices}
8. 在中国共产党的历史上,第一次鲜明地提出“马克思主义中国化”的命题和任务的会议是
\begin{choices}
	\choice0 党的二大
	\choice0 遵义会议
	\choice0 党的六届六中全会
	\choice0 党的七大
\end{choices}
9. 国民革命失败后,毛泽东在八七会议上提出的著名论断是
\begin{choices}
	\choice0 须知政权是由枪杆子中取得的
	\choice0 兵民是胜利之本
	\choice0 一切反动派都是纸老虎
	\choice0 星星之火,可以燎原
\end{choices}
10. 1957年,毛泽东在《关于正确处理人民内部矛盾的问题》中指出,在我国,工人阶级与民族资产阶级的矛盾属于人民内部的矛盾。如果处理不当,会变成
\begin{choices}
	\choice0 对抗性的敌我矛盾
	\choice0 非对抗性的敌我矛盾
	\choice0 对抗性的人民内部矛盾
	\choice0 非对抗性的人民内部矛盾
\end{choices}
11. “三个代表”重要思想的根本出发点和落脚点是
\begin{choices}
	\choice0 实现社会主义现代化
	\choice0 发展社会主义社会生产力
	\choice0 发展社会主义民主,尊重和保障人权
	\choice0 实现人民愿望、满足人民需要、维护人民利益
\end{choices}
12. 社会主义的道德建设的核心是
\begin{choices}
	\choice0 为人民服务
	\choice0 集体主义
	\choice0 诚实守信
	\choice0 爱国主义
\end{choices}
13. 中国共产党和中国政府始终尊重和保护人权,认为首要的人权是
\begin{choices}
	\choice0 参政权、议政权
	\choice0 自由权、平等权
	\choice0 生存权、发展权
	\choice0 选举权、被选举权
\end{choices}
14. 《中共中央关于构建社会主义和谐社会若干重大问题的决定》指出,社会和谐是中国 特色社会主义的
\begin{choices}
	\choice0 根本任务
	\choice0 根本原则
	\choice0 本质属性
	\choice0 基本要求
\end{choices}
15. 胡锦涛在学习《江泽民文选》报告会上的讲话中指出,我们学习《江泽民文选》必须牢牢把握的主题是
\begin{choices}
	\choice0 建设中国特色社会主义
	\choice0 以经济建设为中心
	\choice0 完善社会主义民主和法制
	\choice0 加强社会主义精神文明建设
\end{choices}
16. 中俄两国互办“国家年”活动是两国最高领导人做出的一项重大政治决定。这表明
\begin{choices}
	\choice0 两国的合作重点已转向文化领域
	\choice0 中俄战略协作伙伴关系的内涵已发生根本变化
	\choice0 “国家年”活动将成为中俄双边长期交往的主线
	\choice0 双方将全面提升在各个领域的合作水平
\end{choices}
\vspace{6pt}
