1. 物质和意识的对立只有在非常有限的范围内才有绝对的意义,超过这个范围便是相对的了,这个范围是指
\begin{choices}
	\choice0 物质和意识何者为第一性
	\choice0 物质和意识是否具有统一性
	\choice0 物质和意识何者更为重要
	\choice0 物质和意识何者与社会生活的关系更密切
\end{choices}
2. 1978年关于真理标准大讨论是一场新的思想解放运动。实践之所以成为检验真理的唯一标准是由
\begin{choices}
	\choice0 真理的主观性和实践的客观性所要求的
	\choice0 真理的相对性和实践的决定性所预设的
	\choice0 真理的属性和实践的功能所规定的
	\choice0 真理的本性和实践的特点所决定的
\end{choices}

3. 近年来,马克思的《资本论》在西方一些国家销量大增。列宁曾说,马克思的《资本论》的成就之所以如此之大,是由于这本书使读者看到整个资本主义社会形态是个活生生的形态,既有“骨骼”,又有“血肉”。人类社会作为一种活的有机体,其“骨骼”系统是指
\begin{choices}
	\choice0 地理环境、人口因素和生产方式等社会物质生活条件
	\choice0 与一定的生产力相适应的生产关系
	\choice0 建立在一定经济基础之上的政治法律制度及设施
	\choice0 由政治法律思想、道德、宗教、哲学等构成的社会意识形态
\end{choices}
4. 卢梭在《论人类不平等的起源和基础》中说道:“我认为,在人类的一切知识中,最有用但也最不完善的知识就是关于人的知识。”马克思的唯物史观破解了人是什么这一“司芬克斯之谜”,马克思在《关于费尔巴哈的提纲》中指出,人的本质在其现实性上是
\begin{choices}
	\choice0 自然属性和社会属性的内在统一
	\choice0 所有人共同属性的概括
	\choice0 一切社会关系的总和
	\choice0 自由理性的外化
\end{choices}
5. 流通中的货币需要量是考察经济生活运行的一项重要指标,假设某国去年的商品价格总额为24万亿元,流通中需要的货币量为3万亿元,若今年该国商品价格总额增长10\%,其他条件不变,今年流通中需要的货币量为
\begin{choices}
	\choice0 4.2万亿元
	\choice0 3.5万亿元
	\choice0 3.3万亿元
	\choice0 2.4万亿元
\end{choices}
6. 国家垄断资本主义条件下,政府对经济生活进行干预和调节的实质是
\begin{choices}
	\choice0 维护垄断资产阶级的整体利益和长远利益
	\choice0 维持资本主义经济稳定增长
	\choice0 消除或防止经济危机的爆发
	\choice0 提高资本主义社会的整体福利水平
\end{choices}
7. 某钢铁厂因铁矿石价格上涨,增加了该厂的预付资本数量,这使得该厂的资本构成发生了变化,所变化的资本构成是
\begin{choices}
	\choice0 资本技术构成
	\choice0 资本价值构成
	\choice0 资本物质构成
	\choice0 资本有机构成
\end{choices}
8. 1925年毛泽东在《中国社会各阶级的分析》中指出,中国过去一切革命斗争成效甚少,其基本原因就是
\begin{choices}
	\choice0 没有找到革命的新道路
	\choice0 没有扩大民主主义宣传
	\choice0 没有到群众中做实际的调查
	\choice0 没有团结真正的朋友以攻击真正的敌人
\end{choices}
9. 延安时期,毛泽东写下了著名的《实践论》《矛盾论》,主要是为了克服党内严重的
\begin{choices}
	\choice0 经验主义
	\choice0 冒险主义
	\choice0 机会主义
	\choice0 教条主义
\end{choices}
10. 中共七届二中全会后,党制定和实行的新民主主义经济建设的方针政策是
\begin{choices}
	\choice0 既反保守又反冒进,在综合平衡中稳步前进
	\choice0 公私兼顾、劳资两利、城乡互助、内外交流
	\choice0 调整、巩固、充实、提高
	\choice0 实现速度、结构、质量的统一
\end{choices}
11. 科学发展观的根本方法是
\begin{choices}
	\choice0 把发展作为第一要义
	\choice0 以人为本
	\choice0 统筹兼顾
	\choice0 全面协调可持续
\end{choices}
12. 社会主义新农村建设的中心环节是
\begin{choices}
	\choice0 生产发展
	\choice0 生活宽裕
	\choice0 乡风文明
	\choice0 管理民主
\end{choices}
13. 马克思主义中国化理论成果的精髓是
\begin{choices}
	\choice0 理论联系实际
	\choice0 解放思想
	\choice0 实事求是
	\choice0 与时俱进
\end{choices}
14. 2008年5月28日,中共中央总书记胡锦涛和中国国民党主席吴伯雄在北京人民大会堂举行了两党在新形势下的首次会谈,此次会谈
\begin{choices}
	\choice0 就促进两岸关系改善和发展达成广泛共识
	\choice0 开启了国共两党对话先声
	\choice0 发布了“两岸和平发展共同愿景”
	\choice0 签署了《海峡两岸包机会谈纪要》
\end{choices}
15. 2008年9月25日,我国“神舟七号”航天飞船成功飞入太空,首次实现了
\begin{choices}
	\choice0 载人飞行
	\choice0 绕月探测
	\choice0 天地对话
	\choice0 出舱活动
\end{choices}
16. 在2008年4月中旬举行的尼泊尔制宪会议选举上,一举成为第一大党的是
\begin{choices}
	\choice0 尼泊尔共产党(联合马列)
	\choice0 尼泊尔共产党(毛主义)
	\choice0 尼泊尔大会党
	\choice0 尼泊尔民族主义党
\end{choices}
\vspace{6pt}
