17. 近一年多来由美国次贷危机引发的金融危机,迅速在全球蔓延。在危机面前,人们应该积极主动应对,化“危”为“机”。下列名言中符合意识能动性原理的有
\begin{choices}
	\choice0 信心比黄金更重要
	\choice0 我们唯一恐惧的就是恐惧本身
	\choice0 问题与解决问题的方法是同时产生的
	\choice0 事不避难,知难不难
\end{choices}
18. 邓小平说:“农村搞家庭联产承包,这个发明权是农民的,农村改革中的好多东西,都是基层创造出来的,我们把它拿来加工提高作为全国的指导。”这对我们实现思想理论创新具有普遍指导意义,它要求我们
\begin{choices}
	\choice0 要以解放思想为先导
	\choice0 打破一切理论的约束
	\choice0 关注生活实践的需要
	\choice0 尊重人民群众的诉求
\end{choices}
19. “随着新生产力的获得……人们也就会改变自己的一切社会关系,手推磨产生的是封建主义的社会,蒸汽磨产生的是工业资本家的社会。”这段话表明科学技术是
\begin{choices}
	\choice0 历史上起推动作用的革命力量
	\choice0 历史变革中的唯一决定性力量
	\choice0 推动生产方式变革的重要力量
	\choice0 一切社会变革中的自主性力量
\end{choices}
20. 华罗庚生前曾说:“我们最好把自己的生命看作是前人生命的延续,是现在人类共同的生命的一部分,同时也是后人生命的开端。如此延续下去,科学就会一天比一天更灿烂,社会就会一天比一天更美好。”这段话对我们如何实现人的个人价值的教益是
\begin{choices}
	\choice0 个人价值的实现与社会价值的实现是统一的
	\choice0 个人价值的实现是一个历史过程
	\choice0 个人价值的实现是社会价值实现的归宿
	\choice0 个人价值的实现和个人生命的长短相一致
\end{choices}
21. “信用制度加速了生产力的物质上的发展和世界市场的形成;使这二者作为新生产形式的物质基础发展到一定的高度,是资本主义生产方式的历史使命。同时,信用制度加速了这种矛盾的爆发,即危机,因而加强了旧生产方式的解体的各种因素。”马克思的这一论述表明,资本主义信用制度
\begin{choices}
	\choice0 已成为资本主义经济危机爆发的深层原因
	\choice0 促进了建立社会主义生产方式的物质基础的形成
	\choice0 加速了资本主义生产方式内部矛盾发展和解体要素的形成
	\choice0 既推动商品经济的发展,又加深了商品经济运行中的矛盾
\end{choices}
22. 劳动力是任何社会生产的基本要素,在特定的社会发展阶段和特定的历史条件下,劳动力作为一种特殊商品,其价值的构成包括
\begin{choices}
	\choice0 维持劳动者自身生存所必需的生活资料的价值
	\choice0 劳动者在必要劳动时间内创造的价值
	\choice0 劳动者繁育后代所必须的生活资料的价值
	\choice0 培养和训练劳动者所需要的费用
\end{choices}
23. 党的十七届三中全会通过的《中共中央关于推进农村改革发展若干重大问题的决定》指出:“建立健全土地承包经营权流转市场,按照依法自愿有偿原则,允许农民以转包、出租、互换、转让、股份合作等形式流转土地承包经营权,发展多种形式的适度规模经营。” 上述决定有利于
\begin{choices}
	\choice0 调整农村土地所有制结构
	\choice0 完善土地承包经营权权能
	\choice0 进一步完善生产要素市场
	\choice0 促进土地资源的优化配置
\end{choices}
24. 合理的收入分配制度是社会公平的重要体现。在构建社会主义和谐社会过程中,初次分配和再分配都要处理好效率和公平的关系,再分配更加注重公平,逐步提高居民收入在国民收入分配中的比重,提高劳动报酬在初次分配中的比重。这表明处理好效率与公平的关系,就要
\begin{choices}
	\choice0 把效率和公平相互之间的矛盾协调统一起来
	\choice0 充分发挥市场机制对收入分配的调节作用
	\choice0 改革现有的收入分配制度,规范收入分配秩序
	\choice0 合理调节国民收入分配格局,加大收入分配调节力度
\end{choices}
25. 1921年中国共产党的成立,是中国革命历史上划时代的里程碑,中国革命的面目焕然一新。从此中国革命有了
\begin{choices}
	\choice0 正确的革命道路
	\choice0 科学的指导思想
	\choice0 坚强的领导力量
	\choice0 崭新的奋斗目标
\end{choices}
26. 新民主主义的文化,是民族的科学的大众的文化。其中“民族的”是指
\begin{choices}
	\choice0 反对外来的资本主义文化
	\choice0 反对帝国主义压迫,主张中华民族的尊严和独立
	\choice0 在形式和内容上有中国作风和中国气派
	\choice0 为全民族90\%以上的工农大众服务
\end{choices}
27. 在民主革命和社会主义革命的关系问题上,中国共产党内曾经出现过不同的观点和主张,其中错误的有
\begin{choices}
	\choice0 “毕其功于一役”
	\choice0 “二次革命论”
	\choice0 “无间断”革命
	\choice0 中国革命分“两步走”
\end{choices}
28. 20世纪50年代中期,社会主义改造基本完成,标志着
\begin{choices}
	\choice0 社会主义制度在我国已经确立
	\choice0 我国进入了社会主义初级阶段
	\choice0 我国步入了社会主义改革时期
	\choice0 我国实现了新民主主义向社会主义过渡
\end{choices}
29. 我们所要建设的社会主义和谐社会,应该是民主法制、公平主义、诚信友爱、充满活力、安定有序,人与自然和谐相处的社会,其中“诚信友爱”的内涵包括
\begin{choices}
	\choice0 全社会管理完善,秩序良好
	\choice0 全社会互帮互助,诚实守信
	\choice0 全体人民生活富裕,安居乐业
	\choice0 全体人民平等友爱,融洽相处
\end{choices}
30. 基层群众自治制度是我国政治制度体系中的重要组成部分,其主要内容有
\begin{choices}
	\choice0 农村村民委员会
	\choice0 城市居民委员会
	\choice0 企业职工代表大会
	\choice0 妇女联合会
\end{choices}
31. 2008年5月12日,我国发生了震惊世界的四川汶川特大地震。在这次抗震救灾中,全党全军全国人民在党中央国务院领导下众志成城,坚持把抢救人的生命放在第一位,只要有一线希望就尽百倍努力,84017名群众被从废墟中抢救出来,140万名被困群众得到解救,430多万名伤病员得到及时救治,其中1万多名伤员被快速转送到20个省区市375家医院。这些事实生动体现了
\begin{choices}
	\choice0 我国社会主义制度珍爱生命,保护人民的性质
	\choice0 中华民族关爱生命,崇尚理性的民族品格
	\choice0 党和政府为人民服务的根本宗旨
	\choice0 社会主义核心价值体系建设的显著成效
\end{choices}
32. 2008年6月20号,胡锦涛同志到《人民日报》社考察工作,并在线与网民直接交流,表达了党和政府对网络民意的高度重视。近年来,越来越多的政府官员上网收集民意。这意味着网络表达已成为
\begin{choices}
	\choice0 公民政治参与的新途径
	\choice0 反腐倡廉的新通道
	\choice0 民主政治体制的新形式
	\choice0 密切干群关系的新方式
\end{choices}
33. 2008年8月8日至24日,第29届奥林匹克运动会在北京成功举行。中国政府和人民认真履行了对国际社会的郑重承诺。中国提出的奥运理念丰富了奥林匹克精神,彰显了中国和世界在追求人类共同进步中坚守的共同梦想。中国提出的本届奥运会理念是
\begin{choices}
	\choice0 “平安奥运”
	\choice0 “绿色奥运”
	\choice0 “科技奥运”
	\choice0 “人文奥运”
\end{choices}
\vspace{6pt}
