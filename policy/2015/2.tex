17. 母质、气候、生物、地形、时间是土壤形成的五大关键成土因素。母质是土壤形成物质基础的初始无机养分的最初来源。气候导致矿物的风化和合成,有机质的形成和积累,土壤中物质的迁移,分解和合成。生物包括植物、动物和微生物等,是促进土壤发生发展最活跃的因素。地形可以使物质在地表进行再分配,使土壤及母质在接受光、热、水等条件方面发生差异。时间是阐明土壤形成发展的历史动态过程,母质、气候、生物和地形等对成工过程的作用随着时间延续而加强。土壤的生产过程说明
\begin{choices}
	\choice0 事物总是作为过程而存在
	\choice0 时间是物质运动的存在形式
	\choice0 事物的发展总是呈现出线性上升的态势
	\choice0 事物的产生是多重因素相互作用的结果
\end{choices}
18. 平衡是事物发展的一种状态,小到体操中人在平衡木上的行走,杂技中的骑车走钢丝,独轮车表演,直升机在空中的悬停,大到人类的生存,地球的运转,天体的运行等等,都是保持平衡的一种状态,世间的万事万物,之所以能不停的运动、发展、前进,一个重要原因就在于保持了平衡,要使“平衡”成为人们的“大智慧”,就要
\begin{choices}
	\choice0 精确把握事物的度
	\choice0 准确掌握辩证否定的方式和方向
	\choice0 善于协调事物内部各种因素的相互关系
	\choice0 全面理解绝对运动和相对静止的辩证关系
\end{choices}
19. 马克思主义从必然性和偶然性的辩证统一中理解杰出人物的历史作用,认为
\begin{choices}
	\choice0 杰出人物能够改变历史发展的基本方向
	\choice0 杰出人物的历史作用受到一定历史条件的制约
	\choice0 杰出人物历史作用的形成和发挥与其服务人民群众的意愿,密不可分
	\choice0 杰出人物们因其智慧性格因素对社会进程发生影响
\end{choices}
20. 马克思指出:“资本主义积累不断地并且同它的能力和规模成比例地生产出相对的,即超过资本增殖的平均需要的,因而是过剩的或追加的工人人口。”“过剩的工人人口是积累或资本主义基础上的财富发展的必然产物,但是这种过剩人口反过来又成为资本主义积累的杠杆,甚至成为资本主义生产方式存在的一个条件。”上述论断表明
\begin{choices}
	\choice0 资本主义积累必然导致工人人口的供给相对于资本的需要而过剩
	\choice0 资本主义社会过剩人口之所以是相对的,是因为它不为资本价值增殖所需要
	\choice0 资本主义生产周期性特征需要有相对过剩的人口规律与之相适应
	\choice0 资本主义积累使得资本主义社会的人口失业规模呈现越来越大的趋势
\end{choices}
21. 国家垄断资本主义是国家政权和私人垄断资本融合在一起的垄断资本主义。第二次世界大战结束以来,在国家垄断资本主义获得充分发展的同时,资本主义国家通过宏观调节和微观规制对生产\\流通\\分配和消费各个环节的干预也更加深入。其中,微观规制的类型主要有
\begin{choices}
	\choice0 公众生活规制
	\choice0 公共事业规制
	\choice0 社会经济规制
	\choice0 反托拉斯法
\end{choices}
22. 2014年5月22日,习近平在上海召开的外国专家座谈会上指出,“要实行更加开放的人才政策,不唯地域引进人才,不求所有开发人才,不拘一格用好人才。”当前,我们之所以比历史上任何时期都更加强调重视人才\\用好人才,是因为人才是
\begin{choices}
	\choice0 国家竞争力的重要体现
	\choice0 先进生产力的集中体现
	\choice0 第一资源和国家战略资源
	\choice0 推进中国特色社会主义事业的关键因素
\end{choices}
23. 到2012年底,我国仍在耕地上从事农业生产经营的农民家庭约1.9亿户,所经营的耕地面积占农村家庭承包耕地总面积的92.5\%:仍实行由集体统一经营的村,组约有2000个;已发展起农业专业合作社68.9万个,入社成员5300多万户;各类农业产业化经营组织30余万个,带动的农户约1.18亿户;约有2556万亩耕地由企业租赁经营,上述现象表明在我国农村
\begin{choices}
	\choice0 农业经营主体呈现多样化趋势
	\choice0 土地的集体所有权性质已经发生变化
	\choice0 家庭承包经营仍然是最基本的经营形式
	\choice0 土地经营权的流转使农民失去了对土地的承包权
\end{choices}
24. 与十一届全国人民代表大会相比,十二届人代会在代表结构与组成上,呈现"两升一降"的变化,来自一线的工人、农民代表401名,占代表总数的13.42\%,提高了5.18个百分点;专业技术人员代表610名,占代表总数的20.42\%,提高了1.2个百分点;党政领导干部代表1042名,占代表总数的34.88\%,降低了6.93个百分点。提高基层人大代表特别是一线工人,农民,知识分子代表比例,降低党政领导干部代表比例,有利于
\begin{choices}
	\choice0 推动人民群众最关心最直接最现实问题的解决
	\choice0 调动基层群众参政议政的积极性与主动性
	\choice0 保证人民群众直接参加国家管理
	\choice0 更为充分地发挥全国人大的民意反映与监督职能
\end{choices}
25. 国家统计局发布的最新数据显示,2014年前三季度我国GDP增长为7.4\%,其中第三季度增长为7.3\%,创下了2009年一季度以来的新低。总体上看,虽然经济增长有所放缓,但国民经济继续运行在合理区间,稳中有进的态势没有变,今后一个时期经济保持平稳较快发展的可能性仍然较大。这是一种趋势性的变化,是经济到了新的发展阶段表现出来的一种新常态。我国经济新常态的主要特点是
\begin{choices}
	\choice0 经济増长速度从高速增长转为中高速增长
	\choice0 中国经济对世界市场的需求减弱
	\choice0 经济结构不断优化升级
	\choice0 经济发展动力从要素驱动,投资驱动转向创新驱动
\end{choices}
26. 人类社会发展的历史表明,对一个民族,一个国家来说,最持久、最深层次的力量是全社会共同认可的核心价值观。面对世界范围思想文化交流交融交锋形势下价值观的新态势,面对改革开放和发展社会主义市场经济条件下思想意识多元多样多变的新特点,积极培育和践行社会主义核心价值观,有利于
\begin{choices}
	\choice0 促进人的全面发展和引领社会全面进步
	\choice0 巩固全党全国人民团结奋斗的共同思想基础
	\choice0 巩固马克思主义在意识形态领域的指导地位
	\choice0 集聚实现中华民族伟大复兴中国梦的强大正能量
\end{choices}
27. 甲午,对中国人民和中华民族具有特殊含义,在我国近代史上也具有特殊含义。1894年7月,日本发动甲午战争,清朝在战争中战败。这场战争对中国的影响主要有:
\begin{choices}
	\choice0 中国海关的行政权落入外国人手中
	\choice0 中国人开始有了普遍的民族意识觉醒
	\choice0 台湾被日本侵占
	\choice0 帝国主义列强掀起瓜分中国的狂潮
\end{choices}
28. 1946年1月10日,政治协商会议在重庆开幕,出席会议的有国民党,共产党,民主同盟,青年党和无党派人士的代表38人。会议通过了宪法草案,政府组织案,国民大全案,和平建国纲领,军事问题案五项协议。按照协议规定建立的政治体制相当于英国、法国的议会制和内阁制,不是新民主义性质的而且国民党在政府中占着明显的优势。对政协的上述协议,共产党赞同并决心严格履行,这是因为它有利于:
\begin{choices}
	\choice0 推进民主政治的发展和进步
	\choice0 打破国民党一党独裁的局面
	\choice0 改变国共两党军事力量对比
	\choice0 保障解放区政权的合法地位
\end{choices}
29. 1979年,针对当时存在的是否还要坚持毛泽东思想的问题,邓小平指出“有些同志说,我们只拥护‘正确的毛泽东思想’,而不是拥护错误的毛泽东思想。这种说法也是有错误的。”“这种说法”之所以错误,是因为
\begin{choices}
	\choice0 没有把毛泽东与党的其他领导人对毛泽东思想的贡献区分开
	\choice0 没有把毛泽东思想与有中国特色社会主义理论区分开
	\choice0 没有把毛泽东思想与毛泽东的思想区分开
	\choice0 没有把毛泽东晚年的错误与毛泽东思想的科学体系区分开
\end{choices}
30. 个人品德是通过社会道德教育和个人自觉的道德修养所形成的稳定的心理状态和行为习惯。它是个体对某种道德要求认同和践履的结果,集中体现了道德认知、道德情感、道德意志和道德行为的内在统一,这表明个人品德是
\begin{choices}
	\choice0 个人行为的统一整体及知、情、意、行的综合体现
	\choice0 在实践活动中表现出来的行为的稳定倾向
	\choice0 在实践活动中锻炼而成的一种特殊品性
	\choice0 偶然的、短暂的道德行为现象
\end{choices}
31. 1763年,老威廉·皮特在《论英国人个人居家安全的权利》的演讲中说:“即使最穷的人,在他的小屋里也能够对抗国王的权威。屋子可能很破旧,屋顶可能摇摇欲坠;风可以吹进这所房子,雨可以淋进这所房子,但是国王不能踏进这所房子,他的千军万马也不敢跨进这件破房子的门槛。”这段话后来被浓缩成“风能进,雨能进,国王不能进”。这凸显了权力与权利的关系是
\begin{choices}
	\choice0 权力优先于权利
	\choice0 权力决定权利
	\choice0 权力应当以权利为界限
	\choice0 权力必须受到权利的制约
\end{choices}
32. 2014年2月27日,十二届全国人大常委会第七次会议通过决定,将9月3日确定为中国人民抗日战争胜利纪念日,将12月13日确定为南京大屠杀死难者国家公祭日,设立这两个纪念日:
\begin{choices}
	\choice0 有助于向中国人民和世界各国人民传播历史事实的真相
	\choice0 是对抗击日本帝国主义侵略付出巨大牺牲和作出巨大贡献的人们的敬重与缅怀
	\choice0 彰显了中国作为反法西斯主要战场的伟大作用
	\choice0 是对南京大屠杀中大量死难同胞的告慰和尊重
\end{choices}
33. 2014年11月5日至11日,亚太经济合作组织(APEC)第二十二次领导人非正式会议在北京召开。这是一次开创性的历史盛会,硕果累累,其中,《北京反腐败宣言》的通过尤为引人注目。该《宣言》通过的意义在于,各成员国
\begin{choices}
	\choice0 加强了涉腐、涉案赃款跨境流动的信息共享
	\choice0 将形成携手打击跨境腐败的网络
	\choice0 杜绝了跨国腐败行为的发生
	\choice0 达成了就追逃、追赃开展执法合作的重要共识
\end{choices}
\vspace{6pt}
