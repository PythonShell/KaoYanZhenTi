1. 中国工程院院士袁隆平曾结合自己的科研经历,语重心长地对年轻人说:“书本知识非常重要,电脑技术也很重要,但是书本电脑里种不出水稻来,只有在田里才能种出水稻来。”这表明
\begin{choices}
	\choice0 实践水平的提高有赖于认识水平的提高
	\choice0 实践是人类认识的基础和来源
	\choice0 理论对实践的指导作用没有正确与错误之分
	\choice0 由实践到认识的第一次飞跃比认识到实践的第二次飞跃更加重要
\end{choices}

2. 社会存在决定社会意识,社会意识是社会存在的反映。社会意识具有相对独立性,即它在反映社会存在的同时,还有自己特有的发展形式和规律。社会意识相对独立性最突出的表现是
\begin{choices}
	\choice0 社会意识内部各种形式之间的相互作用和影响
	\choice0 社会意识与社会存在发展的不完全同步性
	\choice0 社会意识各种形式各自具有其历史继承性
	\choice0 社会意识对社会存在具有能动的反作用
\end{choices}

3. 第二次世界大战以后,资本主义国家经历了第三次科技革命,机器大工业发展到自动化阶段。只能化工厂创造出了较高的生产效率,显露出巨大的竞争力。企业在"机器换人"中取得了一定的经济效益。这意味着率先使用机器人的个别企业
\begin{choices}
	\choice0 剩余价值来源的改变
	\choice0 资本技术构成的提高
	\choice0 所生产商品价值提高
	\choice0 获得更多的社会平均利润
\end{choices}
4. 马克思主义政党是科学社会主义与工人运动相结合的产物,是工人阶级的先锋队,这表明
\begin{choices}
	\choice0 马克思主义政党以工人阶级为基础
	\choice0 马克思主义政党即工人阶级本身
	\choice0 马克思主义政党的阶级性是其先进性的根本前提
	\choice0 马克思主义政党的先进性决定了工人阶级的先进性
\end{choices}
5. 新中国的工业化是在苏联的影响下起步的。走中国工业化道路,是中国共产党初步探索我国社会主义建设道路的一个重要思想。当时所讲的工业化道路问题,主要是指
\begin{choices}
	\choice0 经济建设和国防建设的关系问题
	\choice0 中央和地方的关系问题
	\choice0 沿海工业和内地工业的关系问题
	\choice0 重工业、轻工业和农业的发展关系问题
\end{choices}
6. 党的十三大召开前夕,邓小平强调指出:“社会主义本身是共产主义的初阶段,而我们中国又处在社会主义的初级阶段,就是不发达的阶段。一切都要从这个实际出发,根据这个实际来制定规划。”这一论述
\begin{choices}
	\choice0 首次提出了社会主义初级阶段概念
	\choice0 首次系统阐述了社会主义初级阶段理论
	\choice0 首次把社会主义初级阶段作为事关全局的基本国情加以把握
	\choice0 首次对社会主义发展阶段进行了划分
\end{choices}
7. 城镇化是现代化的必由之路,解决好人的问题是推进城镇化的关键。当前,我国实现城镇化的首要任务是
\begin{choices}
	\choice0 推进农业转移人口市民化
	\choice0 使土地的城镇化优于人口的城镇化
	\choice0 促进农村劳动力向非农产业转移
	\choice0 实现“人的无差别发展”
\end{choices}
8. 1997年7月1日,中国政府对香港恢复行使主权,香港特别行政区成立,香港特别行政区基本法开始实施。香港进入“一国两制”、“港人治港”、高度自治的历史新纪元。香港特别行政区的高度自治权是
\begin{choices}
	\choice0 特别行政区的完全自治
	\choice0 中央授权之外的剩余权力
	\choice0 特别行政区本身固有的权力
	\choice0 中央授予的地方事务管理权
\end{choices}
9. 近代中国,一些爱国人士提出过工业救国、教育救国、科学救国等主张,并为此进行过努力,但这些主张并不能从根本上给濒临危亡的中国指明正确的出路,这是因为他们没有认识到
\begin{choices}
	\choice0 争取民族独立和人民解放是实现民族复兴的前提
	\choice0 中国已经被卷入世界资本主义经济体系和世界市场中
	\choice0 中国是一个经济政治发展不平衡的国家
	\choice0 资本主义制度已经过时
\end{choices}
10. 毛泽东在谈到辛亥革命时指出,辛亥革命有它胜利的地方,也有它失败的地方,“辛亥革命把皇帝赶跑,这不是胜利了吗?说它失败,是说辛亥革命只把一个皇帝赶跑。”毛泽东这里所说的“只把一个皇帝赶跑”是指
\begin{choices}
	\choice0 没有推翻帝制
	\choice0 反帝反封建的革命任务没有完成
	\choice0 孙中山没有继续革命
	\choice0 袁世凯窃取了胜利果实
\end{choices}
11. 1914年至1918年的第一次世界大战,是一场空前残酷的大屠杀.它改变了世界政治的格局,也改变了各帝国主义国家在中国的利益格局,对中国产生了巨大的影响.大战使中国的先进分子
\begin{choices}
	\choice0 对中国传统文化产生怀疑
	\choice0 对西方资产阶级民主主义产生怀疑
	\choice0 认识到工人阶级的重要作用
	\choice0 认识到必须优先改造国民性
\end{choices}
12. 1929年12月下旬,红四军党的第九次代表大会在福建上杭县古田村召开,会议总结了红军创立以来的经验,通过了著名的古田会议决议,决议的中心思想是
\begin{choices}
	\choice0 中国共产党必须服从共产国际的领导
	\choice0 武装斗争是中国革命的主要形式
	\choice0 在农村根据地广泛开展土地革命
	\choice0 用无产阶级思想进行军队和党的建设
\end{choices}
13. 习近平在欧美同学会成立100周年庆祝大会上的讲话中说:“希望广大留学人员继承和发扬留学报国的光荣传统,做爱国主义的坚守者和传播者,秉持‘先天下之忧而忧,后天下之乐而乐’的人生理想,始终把国家富强、民族振兴、人民幸福作为努力志向,自觉使个人成功的果实结在爱国主义这颗常青树上。”个人成功的果实之所以应该结在爱国主义这棵常青树上,是因为爱国主义是
\begin{choices}
	\choice0 个人实现人生价值的力量源泉
	\choice0 人人实现人生价值的直接条件
	\choice0 个人成功的根本保障
	\choice0 个人成功的决定性因素
\end{choices}
14. 钱学森曾经说过:“我作为一名中国的科技工作者,活着的目的就是为人民服务。如果人民最后对我的一生所做的工作表示满意的话,那才是最高的奖赏。”这说明评价人生价值的根本尺度是
\begin{choices}
	\choice0 个体在社会中的地位
	\choice0 个体在社会中的影响
	\choice0 个人对社会和他人的生存和发展的贡献
	\choice0 个体从社会获得的满足尺度
\end{choices}
15. 2014年4月15日,中共中央总书记、中央国家安全委员会主席习近平主持召开中央国家安全委员会第一次会议并发表重要讲话,他强调,面对传统安全威胁与非传统安全威胁交织的局面,要准确把握国家安全形式变化新特点新形势,坚持
\begin{choices}
	\choice0 共同安全观
	\choice0 亚洲安全观
	\choice0 总体国家安全观
	\choice0 地区集体安全观
\end{choices}
16. 近年来,中东地区局势持续动荡,恐怖主义、分离主义愈加猖獗,教派矛盾不断升级。尤其是极端恐怖势力于2014年6月29日宣布成立“伊斯兰国”(ISIS),并宣称将建立地跨西亚北非的“哈里发帝国”,对该地区的秩序造成了重大冲击,并且给国家的全球战略带来了影响,这种影响表现为美国
\begin{choices}
	\choice0 “中东收缩”战略提上议程
	\choice0 “和平演变”战略归于失败
	\choice0 “北约东扩”战略被迫搁置
	\choice0 “重返亚太”战略收到牵制
\end{choices}
\vspace{6pt}
