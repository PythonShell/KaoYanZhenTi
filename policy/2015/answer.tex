%!Tex Program = xelatex
\documentclass[a4paper]{article}

% 设置页边距
\usepackage{geometry}
\geometry{left=2cm, right=2cm, top=2.5cm, bottom=1cm}

% 中文断行
\XeTeXlinebreaklocale "zh"
\XeTeXlinebreakskip = 0pt plus 1pt

%% code include by PythonShell
%% From http://tex.stackexchange.com/questions/140923/how-to-automatically-align-the-four-choices-of-a-multiple-choice-question-in-exa
%% thanks to the author ollydbg23 @ stackexchange
%% some tiny modifies from PythonShell
%% 2014-01-08

\usepackage{ifthen}
\usepackage{calc}
\setlength\parindent{0pt}

    %usage \choice{ }{ }{ }{ }
    %(A)(B)(C)(D)
    \newcommand{\fourch}[4]{
	%\par
            \begin{tabular}{*{4}{@{}p{0.23\textwidth}}}
            [A] ~#1 & [B] ~#2 & [C] ~#3 & [D] ~#4
            \end{tabular}
    }

    %(A)(B)
    %(C)(D)
    \newcommand{\twoch}[4]{
	%\par
            \begin{tabular}{*{2}{@{}p{0.46\textwidth}}}
            [A] ~#1 & [B] ~#2
            \end{tabular}
    \par
            \begin{tabular}{*{2}{@{}p{0.46\textwidth}}}
            [C] ~#3 & [D] ~#4
            \end{tabular}
    }

    %(A)
    %(B)
    %(C)
    %(D)
    \newcommand{\onech}[4]{
	%\par
            [A] ~#1 \par [B] ~#2 \par [C] ~#3 \par [D] ~#4
    }

    \newlength\widthcha
    \newlength\widthchb
    \newlength\widthchc
    \newlength\widthchd
    \newlength\widthch
    \newlength\tabmaxwidth

    \setlength\tabmaxwidth{0.96\textwidth}
    \newlength\fourthtabwidth
    \setlength\fourthtabwidth{0.24\textwidth}
    \newlength\halftabwidth
    \setlength\halftabwidth{0.48\textwidth}

    \newcommand{\choice}[4]{
            \settowidth\widthcha{AM.#1}\setlength{\widthch}{\widthcha}
            \settowidth\widthchb{BM.#2}    
            \ifthenelse{\widthch<\widthchb}{\setlength{\widthch}{\widthchb}}{}
            \settowidth\widthchb{CM.#3}    
            \ifthenelse{\widthch<\widthchb}{\setlength{\widthch}{\widthchb}}{}
            \settowidth\widthchb{DM.#4}    
            \ifthenelse{\widthch<\widthchb}{\setlength{\widthch}{\widthchb}}{}     
            \ifthenelse{\widthch<\fourthtabwidth}{\fourch{#1}{#2}{#3}{#4}}
                               {\ifthenelse{\widthch<\halftabwidth\and\widthch>\fourthtabwidth}{\twoch{#1}{#2}{#3}{#4}}
                               {\onech{#1}{#2}{#3}{#4}}}
    }

\usepackage{fontspec}
\setmainfont{SimSun}	% 设置正文默认字体为SimSun

\newcommand\fontnamekai{楷体}	% 设置楷体
\newfontinstance\KAI {\fontnamekai}
\newcommand{\kai}[1]{{\KAI#1}}

\newcommand\fontnamehei{黑体}	% 设置黑体
\newfontinstance\HEI{\fontnamehei}  
\newcommand{\hei}[1]{{\HEI#1}} 

% 设置页眉
\pagestyle{myheadings}
\markright{2015年考研政治答案——PythonShell 工作室}

% 取消缩进
\setlength{\parindent}{0pt}

\begin{document}
\begin{tabbing}
一、单选题\\
\= 01. B \qquad \= 02. D \qquad \= 03. B \qquad \= 04. A \qquad \= 05. D \qquad \= 06. C \qquad \= 07. A \qquad \= 08. D \qquad \= \\
\> 09. A \> 10. B \> 11. B \> 12. D \> 13. A \> 14. C \> 15. C \> 16. A \\
二、多选题\\
\> 17. ABC \> 18. ABCD \> 19. BCD \> 20. AD \> 21. BCD \> 22. ACD \> 23. AC \> 24. ABD\\
\> 25. ACD \> 26. ABD \> 27. BCD \> 28. AB \> 29. ACD \> 30. ABC \> 31. CD \> 32. ABD \> 33. ABD\\
\end{tabbing}

34.

(1)第一,马克思主义哲学认为,意识的能动作用是人的意识所特有的积极反映世界与改造世界的能力和活动。在矛盾群中又存在着根本矛盾和非根本矛盾、主要矛盾和次要矛盾。在每一对矛盾中又有矛盾的主要方面与矛盾的次要方面,所以我们必须坚持"两点论"与"重点论"相结合的方法,抓关键、看主流的方法,互联网是一把双刃剑,既有积极作用,又有消极作用,我们要发挥主观能动性,趋利避害,抓住其积极的作用,避免其消极作用。

第二,作为当代大学生,我们要发挥主观能动性,充分利用互联网的积极方面,为我们的学习、生活服务。

(2)第一,唯物辩证法认为,同一性是指矛盾双方相互联系、相互吸引的性质和趋势,矛盾的斗争性是指对立面之间相互排斥、相互分离的趋势和关系。矛盾的同一性在事物发展具有重要作用,由于矛盾双方相互依存,互为存在的条件,矛盾双方可以利用对方的发展使自己得到发展。由于矛盾双方相互包含,矛盾双方可以相互吸取有利于自身的因素而得到发展。要求我们在对立中把握同一与在同一中把握对立。

第二,在互联网高速发展的时代,在“大象”和“蚂蚁”的竞争中,“大象”要学习“蚂蚁”军团灵活的作战特点,创造条件,使自己的劣势转变为优势,通过竞争使“大象”和“蚂蚁”同时得到发展。

35.

(1)根据材料可知“大猫”是对公有制经济的形象比喻,“小猫”是对非公有制经济的形象比喻。现阶段我国发展社会主义市场经济要坚持“大猫小猫都有路走”的原因:

第一,我国是社会主义国家,必须坚持公有制作为社会主义经济制度的基础。

第二,我国还处在社会主义初级阶段,生产力还不够发达,发展也很不平衡,需要在公有制为主体的条件下发展多种所有制经济。

第三,一切符合“三个有利于”标准的所有制形式,都可以而且应该用来为发展社会主义服务。公有制为主体、多种所有制经济共同发展的基本经济制度是中国特色社会主义制度的重要支柱,也是社会主义市场经济体制的根基。公有制经济和非公有制经济都是社会主义市场经济的重要组成部分,都是我国经济社会发展的重要基础。

(2)更好发挥非公有制经济在经济发展中的作用,鼓励、支持、引导非公有制经济发展:

第一,要坚持权利平等、机会平等、规则平等,废除对非公有制经济各种形式的不合理规定,消除各种隐性壁垒,制定非公有制企业进入特许经营领域具体办法,保证各种所有制经济依法平等使用生产要素、公开公平公正参与市场竞争。

第二,健全归属清晰、权责明确、保护严格、流转顺畅的现代产权制度;坚持平等保护物权,公有制经济财产权不可侵犯,非公有制经济财产权同样不可侵犯;完善产权保护制度,国家保护各种所有制经济产权和合法利益,同等受到法律保护,依法受到监管。

第三,建立适合于家族企业的现代企业治理结构和机制,要引导和鼓励有条件的私营企业利用产权市场,引进国有资本或其他社会资本,改善企业股权结构。鼓励发展非公有资本控股的混合所有制企业。鼓励有条件的私营企业建立现代企业制度,促进非公有制经济健康发展。

第四,坚持和完善基本经济制度,要积极发展混合所有制经济。要允许更多国有经济和其他所有制经济发展成为混合所有制经济,国有资本投资项目允许非国有资本参股。允许混合所有制经济实行企业员工持股,形成资本所有者和劳动者利益共同体。

36.

(1)近代以来,为改变中华民族的命运,中国人民和无数仁人志士进行了千辛万苦的探索和的不屈不挠的斗争。然而不触动封建根基的自强运动和改良主义,旧式的农民战争,资产阶级革命派领导的革命,照搬西方资本主义的其他种种方案,学习西方国家的各种思想文化,都不能完成中华民族救亡图存的民族使命和反帝反封建的历史任务。无产阶级之前的各个阶级或阶层探索之所以失败的原因之一就是没有找到科学的理论作指导。因此,争取民族独立、人民解放,实现国家富强、人民富裕,中国的发展进步,客观上要求有能够指导中国人民进行反帝反封建革命的先进理论即科学的马克思主义。

(2)根据材料我们可以看出孔子的传统儒家思想与马克思主义的“不同”与“一致”。具体内容如下:

“不同”:

孔子创立的儒家学说以及在此基础上发展起来的儒家思想,对中华文明产生了深刻影响,是中国传统文化的重要组成部分。在分配上认为不患寡而患不均,不患贫而患不安。马克思主义是无产阶级思想的科学体系,它始终严格地以客观事实为根据,深刻揭示了人类社会发展规律,坚定维护和发展最广大人民根本利益,是指引人民推动社会进步、创造美好生活的科学理论。实现物质财富极大丰富、人民精神境界极大提高、每个人自由而全面发展的共产主义社会,是马克思主义最崇高的社会理想。在分配上认为,患寡且患不均,患贫且患不安。在共产主义社会,个人消费品的分配方式是"各尽所能,按需分配"。共产主义社会不仅社会是和谐的,而且社会与自然之间也达成了和谐。在共产主义社会,人的发展是自由的发展,是建立在个体高度自由自觉基础上的发展,而不是强迫的发展。

“一致”:

儒家思想和马克思主义在内容上实质上是一致的,孔子的理想社会和大同世界与马克思的共产主义理想社会是不谋而合,比如,孔子关于天下为公、大同世界的思想,关于以民为本、安民富民乐民的思想,与马克思的每个人自由而全面发展的思想一致。

37.

(1)第一,我国正处于社会主义初级阶段,全面建成小康社会进入决定性阶段,改革进入攻坚期和深水区,国际形势复杂多变,我们党面对的改革发展稳定任务之重前所未有、矛盾风险挑战之多前所未有,依法治国在党和国家工作全局中的地位更加突出、作用更加重大。

第二,全面推进依法治国是关系我们党执政兴国、关系人民幸福安康、关系党和国家长治久安的重大战略问题,是完善和发展中国特色社会主义制度、推进国家治理体系和治理能力现代化的重要方面。

第三,我们要实现党的十八大和十八届三中全会作出的一系列战略部署,全面建成小康社会、实现中华民族伟大复兴的中国梦,全面深化改革、完善和发展中国特色社会主义制度,就必须在全面推进依法治国上作出总体部署、采取切实措施、迈出坚实步伐。

(2)第一,全面推进依法治国这件大事能不能办好,最关键的是方向是不是正确、政治保证是不是坚强有力,具体讲就是要坚持党的领导,坚持中国特色社会主义制度,贯彻中国特色社会主义法治理论。

第二,坚持党的领导,是社会主义法治的根本要求,是党和国家的根本所在、命脉所在,是全国各族人民的利益所系、幸福所系,是全面推进依法治国的题中应有之义。

第三,党的领导和社会主义法治是一致的,社会主义法治必须坚持党的领导,党的领导必须依靠社会主义法治。只有在党的领导下依法治国、厉行法治,人民当家作主才能充分实现,国家和社会生活法治化才能有序推进。

38.

(1)第一,蒙哥马利的思维依然是国强必霸的思维,这是世界历史发展过程中,大国崛起进程中的铁律。所谓“国强必霸”的逻辑既有悖于中国的历史,也有违中国人民的意志。

第二,中国主张走和平发展道路,是由中国的国情和自身发展需要决定的,是由中国历史文化传统所决定的,是由当今世界发展潮流所决定的。

(2)第一,中国梦是和平、发展、合作、共赢的梦,与世界各国人民的美好梦想相通。中国梦不仅造福中国人民,而且造福各国人民。

第二,中国一心一意办好自己的事情,永远不称霸、永远不搞扩张。

第三,中国坚持合作共赢,越来越多的国家正从中国的发展中受益。

第四,中国将进一步发挥负责任大国的作用,在力所能及的范围内承担更多国际责任和义务。

\end{document}
