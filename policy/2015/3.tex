\hei{34题、结合材料回答问题:}

材料1

\qquad \kai{2014年11月19日,首届世界互联网大会在浙江乌镇召开,习近平总书记在致大会的贺词中指出,互联网真正让世界变成了地球村,让国际社会越来越成为你中有我、我中有你的命运共同体、李克强总理20日下午在杭州会见出席大会的中外代表并同他们座谈,他表示,互联网是人类最伟大的发明之一,改变了人类世界的空间轴、时间轴和思想维度。中国接入互联网20年来,已发展成为世界互联网大国,不仅培育起一个巨大市场,也促生了许多新技术、新产品、新业态、新模式,创造了上千万就业岗位,很多人特别是年轻人,大学生因此实现了事业梦,人生梦。}

\qquad \kai{目前,全世界网民数量达到了30亿人,普及率达40\%,全球范围内实现了网络互联、信息互通。即使是世界上最偏僻的一角,只要接入互联网,就接入了人类这个大家庭。同住地球村的“居民”,借助互联网的力量极大的拉近了距离,互联经济已经成为世界经济发展速度最快、潜力最大、合作最活跃的领域之一,形成了世界网络大市场;一个短小的视频通过全世界网民的点击,可以一夜之间成为全球流行文化的宠儿;提供高速的移动通信和无线宽带服务,几乎已是各国旅游“设施”的标配。}

\qquad \kai{当然,互联网发展过程中也产生了一系类问题,如网络信息安全、网络犯罪等,甚至对国家主权、安全、发展利益形成了新的挑战。互联网到底是阿里巴巴的宝库,还是潘多拉的魔盒?这取决于“命运共同体”如何认真应对,谋求共治。从这样的视野来看,已走过20年岁月的中国互联网,站在了大有可为的新起点上。}

\begin{flushright}\kai{摘编自《人民日报》(2014年11月21日)}\end{flushright}

材料2

\qquad \kai{中国互联网从1994年全功能接入国际互联网至今,实现了20年的高速发展,不仅在技术层面一再突破,而且带来了新的思维理念,有人把他概括为“互联网思维”。对何谓“互联网思维”目前还没有定论。然而,打破思维定势,主动革新自我是互联网思维不变的主题,意味着“便捷、互动、用户至上”等理念。这些理念让人们不断感受到互联网带来的变化与变革。}

\qquad \kai{如果把沃尔玛等传统龙头企业比多大象,那么互联网上的小商户只能叫蚂蚁。数百万只蚂蚁聚合起来,吃掉大象并非没有可能。试想,如果没有互联网,任何一家传统商业企业要想把数百万个商家和近亿客户装进来是不可想象的。}

\qquad \kai{既然去中心的互联网更有利于“蚂蚁”生存,“大象”要彻底摆脱危机,可能就要让自己某种程度上也变成“蚂蚁”,至少自身要具备“蚂蚁”的特性。道理很简单,在互联网环境下与灵活的“蚂蚁军团”作战,庞大的体量以及传统组织形式带来的大企业病,很可能让“大象”的优势转变为劣势。只有彻底改变基因,让:“象”内部产生无数个热衷创业的“蚂蚁”,这仗才有的打,毕竟,和蚂蚁打仗,大象有力用不上,但更庞大、更强大的蚁群则成为最后的胜者。}

\begin{flushright}\kai{摘编自《人民日报》(2014年5月5日、5月26日等)}\end{flushright}

(1) 联系自身实际,分析为什么“互联网到底是阿里巴巴的宝库,还是潘多拉的魔盒”取决于互联网的“命运共同体”?(5分)

(2) 怎样以辩证的思维方式认识与处理“蚂蚁”与“大象”的关系?(5分)

\clearpage

\hei{35题、结合材料回答问题:}

让大猫小猫都有路走

\qquad \kai{计划经济时期全靠国家管理市场,市场边角被忽略,很多小商品没人去生产,有些新的市场需求也没人去注意。非公有制经济的特点是只要市场有需求,它就会去满足。要保证各种所有制经济依法平等使用生产要素,公平参与市场竞争,同等收到法律保护,就需要重视中小企业融资难的问题,经济学家成思危讲过这样一则寓言:著名科学家牛顿养了两只猫。一只大猫一只小猫。他在墙上开了两个洞,一个大洞。一个小洞,有人笑话他说,你还是大科学家呢,开一个洞就够了,小猫也可以走大洞嘛。牛顿说不对,如果两个猫同时要出去,那大洞肯定被大猫占住了,小猫就无路可走。要真正解决小微企业的问题,就要建立真正为小微企业服务的小型银行,让大银行服务大企业,小银行服务小企业。因为从商业角度说,大银行本身就嫌平爱富,嫌小爱大。小微企业市场风险很大,交易成本也高。跟大企业签一个1亿元的合同,相当于跟小企业签20个500万的合同。现在居民和企业手中有大量存款,而小微企业却贷不到款。这就需要一条通道,这条通道就是社区银行等中小银行。发展民营的、小型的金融机构有利于解决好小微企业的融资困难。大企业是我国经济的脊梁,小微企业是血肉。没有大企业国民经济站不起来,但是如果小微企业垮了,那国民经济不成了骨头架子了吗?十八届三中全会以来,国务院陆续出台了关于大力扶持小微企业健康发展的多项政策,国务院常务会议也多次强调要加快发展民营银行等中小金融机构,为小微企业减负添力。2014年11月,李克强总理在浙江考察时再次对民营银行长期致力于服务小微企业给与了充分肯定。为小微企业打开直接融资大门,是开创性的制度安排。一大批有巨大市场潜力的小微企业将会成长为支撑中国经济转型升级的参天大树。}

\begin{flushright}\kai{摘编自光明网(2013年11月15日)、新华网(2014年11月21日)}\end{flushright}

(1) 现阶段我国发展社会主义市场经济为什么应坚持"让大猫小猫都有路走"?(6分)

(2) 如何更好地发挥非公有制经济在经济发展中的作用?(4分)

\clearpage

\hei{36. 结合材料回答问题:}

1925年郭沫若在一篇文章中讲述了这样一个故事:

\qquad \kai{十月十五日丁祭过后的第二天,孔子和他的得意门生颜回、子路、子贡三位在上海的文庙里吃着冷肉的时候,有四位年轻的大班抬了一乘朱红漆的四轿,一直闯进庙来,里面走出一位脸如螃蟹,胡须满腮的西洋人来,原来这位胡子螃蟹脸就是马克思。}

\qquad \kai{孔子一见来的是马克思,他便禁不得惊喜着叫出:啊啊,有朋自远方来,不亦乐乎呀!你来到敝庙来,有什么见教呢?}

\qquad \kai{马克思说:我是特为领教而来,我们的主义已经传到你们中国,我希望在你们中国能够实现,但是近来有些人说,我的主义和你的思想不同,所以在你的思想普遍着的中国,我的主义是没有实现的可能性,因此我便来直接领教你:究竟你的思想是怎么样?和我的主义怎样不同?}

\qquad \kai{孔子说:难得你今天亲自到了我这里来,太匆促了,不好请你演讲,至少请你作一番谈话罢。你的理想的世界是怎样的呢?}

\qquad \kai{马克思说:我的理想的世界,是我们生存在这里面,万人要能和一人一样自由平等地发展他们的才能,人人都各能尽力做事而不望报酬,人人都各能得生活的保障而无饥寒的忧虑,这就是我所谓“各尽所能,各取所需”的共产社会。}

\qquad \kai{孔子说:你这个理想社会和我的大同世界竟是不谋而合,你请让我背一段我的旧文章给你听罢。“大道之行也,天下为公,选贤与能,讲信修睦;故人不独亲其亲,不独子其子,使老有所终,壮有所用,幼有所长,矜寡孤独废疾者皆有所养,男有分,女有归;货恶其弃于地也不必藏于己;力恶其不出于身也不必为己;是故谋闭而不兴,盗窃乱贼而不作,故外户而不闭,是谓大同”,这不是和你的理想完全一致的吗?}

\qquad \kai{马克思说我的理想和有些空想家不同,我的理想不是虚构出来的,也并不是一步可以跳到的。我们先从历史上证明社会的产业有逐渐增值之可能,其次是逐渐增值的财产逐渐集中于少数人之手中,于是使社会生出贫乏病来,社会上的争斗便永无宁日。}

\qquad \kai{孔子说:我从前也早就说过“不患寡而患不均,不患贫而患不安”的呀!}

\qquad \kai{孔子的话还没有十分钟落脚,马克思早反对起来了:不对,不对!你和我的见解终竟是两样,我是患寡且患不均,患贫且患不安的,你要晓得,寡了便不均起来,贫了便是不安的根本。所以我对于私产的集中虽是反对,对于产业的增值却不惟不敢反对,而且还极力提倡,所以我们一方面用莫大的力量去剥夺私人的财产,而同时也要以莫大的力量来增值社会的产业。}

\qquad \kai{孔子说:尊重物质本是我们中国的传统思想,洪范八政食货为先,管子也说过“仓源实而知礼节,衣食足而知荣辱”,我的思想乃至我国的传统思想,根本和你一样,总要先把产业提高起来,然后才来均分。}

\qquad \kai{马克思到此才感叹起来:我不想在两千年前,在遥远的东方,已经有了你这样的一个老同志!你我的见解完全一致的,怎么有人曾说我的思想和你的不合,和你们中国的国情不合,不能施行于中国呢?}

(1) “马克思进文庙”的历史背景是什么?(4分)

(2) 如何理解孔子与马克思对话中谈到的他们之间思想上的“不同”与“一致”?(6分)

\clearpage

\hei{37. 结合材料回答问题:}

材料1

\qquad \kai{2014年10月闭幕的十八届四中全会,是党在中央全会上第一次专题讨论已发治国的问题,体现了对法治的高度重视。会议结束后,微博上的各种评论满是对法治进步的热望:“想要法治的果实,就要给它阳光雨露”“期待法治进入与人民互动的2.0时代”“法治不仅是宏大的,也是具体的;它关乎国家治理,更关于百姓福祉”……}

\qquad \kai{《韩非子》有句名言“国无常强,无常弱。奉法者强则国强,奉法者弱则过弱”。尊奉法律,需要执政者,治理者发力,引导之,提倡之,遵守法律,需要全体公民给力,用法律来定分止争,维护之,践行之。网络上已经有人以普通人“小明”为例,演绎“四中全会与你我有啥关系”。有认识,法治于人就如同空气,你可能不会时时刻刻意识到它的存在,可一旦缺少就立刻窒息。的确,从出生到成长,从成家到立业,舞步需要法治的护航:加强对财产权的保护,完善教育、医疗、视频安全等方面的法律法规,提高环境污染的违法成本……四中全会促动“法治的春天”有着温暖人心的春意。当越来越多人在法治的护佑下感受着畅快的呼吸,法治才能成为内心时时恪守的律令。}

\qquad \kai{也不用回避中国的法治还有很多问题,从“暂行50多年”的高温条例,到保护个人信息安全等方面尚无完善法律,中国的法治进程需要紧跟时代的步伐。四中全会从立法、司法、执法、守法等方面开出来药方,但最根本的还是提升全社会对法治的信心与信任,正如党的十八届四中全会公报所说,法律的权威源自于人民内心的拥护和真诚信仰。这才是法治的力量所在,尊严所系。}

\begin{flushright}\kai{摘自《人民日报》2014年10月24日}\end{flushright}

材料2

\qquad \kai{法治是人类为了征服自己,由人类自己立法进行自我管理,这远比征服自然困难得多。特别是约束公权力,非有高度的觉悟,顽强的毅力和坚强的意志难以成其事。任何国家法治的确立都不是在一盘散沙的状态下随随便便建立起来的,而是必须有坚定有力的集中统一领导和部署。}

\qquad \kai{迄今为止,尚未有法治成功的国家是在群龙无首,四分五裂的状态下实现法治的。恰恰相反,就法治发达国家的经验来看,这些国家的法治之所以能够最终确立,都是自上而下,从官到民表现出对法治制作的追求,付出巨大的努力。在中国这个拥有13忆人口,情况极其复杂的大国建设法治,更需要有自上而下将强统一的领导,要有统一的意志,坚决果断一体推行。正是基于这样的情况,十八界四中全会指出,全面推进依法治国,必须坚持党的领导。}

\begin{flushright}\kai{摘自《人民日报》2014年10月29日}\end{flushright}

(1) 如何理解"法治关乎国家治理,更关乎百姓福祉"?(6分)

(2) 为什么"全面推进依法治国,必须坚持党的领导"(4分)

\clearpage

\hei{38题、结合材料回答问题:}

材料1

\qquad \kai{1960年5月27日,毛泽东与来华访问的英国元帅蒙哥马利,围绕"50年以后中国的命运"有一段深刻的对话。蒙哥马利说,我有一个有趣的问题想问下主席:中国大概需要50年,一切事情就办得差不多了。到那时候,你看中国的前途将会怎样?历史的教训是,当一个国家非常强大的时候,就倾向于侵略,是不是?要向外国侵略,就会被打回来;外国是外国人住的地方,别人不能去,没有权利也没有理由硬挤进去,如果去,就要被赶走,这是历史教训。如果我们占人家一寸土地,我们就是侵略者。}

\qquad \kai{"蒙哥马利之问"折射的是一些西方人内心深处的"国强必霸"逻辑。然而,这样的逻辑与中国人千百年来的民族心理完全不在一个"频道"上。正如习近平所说,"中华民族的血液中没有侵略他人,称霸世界的基因。"}

\qquad \kai{600多年前,郑和受命出使西洋,足迹遍布30多个国家和地区。明朝初期的中国,是综合国力位居世界前列的强国。但是,与地理大发现时期欧洲国家的殖民政策不同。郑和船队始终奉行"共享太平之福"的宗旨,尊重当地习惯,平等开展多边贸易,把中国的建筑、绘画、雕刻、服饰等领域的精湛技术带入亚非国家,促进了中外文化的双向交流和共同进步。郑和下西洋的"和平之旅"永载史册。}

\qquad \kai{明朝洪武年间,缅甸与百夷(今缅甸北部)交战,明太祖未发一兵,派李思聪、钱古训二人劝和。二人先奉劝缅甸"两国之民居处虽分,惟存关市之讥。是其和也,其或纷争不已,天将昭鉴福善祸淫",又告诫百夷"莫如守全,以图绵长,不亦美乎"。双方均为道义所感悟,最终罢战息兵。}

\qquad \kai{孙中山先生曾在日本演说,"东方的文化是王道,主张仁义道德,西方的文化是霸道,主张功利强权。讲仁义道德,是由正义公理来感化人,讲功利强权,是用洋枪大炮来压迫人。"近代中国遭受列强欺凌,无数仁人志士高喊"落后就要挨打""振兴中华",但只是为了获得免于被欺凌的自由,为了以平等的姿态屹立于世界民族之林。正如"和平学之父"约翰·加尔通所说,有些人总希望有一个暴力选择,但中国以自己特有的视角来观察现实,阴阳平衡、尊重智慧、众生平等理念被视为理所当然,和平关系的普遍原则以相互合作、平等互利为起点。}

\qquad \kai{拿破仑说,中国是一头沉睡的狮子,当这头睡狮醒来时,世界都会为之发抖。今年3月,习近平同志在法国巴黎向世界宣示,中国这头狮子已经醒了,但这是一只和平的、可亲的、文明的狮子。读懂了"和"文化是中国人千百年来流淌的血脉,就感受到了走向世界的中国那种无法改变的"和"的气度与内质。}

\begin{flushright}\kai{摘编自《人民日报》(2010年12月22日,2014年5月20日)}\end{flushright}

(1) 毛泽东和蒙哥马利的"对话"反映了什么?(4分)

(2) 如何理解习近平所说"中国这头狮子已经醒了,但这是一只和平的、可亲的、文明的狮子"?(6分)

