1. 1894年1月3日,意大利人卡内帕给恩格斯写信,请求他为即将在日内瓦出版的《新纪元》周刊的创刊号题词,而且要求尽量用简短的字句来表述未来的社会主义纪元的基本思想,以区别于伟大诗人但丁的对旧纪元所作的“一些人统治,另一些人受苦难”的界定。恩格斯回答说,除了从《共产党宣言》中摘出下面一段话外,再也找不到合适的了。这就是:“代替那存在着阶级和阶级对立的资产阶级旧社会的,将是这样一个联合体,在那里,每个人的自由发展是一切人的自由发展的条件。”这段话表明,马克思主义追求的根本价值目标是
\begin{choices}
	\choice0 实现人的自由而全面的发展
	\choice0 实现人类永恒不变的普适价值
	\choice0 建立一个四海之内皆兄弟的大同世界
	\choice0 建立一个自由,平等,博爱的理性王国
\end{choices}
2. 有一则箴言:“在溪水和岩石的斗争中,胜利的总是溪水,不是因为力量,而是因为坚持。”“坚持就是胜利”的哲理在于
\begin{choices}
	\choice0 必然性通过偶然性开辟道路
	\choice0 肯定中包含着否定的因素
	\choice0 量变必然引起质变
	\choice0 有其因必有其果
\end{choices}

3. 右边这张照片(图片略)反映出由于气候变暖,北极冰盖融化,致使北极熊无处可去的场景,颇具震撼力。它给我们地球上的人类发出的警示是
\begin{choices}
	\choice0 人与自然的关系成为人与人之间一切社会关系的核心
	\choice0 生态失衡已成为自然界自身周期演化不可逆转的趋势
	\choice0 自然地理环境已成为人类社会发展的根本决定力量
	\choice0 生态环境已日益成为人类反思自身活动的重要前提
\end{choices}
4. 劳动力成为商品是货币转化为资本的前提条件,这是因为
\begin{choices}
	\choice0 资本家购买的是劳动力的价值
	\choice0 劳动力商品具有价值和使用价值
	\choice0 货币所有者购买的劳动力能够带来剩余价值
	\choice0 劳动力自身的价值能够在消费过程中转移到新的商品中去
\end{choices}
5. 1981年党的十一届六中全会通过《关于建国以来党的若干历史问题的决议》,对我国社会主要矛盾作了规范的表述:“社会主义改造完成以后,我国所要解决的主要矛盾,是人民日益增长的物质文化需要同落后的社会生产之间的矛盾。”我国社会主要矛盾的主要方面将长期是
\begin{choices}
	\choice0 生产力落后
	\choice0 生产力不断发展的要求
	\choice0 经济文化发展不平衡
	\choice0 人民日益增长的物质文化需要
\end{choices}
6. “发展才是硬道理”、“发展是党执政兴国的第一要务”、“发展是解决中国一切问题的总钥匙”,这是对社会主义建设历史经验的深刻总结。中国解决所有问题的关键是要靠自己的发展,而发展的根本目的是
\begin{choices}
	\choice0 增强综合国力
	\choice0 体现社会主义优越性
	\choice0 消灭剥削,消除两极分化
	\choice0 使人民共享发展成果,实现共同富裕
\end{choices}
7. 党的十七大报告指出,坚持节约资源和保护环境的基本国策,关系人民群众切身利益和中华民族的生存发展,必须把建设资源节约型、环境友好型社会放在工业化、现代化发展战略的突出位置。建设资源节约型社会的核心是
\begin{choices}
	\choice0 节约使用资源和提高能源资源利用效率
	\choice0 加强减排和生态保护工作
	\choice0 限制能源开发和利用
	\choice0 发展循环经济
\end{choices}
8. 随着经济的快速发展和物质生活水平的提高。人们的精神文化需求日益增长,迫切要求通过深化体制改革,激发文化发展的活力,为人民群众提供更多更好的文化产品和文化服务,保障人民的基本文化权益。保障人民基本文化权益的主要途径是
\begin{choices}
	\choice0 繁荣社会主义文化,提高文化软实力
	\choice0 协调发展公益性文化事业,经营性文化产业
	\choice0 发展公益性事业,建立政府主导的公共文化体系
	\choice0 调动社会力量在市场竞争中壮大文化事业
\end{choices}
9. “十月革命一声炮响给中国送来了马克思列宁主义”,五四运动后,马克思列宁主义得到广泛传播。在中国最早讴歌十月革命、比较系统的介绍马克思主义的是
\begin{choices}
	\choice0 陈独秀
	\choice0 李大钊
	\choice0 毛泽东
	\choice0 瞿秋白
\end{choices}
10. 1956年4-5月,毛泽东先后在中共中央政治局扩大会议和最高国务会议上作的《论十大关系》报告中指出“最近苏联方面暴露了他们在建设社会主义过程中的一些缺点和错误,他们走过的弯路你还想走?过去,我们就是鉴于他们的经验教训,少走了一些弯路,现在当然更要引以为戒”,这表明以毛泽东为主要代表的中共党员
\begin{choices}
	\choice0 实现了马克思主义同中国实际的第二次结合
	\choice0 开始探索自己的社会主义建设道路
	\choice0 开始找到自己的一条适合中国的路线
	\choice0 已经突破社会主义苏联模式的束缚
\end{choices}
11. 爱因斯坦曾经说过“大多数人都以为是才智成就了科学家,他们错了,是品格”下列名言与这段话含义一致的是
\begin{choices}
	\choice0 道虽迩,不行不至;事虽小,不为不成
	\choice0 才者,德之资也;德者,才之帅也
	\choice0 不学礼,无以立
	\choice0 是非之心,智也
\end{choices}
12. 中华民族精神源远流长,包含着丰富的内容,其中,夸父追日、大禹治水、愚公移山、精卫填海等动人的传说,其中体现的是中华民族精神的
\begin{choices}
	\choice0 勤劳勇敢
	\choice0 团结统一
	\choice0 自强不息
	\choice0 爱好和平
\end{choices}
13. 2001年中共中央印发的《公民道德建设实施纲要》中规定了公民基本道德规范的主要内容。公民道德建设的重点是
\begin{choices}
	\choice0 爱国守法
	\choice0 诚实守信
	\choice0 勤奋自强
	\choice0 团结友善
\end{choices}
14. 我国宪法明确规定,实行依法治国建设社会主义法治国家。依法治国的根本要求是
\begin{choices}
	\choice0 有法可依、有法必依、执法必严、违法必究
	\choice0 保障公民的知情权、参与权、表达权、监督权
	\choice0 立法公开、执法公开、司法公开
	\choice0 社会生活的法制化、规范化、民主化
\end{choices}
15. 2009年3月28日西藏自治区各族各界万余人身着节日盛装在拉萨布达拉宫广场隆重集会。热烈庆祝
\begin{choices}
	\choice0 西藏和平解放58周年
	\choice0 西藏自治区成立44周年
	\choice0 西藏自治区九届人大二次会议召开
	\choice0 首个西藏“百万农奴解放纪念日”
\end{choices}
16. 胡锦涛主席在2009年9月二十国集团领导人比斯堡峰会上,发表了题为《全力促进增长,推动平衡发展》的讲话中指出:当前国际社会十分关注全球经济失衡问题,失衡既表现为部分国家储蓄消费失衡、贸易收支失衡,更表现为世界财富分配失衡、资源拥有和消费失衡、国际货币体系失衡。导致失衡的原因是复杂的、多方面的。从根本上看,失衡根源是
\begin{choices}
	\choice0 经济全球化深入发展、国际产业分工转移、国际资源流动
	\choice0 现行国际经济体系、主要经济体宏观经济政策
	\choice0 各国消费文化和生活方式
	\choice0 南北发展严重不平衡
\end{choices}
\vspace{6pt}
