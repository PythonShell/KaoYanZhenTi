\hei{34题、结合材料回答问题:}

\qquad \kai{早年,梅兰芳与人合演《断桥》,也就是《白蛇传》,剧情是白娘子和许仙两个人悲欢离合的爱情故事,梅兰芳在剧中饰演白娘子。剧中,白娘子有一个动作就是面对负心的丈夫许仙追赶、跪在地上哀求她的时候,她爱恨交加、五味杂陈,就用一根手指头去戳许仙的脑门儿,不想,梅兰芳用力过大,跪在那里扮演许仙的演员毫无防备地向后仰去。这是剧情里没有设计的动作,可能是梅兰芳入戏太深,把对许仙的恨全都聚集在了手指头上,才造成了这样的失误。眼见许仙就要倒地,怎么办?梅兰芳下意识地用双手去扶许仙。许仙是被扶住了,没有倒下。可梅兰芳马上意识到,我是白娘子,他是负心郎许仙,我去扶他不合常理,这戏不是演砸了吗?大师到底是大师,梅兰芳随机应变,在扶住他的同时,又轻轻地推了他一下。所以,剧情就由原来的一戳变成了一戳、一扶和一推,更淋漓尽致地表现出了白娘子对许仙爱恨交织的复杂心情。这个动作,把险些造成舞台事故的错误演得出神入化,得到了大家的认可。从此,在以后的演出中,梅兰芳就沿用了这个动作,而且,其他剧种也都移植采用了这个动作处理,这个动作成了经典之作。}

\qquad \kai{由此可见,不仅在舞台上,在各行各业,在各个岗位,在工作中,在生活中,无论是大师还是普通人,失误和错误是难免的,关键是出现失误和错误以后怎么去对待,怎么去处理。处理不当,会酿成事故,导致全盘失败;处理得当,能败中取胜,化腐朽为神奇。}

(1) 为什么“无论是大师还是普通人,失误和错误是难免的”?

(2) 梅兰芳为什么能“把险些造成舞台事故的错误”变为成功的“经典之作”?

(3) 当我们在认识和实践活动中出现错误或失败该怎样对待和处理?

\clearpage

\hei{35题、结合材料回答问题:}

\hei{材料一}

\qquad \kai{新中国成立60年来,党和政府高度重视社会发展事业,着力保障和改善民生。改革开放以来,在社会建设方面取得显著成就。废除农业税,使延续几千年的“皇粮国税”成为历史。随着经济社会发展,人民生活水平显著改善,“吃穿住行用”水平显著提高。从1949年到2008年,城镇居民人均可支配收入从一年的不到100元增加到15781元,农村居民人均纯收入从44元增加到4761元。从1978年到2008年,城市人均住宅建筑面积和农村人均住房面积,已分别从6.7平方米和8平方米增加到30.0平方米和32.4平方米。2008年城乡居民人民币储蓄存款余额达21.8万亿元,比新中国成立初期的1952年增加了2.5万倍。}

\qquad \kai{我们在看到成绩的同时,也要清醒认识到,我国是世界上最大的发展中国家,人口众多,经济发展起点低,地区之间,城乡之间发展不平衡,造成社会保障体系建设与经济社会发展还有不适应之处,与人们的期望功能和需求还有一定差距。}

\begin{flushright}\kai{摘编自《人民日报》,《理论热点面对面·2009》}\end{flushright}

\hei{材料二}

\qquad \kai{2007年10月,党的十七大对医药卫生事业的发展做出了整体规划。2009年四月,新医改《意见》和《实施方案》正式推出。新医改明确建立了覆盖城乡居民的基本医疗卫生制度的任务和工作。}

\qquad \kai{国务院决定从2009年开始在10\%的县(市、区)实行新型农村社会养老保险的试点,2020年前将覆盖全国农民。60岁后享有“普惠式养老金”,对广大农民来说,是一条振奋人心的利好消息。农民在“种地不交税、上学不付费、看病不太贵”之后,又向“养老不犯愁”的新梦想迈出了坚实的一步。}

\begin{flushright}\kai{摘编自人民网、中国网}\end{flushright}

(1) 为什么在经济发展的同时要加快推进以改善民生为重点的社会建设?

(2) 如何推进以改善民生为重点的社会建设?

\clearpage

\hei{36. 结合材料回答问题:}

\hei{材料1}

\qquad \kai{1949年10月1日,下午15时整,北京天安门城楼,毛泽东向全世界庄严宣告:“中华人民共和国中央人民政府已于本日成立了!”}

\qquad \kai{广场沸腾了!震天的欢呼直冲云霄,帽子、围巾甚至报纸在空中飞舞……}

\qquad \kai{身着深色旗袍的宋庆龄站在城楼上,看着眼前涌动的人潮,看着广场上矗立的孙中山画像,不禁热泪盈眶。8天后,她这样向世人讲述在天安门城楼的那一刻——}

\qquad \kai{“连年的伟大奋斗和艰苦的事迹,又在我眼前出现。但是另一个念头抓住我的心,我知道,这一次不会再回头了,不会再倒退了,这一次,孙中山的努力终于结了果实,而且这果实显得这样美丽……”}

\begin{flushright}\kai{摘编自2009年9月6日《人民日报》}\end{flushright}

\hei{材料2}

\qquad \kai{2009年10月1日,上午10时整,首都各界庆祝中华人民共和国成立60周年大会在北京天安门广场隆重举行,20万军民以盛大的阅兵仪式和群众游行欢庆伟大祖国的这一盛大节日。}

\qquad \kai{天安门城楼红墙正中悬挂着新中国缔造者毛泽东的巨幅彩色画像。人民英雄纪念碑前竖立着伟大的革命先行者孙中山先生的画像,纪念碑两侧超宽电子屏上“伟大的中华人民共和国万岁”、“伟大的中国共产党万岁”等标语格外醒目。广场东西两侧,56根绘有各族群众载歌载舞图案的民族团结柱,象征着56个民族共同擎起祖国繁荣富强的伟大基业。}

\qquad \kai{胡锦涛发表重要讲话。他指出:“60年前的今天,中国人民经过近代以来100多年的浴血奋战终于夺取了中国革命的伟大胜利,毛泽东主席在这里向世界庄严宣告了中华人民共和国的成立。中国人民从此站起来了,具有5000多年文明历史的中华民族从此进入了发展进步的历史新纪元。”}

\begin{flushright}\kai{摘编自2009年10月2日《人民日报》}\end{flushright}

(1)如何理解宋庆龄所说的“孙中山的努力终于结了果实”?

(2)为什么说中华人民共和国的成立标志着“中华民族从此进入了发展进步的历史新纪元”?

\clearpage

\hei{37. 结合材料回答问题:}

\qquad \kai{交通环境是由人、车、路构成的公共生活之一,目前,我国机动车拥有量已超过1.78亿辆,拥有驾照的公民已超过1.3亿人。由此带来一系列的交通安全问题,引发社会公众强烈反响。}

\qquad \kai{下列是有关交通问题的一些调查数据:}

\begin{center}\hei{《人民日报》关于不文明开车行为及其原因的调查}\end{center}

\begin{center}
	\begin{tabular}{|l r|l r|l r|}
	\hline
	\multicolumn{4}{|c|}{个人反感的不文明开车行为} & \multicolumn{2}{|c|}{不文明开车的原因}\\
	\hline
	斑马线不减速让行 & 2156票 & 乱停车挡道 & 1687票 & 司机素质普遍有待提高 & 2269票\\
	\hline
	夜间会车不关远光灯 & 2045票 & 胡乱鸣笛 & 1412票 & 跟风,随大流 & 1469票\\
	\hline
	“加塞儿”,并线不打灯 & 1928票 & 司机出口成“脏” & 1076票 & 行人不文明导致司机不文明 & 757票\\
	\hline
	雨天不减速水溅路人 & 1902票 & 抢黄灯 & 944票 & 因车多路堵无法文明驾驶 & 464票\\
	\hline
	\end{tabular}
\end{center}

\begin{center}\hei{某市交管局一年中查处交通违章的数据统计}\end{center}

\begin{center}
	\begin{tabular}{|r l|c|c|}
	\hline
	\multicolumn{2}{|c|}{全年查处交通违章总数} & 207万起 & 比例:100\%\\
	\hline
	其中:&机动车违章&112.2万起&54.2\%\\
	\hline
	&非机动车违章&80.5万起&38.9\%\\
	\hline
	&行人违章&14.3万起&6.9\%\\
	\hline
	\end{tabular}
\end{center}

\qquad \kai{有专家指出,道路交通上普遍存在的交通不文明现象看似个人的私事,但却折射出某些公民在公共生活领域社会公德和法律意识的缺失。要构建文明出行风尚,既是道德呼唤,也是法律要求。}

(1)为什么文明出行“既是道德呼唤,也是法律要求”?

(2)我们应如何从自身做起,构建文明的公共生活秩序?

\clearpage

\hei{38题、结合材料回答问题:}

\hei{材料一}

\qquad \kai{从2009年11月23日起,一则时长30秒“以中国制造世界合作”为主题的广告在美国有线电视新闻网(CNN)正式播出。该广告由中国商务部会同4家中国行业协会共同委托制作,被认为是中国政府的首个品牌宣传活动,接下来还计划在包括北美、欧洲等中国的主要贸易对象地区播出。}

\qquad \kai{广告围绕“中国制造,世界合作”这一主题,强调中国企业为生产高质量的产品,正不断与海外各国公司加强合作。广告中展示了一系列带有“中国制造”标签的产品,例如,一个类似ipod的mp3播放器上用英文标注“在中国制造,但我们使用来自硅谷的软件”,一双运动鞋和一套衣服上标注有“在中国制造,但我们的设计来自于法国”,一台冰箱上写着“中国制造,但我们采用欧洲风格”。}

\qquad \kai{广告在创意上独树一帜,从引导世界受众重新认识畅销全球的中国产品入手,能够启发世界各地的消费者对“中国制造”和全球贸易的重新思考,从而逐渐抛弃对“中国制造”的偏见。}

\begin{flushright}\kai{摘自人民网}\end{flushright}

\hei{材料二}

\qquad \kai{在美国有线电视新闻网热播的一则“携手中国制造”为主题的广告引发的关注和反响正在持续发酵。有分析人士认为,在当前金融危机阴霾尚未散去,贸易保护主义有所抬头的背景下,主动出击展示国家形象,是一次很好的尝试,有利于提升中国的软实力。}

\qquad \kai{有评论认为,近年来中国经济实力崛起,如何建立国际形象成为当务之急,政府近期启动了国际公关战略,继新华社,《人民日报》等中央媒体率先向世界发声之后,国家形象广告或许会成为提升国家软实力和对外形象的新渠道。还有人认为,中国早前一些产品安全事件令世界关注,现在希望能通过在全球投放广告推广“中国制造”以提升在国际上的形象,广告中出现的法国设计、硅谷技术等字样,说明中国目前还处于产业链的低端,它继续把中国定义为世界工厂,因此,令消费者认为中国还只是产品的制造商,现在是从“中国制造”的地位上升为“中国创造”的时候了。}

\begin{flushright}\kai{摘编自《参考消息》、新华网}\end{flushright}

(1)“中国制造,世界合作”的广告说明了什么?

(2)为什么说现在是从“中国制造”的地位上升为“中国创造”的时候了?
