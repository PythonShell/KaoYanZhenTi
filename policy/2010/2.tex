17. 从上世纪70年代至今,商务印书馆先后出版了多个版本的《新华字典》,删除了一些旧的词条,增加了一些新的词条,并对若干词条的词义作了修改。例如1971年版对“科举”这个词的解释是:“从隋唐到清代的封建王朝为了维护其反动统治而设立分科考选文武官吏后备人员的制度”,1992年版删去“反动“二字,1998年版又删去“为了维护其统治而设”。再如1971年版也一样,直到2008年版删去了这句话。一本小字典,记载着词语的发展变化,也记录着时代前进的印证。字典词条释义的变化表明人们的意识
\begin{choices}
	\choice0 是客观世界的能动反映
	\choice0 取决于词语含义的改变
	\choice0 随着社会生活变化而变化
	\choice0 需要借助语言这一物质外壳表达出来
\end{choices}
18. 历史经验表明,经济危机往往孕育着新的科技革命。1857年世界经济危机引发了电气革命,推动人类社会从蒸汽时代进入电气时代。1929年的世界经济危机,引发了电子革命,推动人类社会从电气时代进入电子时代,由此证明
\begin{choices}
	\choice0 科技革命是摆脱社会危机的根本出路
	\choice0 科学技术是社会形态更替的根本标志
	\choice0 社会实践的需要是科技发展的强大动力
	\choice0 科技创新能够推动社会经济跨越式发展
\end{choices}
19. 有一则寓言讲到:狐狸把鱼汤盛在平底的盘子里,请仙鹤来和它一起“平等”的喝鱼汤,结果仙鹤一点也没喝到,全被狐狸喝去了。这个寓言给人们的启示是,尽管资产阶级宣布“法律面前人人平等”,但是
\begin{choices}
	\choice0 法律名义上的平等掩盖着事实上的不平等
	\choice0 这种形式上的平等既是资本主义制度的本质
	\choice0 它的实质是将劳资之间经济利益的不平等合法化
	\choice0 这种平等的权利是建立在财产不平等基础之上的权利
\end{choices}
20. 1989年,时任美国国务院顾问的弗朗西斯·福山抛出了所谓的“历史终结论”,认为西方实行的自由民主制度是“人类社会形态进步的终点”和“人类最后一种的统治形式”。然而,20年来的历史告诉我们,终结的不是历史,而是西方的优越感。就在柏林墙倒塌20年后的2009年11月9日,BBC公布了一份对27国民众的调查。结果半数以上的受访者不满资本主义制度,此次调查的主办方之一的“全球扫描”公司主席米勒对媒体表示,这说明随着1989年柏林墙的倒塌,资本主义并没有取得看上去的压倒性胜利,这一点在这次金融危机中表现的尤其明显,“历史终结论”的破产说明
\begin{choices}
	\choice0 社会规律和自然规律一样都是作为一种盲目的无意识力量起作用
	\choice0 人类历史的发展的曲折性不会改变历史发展的前进性
	\choice0 一些国家社会发展的特殊形式不能否定历史发展的普遍规律
	\choice0 人们对社会发展某个阶段的认识不能代替对社会发展的整个过程的认识
\end{choices}
21. 中国革命、建设和改革的实践证明,要运用马克思主义指导实践,必须实现马克思主义中国化,马克思之所以能够中国化的原因在于
\begin{choices}
	\choice0 马克思主义理论的内在要求
	\choice0 马克思主义与中华民族优秀文化具有相融性
	\choice0 中国革命建设和改革的实践需要马克思主义指导
	\choice0 马克思主义为中国革命建设和改革提供了现成发展模式
\end{choices}
22. 1952年,党中央在酝酿过渡时期总路线时,毛泽东把实现向社会主义转变的设想,由建国之初的“先搞工业化建设”再一举过渡,改变为“建设和改造同时并举,逐步过渡”,这一改变原因和条件是
\begin{choices}
	\choice0 我国社会主义经济因素的不断增长和对资本主义经济的限制
	\choice0 为了确定我国工业化建设的社会主义方向
	\choice0 我国工业化建设取得了重大成就
	\choice0 民主革命的遗留任务已经完成
\end{choices}
23. 1954年9月,第一届全国人民代表大会第一次会议在北京召开,标志着人民代表大会制度在全国范围内建立起来,人民代表大会制度是中国人民当家作主的根本政治制度,这一制度是
\begin{choices}
	\choice0 中国共产党把马克思主义与中国实际相结合的伟大创造
	\choice0 中国共产党带领全国人民长期奋斗的重要成果
	\choice0 全国各族人民的共同利益和共同愿望的反映
	\choice0 近代以来中国社会发展的必然选择
\end{choices}
24. 我国是一个多民族国家,在社会主义时期处理民族问题的基本原则是
\begin{choices}
	\choice0 实行民族区域自治
	\choice0 维护祖国统一
	\choice0 反对民族分裂
	\choice0 坚持民族平等,民族团结,多民族共同繁荣
\end{choices}
25. 改革开放以来,中国成功走上了一条与本国国情和时代特征相适应的和平发展道路。坚持走和平发展道路,符合中国历史文化传统,这是因为
\begin{choices}
	\choice0 中华民族是热爱和平的民族
	\choice0 和平与发展成为时代发展的潮流
	\choice0 中国人民在对外交流中始终强调“亲仁善邻,和而不同”
	\choice0 中华文化是一种和平的文化,渴望和平始终是中国人民的精神特征
\end{choices}
26. 十九世纪下半叶,以“自强”“求富”为目标的洋务运动历时30年,最终失败的重要原因
\begin{choices}
	\choice0 指导思想的封建性
	\choice0 对外具有依赖性
	\choice0 资金人才的匮乏
	\choice0 洋务企业管理的腐朽性
\end{choices}
27. 邓小平指出:“马克思、列宁从来没有说过农村包围城市,这个原理在当时世界上还是没有的。但是毛泽东同志根据中国的具体条件指明了革命的具体道路”。毛泽东找到农村包围城市、武装夺取政权这条道路的根据是
\begin{choices}
	\choice0 中国内务民主制度,外无民族独立
	\choice0 农民占人口绝大多数,是民主革命的主力军
	\choice0 中国革命的敌人长期占据着中心城市,农村是其统治的薄弱环节
	\choice0 中国经济政治发展的不平衡
\end{choices}
28. 1941年1月,震惊中外的皖南事变爆发后,《新华日报》刊出周恩来的题词手迹:“为江南死国难者致哀。”“千古奇冤,江南一叶,同室操戈,相煎何急?”大敌当前,中国共产党以民族利益为重,坚持正确的方针和原则,避免了抗日民族统一战线的破裂,这些方针和原则有
\begin{choices}
	\choice0 既联合又斗争
	\choice0 有理,有利,有节
	\choice0 针锋相对,寸土必争
	\choice0 发展进步势力,争取中间势力,孤立顽固势力
\end{choices}
29. 解放战争时期,在国民党统治区形成了以学生运动为先导的人民民主运动。成为配合人民解放战争的第二条战线。第二条战线形成的原因是
\begin{choices}
	\choice0 国民党政府专制独裁、官员贪污腐败
	\choice0 国民党在军事上的失利
	\choice0 国民党顽固坚持内战政策
	\choice0 国统区爆发严重经济危机
\end{choices}
30. 1955年,钱学森冲破冲破重重阻力,回到魂牵梦绕的祖国。当有人问他为什么回国时,他说:“我为什么要走回归祖国这条道路?我认为道理很简单——鸦片战争近百年来,国人强国梦不息,抗争不断。革命先烈为兴邦,为了炎黄子孙的强国梦,献出了宝贵的生命,血沃中华热土。我个人作为炎黄子孙的一员,只能追随先烈的足迹。在千万般艰险中,探索追求,不顾及其他,再看看共和国的缔造者和建设者们,在百废待兴的贫瘠土地上,顶住国内的贫穷,国外的封锁,经过多少个风风雨雨的春秋,让一个社会主义新中国屹立于世界东方。想到这些,还有什么个人利益不能丢弃呢?”钱学森发自肺腑的言语,对我们在新时期弘扬爱国主义精神的启示是
\begin{choices}
	\choice0 科学没有国界,但科学家有祖国
	\choice0 个人的理想要与国家的命运、民族命运相结合
	\choice0 爱国主义与爱社会主义具有深刻的内在一致性
	\choice0 爱国主义是爱国情感、爱国思想和爱国行为的高度统一
\end{choices}
31. 政治权利和自由是指公民作为国家政治生活主体依法享有的参加国家政治生活的权利和自由,是国家为公民直接参与政治活动提供的基本保障,这一基本权利具体包括
\begin{choices}
	\choice0 人身自由权
	\choice0 选举权和被选举权
	\choice0 宗教信仰自由
	\choice0 政治自由
\end{choices}
32. 2009年9月18日,中国共产党第十七届中央委员会第四次全体会议胜利闭幕,全会审议通过了《中共中央关于加强和改进新形势下党的建设若干重大问题的决定》,对当前和今后一个时期加强和改进党的建设做出了部署,其中除了强调要建设马克思主义学习型政党,坚持和健全民主集中制外,还要
\begin{choices}
	\choice0 弘扬党的优良作风
	\choice0 深化干部人事制度改革
	\choice0 做好抓基层,打基础工作
	\choice0 加快推进惩治和预防腐败体系建设
\end{choices}
33. 第八次中国——东盟经贸部长会议于2009年8月15日在泰国首都曼谷召开,双方共同签署了中国——东盟自贸区《投资协议》,标志着中国与东盟历时7年之久的自贸区主要谈判任务已经完成,该协议的重要意义在于
\begin{choices}
	\choice0 确保中国对外建立的第一个自贸区于2010年全面建成
	\choice0 将中国——东盟战略伙伴关系提升到更高水平
	\choice0 为地区和全球经济复苏与发展作出积极贡献
	\choice0 为东亚自由贸易区的建立提供法律保障
\end{choices}
\vspace{6pt}
