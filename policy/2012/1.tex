1. 恩格斯说:“鹰比人看得远得多,但是人的眼睛识别的东西远胜于鹰。狗比人具有锐敏得多的嗅觉,但是它连被人当作为各种物的特定标志的不同气味的百分之一也辨别不出来。”人的感官的识别能力高于动物,除了人脑及感官发育得更完美之外,还因为
\begin{choices}
	\choice1 人不仅有感觉还有思维
	\choice0 人不仅有理性还有非理性
	\choice0 人不仅有直觉还有想象
	\choice0 人不仅有生理机能还有心理活动
\end{choices}

2. 有这样一道数学题:“90\%×90\%×90\%×90\%×90\%=?”其答案是约59\%,90分看似一个非常不错的成绩,然而,在一项环环相扣的连续不断的工作中,如果每个环节都打点折扣,最终得出的成绩就是不及格。这里蕴含的辩证法道理是
\begin{choices}
	\choice0 肯定中包含否定
	\choice1 量变引起质变
	\choice0 必然性通过偶然性开辟道路
	\choice0 可能和现实是相互转化的
\end{choices}

3. 在资本主义社会里,资本家雇佣工人进行劳动并支付相应的工资。资本主义工资本质是
\begin{choices}
	\choice0 工人所获得的资本家的预付资本
	\choice0 工人劳动力的价值或价格
	\choice0 工人所创造的剩余价值的一部分
	\choice0 工人全部劳动的报酬
\end{choices}

4. 2011年9月以来美国爆发的“占领华尔街”抗议活动中示威者打出“我们是99\%”的标语,向极富阶层表示不满。漫画(漫画略)所显示的美国社会财富占有的两级分化,是资本主义制度下
\begin{choices}
	\choice0 劳资冲突的集中表现
	\choice0 生产社会化的必然产物
	\choice0 资本积累的必然结果
	\choice0 虚拟资本泡沫化的恶果
\end{choices}

5. 毛泽东曾在不同的场合多次谈到,调查研究有两种方法:一是走马看花、一是下马看花。走马看花,不深入,还必须用第二种方法,就是下马看花,过细看花,分析一朵花。毛泽东强调“下马看花”的实际意义在于
\begin{choices}
	\choice0 解决实际问题必须要有先进理论的指导
	\choice0 运用多种综合方法分析调查研究的材料
	\choice0 马克思主义理论必须适合中国革命的具体实际
	\choice0 只有全面深入地了解中国的实际,才能找出规律
\end{choices}

6. 改革开放以来,我们党对公有制认识上的一个重大突破,就是明确了公有制和公有制的实现形式是两个不同层次的问题。公有制的实现形式是指资产或资本的
\begin{choices}
	\choice0 占有形式
	\choice0 分配形式
	\choice0 所有权归属
	\choice0 组织形式与经营方式
\end{choices}

7. 2011年进行的全国县乡两级人大换届选举,是2010年3月选举法修改后首次实行城乡按相同人口比例选举人大代表。这是我国政治生活中的一件大事,它
\begin{choices}
	\choice0 更好的体现了人人平等、地区平等和民族平等
	\choice0 有利于党在国家政权中发扬民主、贯彻党的群众路线
	\choice0 集中反映了人民代表大会是人民当家作主的根本途径
	\choice0 表明我国人大代表的产生与西方议会成员的产生有根本区别
\end{choices}

8. 邓小平指出:“解决民族问题,中国采取的不是民族共和联邦的制度,而是民族区域自治的制度。我们认为这个制度比较好,适合中国的情况。”我们实行民族区域自治的历史依据是
\begin{choices}
	\choice0 各民族聚居区发展的不平衡性
	\choice0 统一的多民族国家的长期存在和发展
	\choice0 各民族大杂居、小聚居的人口分布格局
	\choice0 近代以来各民族在共同反抗外来侵略斗争中形成的爱国主义精神
\end{choices}

9. 19世纪40年代以后,资本帝国主义势力一次又一次地发动对中国的侵略战争,妄图瓜分中国、灭亡中国。但是,帝国主义列强并没有能够实现他们的这一图谋,其根本原因是
\begin{choices}
	\choice0 中西文化存在巨大差异
	\choice0 中国经济政治发展不平衡
	\choice0 帝国主义列强之间的矛盾和相互制约
	\choice0 中华民族进行的不屈不挠的反侵略战争
\end{choices}

10. 毛泽东在《中国革命和中国共产党》中论述了民主革命和社会主义革命的关系。他指出:“民主革命是社会主义革命的必要准备,社会主义革命是民主革命的必然趋势。”这两个革命阶段能够有机连接的原因是
\begin{choices}
	\choice0 资本主义道路在中国走不通
	\choice0 俄国十月革命为中国提供了经验
	\choice0 民主革命包含了社会主义因素
	\choice0 中国国情决定中国革命必须分两步走
\end{choices}

11. 道德修养是一个循序渐进的过程,古人云:“积土成山,风雨兴焉;积水成渊,蛟龙生焉;积善成德,而神明自得,圣心备焉。故不积跬步,无以至千里;不积小流,无以成江海。”下列名言中与这段话在含义上近似的是
\begin{choices}
	\choice0 仁远乎哉?我欲仁,斯仁至矣
	\choice0 勿以善小而不为,勿以恶小而为之
	\choice0 君子求诸已,小人求诸人
	\choice0 有能一日用其力于仁矣乎?我未见力不足者
\end{choices}

12. 中国特色社会主义法律体系是以我国全部现行法律规范按照一定的标准和原则划分为不同的法律部门,并由这些法律部门所构成的具有内在联系的统一整体。每一法律部门均由一系列调整相同类型社会关系的众多法律、法规所构成。下列选项中属于独立法律部门的是
\begin{choices}
	\choice0 知识产权法
	\choice0 商法
	\choice0 公司法
	\choice0 民法商法
\end{choices}

13. 人生目的是人在人生实践中关于自身行为的根本指向和人生追求,它所认识和回答的根本问题是
\begin{choices}
	\choice0 人为什么活着
	\choice0 人如何对待生活
	\choice0 怎样对待人生境遇
	\choice0 怎样选择人生道路
\end{choices}

14. 社会主义道德建设的核心
\begin{choices}
	\choice0 爱国主义
	\choice0 集体主义
	\choice0 为人民服务
	\choice0 社会主义荣辱观
\end{choices}

15. 与“天宫一号”两度完成“太空之吻”的“神州八号”飞船,于2011年11月17日顺利回“家”,天宫一号与神州八号空间完全对接任务获得圆满成功。这标志着我国
\begin{choices}
	\choice0 载人航天技术已经完全成熟
	\choice0 实现了由航天大国向航天强国的转变
	\choice0 实现了载人航天工程“三步走”的战略
	\choice0 为今后建造载人空间站奠定了坚实的技术基础
\end{choices}

16. 2011年5月18日,国际货币基金组织(IMF)总裁多米尼克·斯特劳斯·卡恩因涉案而辞去总裁职务,引发了欧美等发达国家与发展中国家关于IMF总裁继任人的争夺,6月28日,IMF宣布,该组织新一任总裁是法国经济、财政与工业部长克里斯蒂娜·拉加德,这表明
\begin{choices}
	\choice0 国际货币基金组织改革进程加快
	\choice0 新兴国家的话语权和代表性得到提升
	\choice0 欧美主导国际金融机构的局面仍未改变
	\choice0 发展中国家作为一支独立力量登上世界舞台
\end{choices}

