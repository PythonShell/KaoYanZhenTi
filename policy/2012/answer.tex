%!Tex Program = xelatex
\documentclass[a4paper]{article}

% 设置页边距
\usepackage{geometry}
\geometry{left=2cm, right=2cm, top=2.5cm, bottom=1cm}

% 中文断行
\XeTeXlinebreaklocale "zh"
\XeTeXlinebreakskip = 0pt plus 1pt

%% code include by PythonShell
%% From http://tex.stackexchange.com/questions/140923/how-to-automatically-align-the-four-choices-of-a-multiple-choice-question-in-exa
%% thanks to the author ollydbg23 @ stackexchange
%% some tiny modifies from PythonShell
%% 2014-01-08

\usepackage{ifthen}
\usepackage{calc}
\setlength\parindent{0pt}

    %usage \choice{ }{ }{ }{ }
    %(A)(B)(C)(D)
    \newcommand{\fourch}[4]{
	%\par
            \begin{tabular}{*{4}{@{}p{0.23\textwidth}}}
            [A] ~#1 & [B] ~#2 & [C] ~#3 & [D] ~#4
            \end{tabular}
    }

    %(A)(B)
    %(C)(D)
    \newcommand{\twoch}[4]{
	%\par
            \begin{tabular}{*{2}{@{}p{0.46\textwidth}}}
            [A] ~#1 & [B] ~#2
            \end{tabular}
    \par
            \begin{tabular}{*{2}{@{}p{0.46\textwidth}}}
            [C] ~#3 & [D] ~#4
            \end{tabular}
    }

    %(A)
    %(B)
    %(C)
    %(D)
    \newcommand{\onech}[4]{
	%\par
            [A] ~#1 \par [B] ~#2 \par [C] ~#3 \par [D] ~#4
    }

    \newlength\widthcha
    \newlength\widthchb
    \newlength\widthchc
    \newlength\widthchd
    \newlength\widthch
    \newlength\tabmaxwidth

    \setlength\tabmaxwidth{0.96\textwidth}
    \newlength\fourthtabwidth
    \setlength\fourthtabwidth{0.24\textwidth}
    \newlength\halftabwidth
    \setlength\halftabwidth{0.48\textwidth}

    \newcommand{\choice}[4]{
            \settowidth\widthcha{AM.#1}\setlength{\widthch}{\widthcha}
            \settowidth\widthchb{BM.#2}    
            \ifthenelse{\widthch<\widthchb}{\setlength{\widthch}{\widthchb}}{}
            \settowidth\widthchb{CM.#3}    
            \ifthenelse{\widthch<\widthchb}{\setlength{\widthch}{\widthchb}}{}
            \settowidth\widthchb{DM.#4}    
            \ifthenelse{\widthch<\widthchb}{\setlength{\widthch}{\widthchb}}{}     
            \ifthenelse{\widthch<\fourthtabwidth}{\fourch{#1}{#2}{#3}{#4}}
                               {\ifthenelse{\widthch<\halftabwidth\and\widthch>\fourthtabwidth}{\twoch{#1}{#2}{#3}{#4}}
                               {\onech{#1}{#2}{#3}{#4}}}
    }

\usepackage{fontspec}
\setmainfont{SimSun}	% 设置正文默认字体为SimSun

\newcommand\fontnamekai{楷体}	% 设置楷体
\newfontinstance\KAI {\fontnamekai}
\newcommand{\kai}[1]{{\KAI#1}}

\newcommand\fontnamehei{黑体}	% 设置黑体
\newfontinstance\HEI{\fontnamehei}  
\newcommand{\hei}[1]{{\HEI#1}} 

% 设置页眉
\pagestyle{myheadings}
\markright{2012年考研政治答案——PythonShell 工作室}

% 取消缩进
\setlength{\parindent}{0pt}

\begin{document}
\begin{tabbing}
一、单选题\\
\= 01. A \qquad \= 02. B \qquad \= 03. B \qquad \= 04. C \qquad \= 05. D \qquad \= 06. D \qquad \= 07. A \qquad \= 08. B \qquad \=\\
\> 09. D \> 10. C \> 11. B \> 12. D \> 13. A \> 14. C \> 15. D \> 16. C \\
二、多选题\\
\> 17. BCD  \> 18. ABC  \> 19. CD   \> 20. ABD  \> 21. ABCD \> 22. ABD  \> 23. ABD  \> 24. ABD  \\
\> 25. ABCD \> 26. BC   \> 27. ABC  \> 28. AC   \> 29. ABCD \> 30. ACD  \> 31. ABCD \> 32. ABC  \> 33. BCD
\end{tabbing}

34.

(1)任何事物都是共性和个性的统一,共性寓于个性之中,个性中包含共性。(2分)人们的认识就是从个别到一般再到个别的过程。碎花瓶理论是对碎花瓶这一个别事物中所包含的一般特征的概括和反映,这就使得人们能够举一反三地认识和处理其他事物。(2分)

(2)真理和谬误是辩证统一的,它们相互依存、相互贯通。真理中包含着某些以后会暴露出来的错误的方面或者因素,错误中也隐藏着以后会显露出来的真理的成分或者萌芽。(2分)“世界上没有真正的错误,只有被忽略的智慧”并不是抹杀真理和错误之间的区别,而是指没有单纯的绝对的错误,是指由此看来不能辩证地对待错误而失去在错误中发现真理的可能。(2分)

(3)创新意识在我们认识世界和改造世界中具有重要作用。增强创新意识一要注重实践,从中汲取智慧;二要辩证思维,全面地发展地看问题。透过现象揭示本质,善于从偶然中发现必然。(4分)

35.

(1)全国人大常委会吸收各方意见,对草案相关条款做出修改,既是人民代表大会制度集中人民共同意志、保障人名根本利益的体现,也是民主立法、科学立法的体现。全国人大常委会向全社会公开征求意见,是对人民知情权、参与权、表达权、监督权的尊重。人民群众踊跃发表意见、建言献策,是公民积极有序参与国家决策的体现。

(2)加大收入分配调节力度,合理调整收入分配格局,缓解和缩小收入分配差距,解决收入分配不公问题,坚持发展成果由人民共享,是我国当前收入分配制度改革的基本趋向。此次个税法的修改,紧紧抓住了个人收入分配制度改革这个当前人民群众最直接、最现实的利益问题。现阶段合理调整收入分配格局的基本要求是:着力提高低收入者收入,努力扩大中等收入者的比重,有效调节过高收入。此次个税法的修改,较好地贯彻了这一要求。

36.

(1)从世界范围看,19世纪末20世纪初,民族民主革命已经成为世界潮流;由于帝国主义的侵略,清王朝的腐朽无能,使民族危机日益加深、社会矛盾不断激化;清王朝已成为中国经济发展和社会进步的主要障碍。革命的目的不只是要推翻清王朝的统治,而且要在中国建立共和制度,因此即使是汉族人当皇帝,也必须革命。

(2)中国共产党人继承了孙中山先生开创的民族民主革命,取得了新民主主义革命胜利,建立了中华人民共和国,实现了民族独立、人民解放;中国共产党人继承了孙中山先生建立民主共和国的理想,实现了从新民主主义社会到社会主义社会的转变,确立了社会主义基本制度;中国共产党人继承了孙中山振兴中华的理想,开展了大规模社会主义建设,进行了改革开放新的为大革命,中国特色社会主义事业取得了巨大成就,中华民族伟大复兴展现出光明前景。

37.

(1)诚信在道德体系中具有重要地位,诚信是为人之道,立足之本,人无信不立。诚实守信是公民道德建设的重点,是中华民族的传统美德。在发展社会主义市场经济、构建社会主义和谐社会的过程中,更加需要大力倡导诚实守信的美德。首先,诚实守信是市场经济条件下经济活动的一项基本道德准则。市场经济越发达,对诚实守信的道德要求就越高。其次,诚实守信是职业道德的一项基本要求。最后,诚实守信是做人的一项基本道德准则。

在“信义兄弟”事件中,孙家和工友都表现出诚信为本的美德。哥哥、弟弟、母亲都讲诚信,工友以诚相待,体现出诚实守信的优良品德。“信义兄弟”之举体现了诚信美德的凝聚力和影响力。

(2)法律与道德互为补充,相辅相成,二者统一于社会发展和社会管理中,缺一不可。道德规范作用的更好发挥,需要法律支持;而法律作用的更好实现,则需要以道德建设为重要条件。良好社会秩序的形成、巩固和发展,要靠道德,也要靠法律。

诚实信用是我国民法的一项基本原则。法律要求民事主体从事民事活动、行使民事权利或履行民事义务时,应善意无欺,讲求信用;不规避法律和约定。在民事法律关系中,民事主体必须信守合同,依法行使权力,履行义务,坚守信用原则。

孙家虽然惨遭不幸,仍履行了按时发放工钱的承诺,这是诚实信用法律精神的要求和体现。

38.

经济全球化深入发展,中国与世界的联系日益紧密。随着中国经济实力的提升,国际影响力日益广泛,但西方一些国家和民众长期以来对中国缺乏了解,心存疑虑和误解,甚至有着根深蒂固的偏见。只有主动融入世界、加强沟通、增进了解、建立互信,才能消除误解,让世界了解一个全面真实的中国。

\end{document}

