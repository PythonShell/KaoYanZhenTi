\hei{34题、结合材料回答问题:}

材料1

\qquad \kai{有个人不小心打碎一个花瓶,但他没有陷入沮丧,而是细心地收集起满地的碎片。他把这些碎片按大小分类称出重量,结果发现:10~100克的最少,1~10克的稍多,0。1~1克和0。1克以下的最多;同时他还发现这些碎片的重量之间存在着倍数关系,即较大块的重量是次大块的重量的16倍……因此他发现了“碎花瓶理论”。这个理论可以帮助人们恢复文物、陨石等不知其原貌的物体,给考古和天体的研究带来了意想不到的效果。这个人就是丹麦的物理学家雅各布·博尔。}

\begin{flushright}\kai{摘编自《光明日报》(2011年2月21日)}\end{flushright}

材料2

\qquad \kai{迪迪·艾伦年轻时到一家电影公司打工,跟着知名电影剪辑师罗伯特·怀斯学习。她在给电影《江湖浪子》剪辑时,犯了一个非常不应该的错误:在从一个镜头切换到另一个镜头时,第一个镜头中的声音竟然延续到第二个镜头中去,并且长达三秒钟,导致的结局:主人公驾驶汽车逐渐远去,镜头随之切换到达的目的地场景,而这时依旧可以听见第一个镜头中的汽车声!罗伯特·怀斯非常生气,他把这段影片往艾伦面前一扔说:“把你所犯的错误剪掉!”艾伦沮丧极了,正在她准备剪去自己所犯的那个“错误”时,她忽然看见窗台上的一个小盆景,那是一株地莓,她曾经生长在艾伦家的园子里。只是别的地莓都能长出又甜又红的果实,唯独这株地莓不会结果,可它虽然不会结果,却能开出特别鲜红的花朵!所以艾伦把它移植到了这里,成了一道美丽的风景!如果说不会结果是一种“错误”,但就在这种错误中,它却开出了最美丽的花!想到这里,艾伦怦然心动,她开始重新审视起那段影片,猛然意识到:这个错误的本身,其实就是一朵最美丽的地莓花!按照传统的技法,在镜头切换的同时声音也随之戛然而止,艾伦却把声音延续到第二个镜头中,而这不仅能巧妙糅合由镜头切换而产生的断裂感,还能更加有序地连贯电影节奏!艾伦由此想到,有些时候,把第二个镜头中的声音提前一点出现在第一个镜头的结尾处,也是一种能巧妙显示电影节奏的手法。于是,她把这种“错位剪辑”用到了这部影片的每一个切换的镜头中。影片上映后,这种剪辑效果让所有观众耳目一新,并引起了电影同行的关注和沿用,一场电影剪辑艺术的革新悄悄开始了!当86岁高龄的艾伦病逝后,人们对艾伦的人生态度和对电影的贡献作了这样的总结:“她深信这个世界上没有真正的错误,只有被忽略的智慧!即便是一株无法结出果实的地莓,也不要轻易扔掉,因为它可能会开出最美丽的花朵!”}

\begin{flushright}\kai{摘编自 《扬子晚报》(2011年6月27日)}\end{flushright}

(1) 从打碎花瓶这一现象中所概括出的“碎花瓶理论”为什么能帮助人们恢复文物、陨石等不知原貌的物体?(4分)

(2) 如何理解“这个世界上没有真正的错误,只有被忽略的智慧”?(4分)

(3) 上述两例对我们增强创新意识有何启示?(4分)

\clearpage

\hei{35题、结合材料回答问题}

材料1

\qquad \kai{十一届全国人大常委会第二十次会议初次审议的《中华人民共和国个人所得税法修正案(草案)》将个人所得税免征额由现行的每月2000元调至3000元。随后,全国人大常委会通过中国人大网向社会公开征求意见,共收到82707人提出的意见23万余件。82536人对个人所得税起征点发表意见,其中要求提高起征点的意见高达83\%。}

\qquad \kai{2011年5月10日和20日,全国人大法律委员会、财政经济委员会和全国人大常委会法制工作委员会联合召开座谈会,还分别听取11位专家和16位来自不同地区、不同职业、不同收入群体具有一定代表性的社会公众对草案的意见。}

\qquad \kai{6月27日,个税法修正案草案再次提交全国人大常委会审议时,二审稿对3000元起征点仍未作修改。在审议过程中,有委员表示,网上征求意见中,要求提高起征点的占83\%。对如此集中的意见,草案未充分回应,很难向公众解释清楚。更有委员指出,个人所得税法不是5000元、3000元的问题,而是如何更认真地对待群众意见和老百姓的关注问题。}

\qquad \kai{6月30日,全国人大常委会第二十一次会议以134票赞成、6票反对、11票弃权,决定对《中华人民共和国个人所得税法》作如下修改:一、第三条第一项修改为:“工资、薪金所得,适用超额累进税率,税率为百分之三至百分之四十五。”二、第六条第一款第一项修改为:“工资、薪金所得,以每月收入额减除费用三千五百元后的余额,为应纳税所得额。”}

\begin{flushright}\kai{摘编自 中国人大网(2011年6月30日、2011年7月1日)}\end{flushright}

材料2

\qquad \kai{此次个人所得税法的修改将在社会生活中发挥积极作用。首先,大幅度减轻中低收入纳税群体的负担。一方面,减除费用标准由2000元提高到3500元后,纳税人纳税负担普遍减轻。工薪收入者纳税面调整后,纳税人数由约8400万人减至约2400万人。另一方面,通过调整工薪所得税率结构,使绝大部分工薪所得纳税人在享受提高减除费用标准的同时,进一步减轻税负。这两个措施是不一样的,减除费用的提高是普惠,通过税率级距调整进行结构性的变化,是使中低收入纳税群体在减税的基础上进一步减税。此外,适当扩大低档税率和最高档税率的适用范围,使低税率向大部分纳税人倾斜。其次,适当加大对高收入者的调节力度。实行提高工薪所得减除费用标准和调整工薪所得税率结构变化联动,能够使一部分高收入者在抵消减除费用标准提高得到的减税好处以后,适当增加一些税负。}

\begin{flushright}\kai{摘编自 中国人大网(2011年6月30日)}\end{flushright}

(1) 此次个税法修改过程如何体现了中国特色社会主义民主?(5分)

(2) 结合此次个税法的修改,分析当前我国收入分配制度改革的趋向和合理调整收入分配格局的要求。(5分)

\clearpage

36题、结合材料回答问题:

材料1

\qquad \kai{“余维欧美之进化,凡以三大主义:曰民族、曰民权、曰民生。罗马之亡,民族主义兴,而欧洲各国以独立。洎自帝其国,威行专制,在下者不堪其苦,则民权主义起。十八世纪之末,十九世纪之初,专制仆而立宪政体殖焉。世界开化,人智益蒸,物质发舒,百年锐于千载,经济问题继政治问题之后,则民生主义跃跃然动,二十世纪不得不为民生主义之擅场时代也。是三大主义皆基本于民,递 变易,而欧洲之人种胥治化焉。”}

\qquad \kai{“中国数千年来都是君主专制政体,这种政体,不是平等自由的国民所堪受的,要去这种政体,不是专靠民族革命可以成功……我们推到满洲政府,从驱除满人那一面说是民族革命,从颠覆君主政体那一面说是政治革命,并不是把来分作两次去做。讲到那政治革命的结果,是建立民主政体立宪政体。照现在这样的政治论起来,就算汉人为君主,也不能是革命。”}

\begin{flushright}\kai{摘自《孙中山全集》第一卷}\end{flushright}

材料2

\qquad \kai{“一百年以来,我们的先人以不屈不挠的斗争反对内外压迫者,从来没有停止过,其中包括伟大的中国革命先行者孙中山先生所领导的辛亥革命在内,我们的先人指示我们,叫我们完成他们的遗志。我们现在是这样做了。我们团结起来,以人民解放战争和人民大革命打倒了内外压迫者,宣布中华人民共和国成立了。我们的民族将从此列入爱好和平自由的世界各民族的大家庭,以勇敢而勤劳的姿态工作着,创造自己的文明和幸福,同时也促进世界的和平和自由。我们的民族将再也不是一个被人侮辱的民族了,我们已经站起来了。”}

\begin{flushright}\kai{摘自《毛泽东文集》第五卷}\end{flushright}

(1)如何理解“就算汉人为君主,也不能不革命”?(5分)

(2)为什么说中国共产党人是孙中山开创的革命事业“最忠实的继承者”?(5分)

\clearpage

37题、结合材料回答问题:

\qquad \kai{“新年不欠旧年账,今生不欠来生债”,这是孙东林和哥哥孙水林的共同准则。1989年,孙东林与哥哥孙水林一同组建起建筑队伍,开始在北京、河南等地承接建筑工程和装饰工程。此后的20年中,无论遇到什么状况,孙东林从未拖欠过工人的工资。有时工程款不能及时拿到,他四处借钱,也要坚持将工资发放。他说,“诚信,是为人之道,也是立足之本。”}

\qquad \kai{2010年2月9日,在天津承包建筑工程的孙水林,为抢在春节前赶回武汉给先期返乡的农民工发放工资,不顾路途遥远、天气恶劣,连夜赶路千里送薪。不料,2月10日凌晨遭遇车祸,一家五口不幸遇难。得知噩耗,孙东林悲痛不已。为了替哥哥完成遗愿,他带上哥哥车上的26万元钱,连续驱车15小时,返乡代兄为农民工发放工资。两天未合眼的孙东林流着泪眼赶回家中,和老人商议决定,先替哥哥完成遗愿,年前发完工钱再办丧事。他自己垫上6万以后,还差1万多元。这个时候,他们的老母亲拿着1万块现金交到儿子手上。这可是老人家的养老钱呀!}

\qquad \kai{随后,孙家立即让工友互相通知上门领钱。发工资的时候,孙东林和工友们找不到账单,都是凭着一本“良心账”,工友们说多少,孙东林就给多少。腊月二十九晚上,33。6万元工钱全部发完,竟与哥哥遇难前哥俩说过的数额相差无几。69名拿着工钱的工友对孙东林说:“明年我们跟你接着干”。}

(1)基于“信义兄弟”这个事例,怎样理解诚信及其道德力量?(6分)

(2)在法律关系中,为什么也要坚守诚信?(4分)

\clearpage

38题、阅读下列材料

\hei{材料1}

\qquad \kai{纽约曼哈顿的时报广场,被称为“世界的十字路口”。在胡锦涛主席2011年访美前夕,从1月17日起,一抹亮丽的“中国红”在这里明艳绽放——首都中国国家形象片《人物篇》在时报广场的大屏幕上持续滚动播放,路人纷纷驻足观看。}

\qquad \kai{在这段时长60秒的宣传片中,不仅有袁隆平、杨利伟、姚明等各领域杰出代表,还有“轮椅天使”金晶、“抗震小英雄”林浩、“一辈子在献血”的郭明义等感动中国的普通百姓。他们都面带微笑,神色自信,于静默之间传递着中国的声音。}

\qquad \kai{历史翻回到1904年5月的一天,在一家银行的外墙,贴出了时报广场上的第一张广告。100多年来,这里广告的每一次变幻刷新,都展示着时代的表情。可是在这当中,鲜有中国人的面孔和身影。百年沧桑,终以微笑定格,来到“世界的十字路口”,中国人自信、平和、友好。}

\begin{flushright}\kai{摘编自 人民网(2011年1月20日)}\end{flushright}

\hei{材料2}

\qquad \kai{作为《人物篇》的姐妹篇,一部时长17分钟的中国国家形象片《角度篇》自2011年2月3日起将陆续在亚洲、欧美等地的多家电视台播出。《角度篇》分为“开放而有自信”、“发展而能共享”等8个部分,向世界展现了一个更丰富、多元的中国,其中70\%以上的画面呈现的都是中国普通老百姓的生活。值得一提的是,它没有回避中国目前存在的一些问题,农民工及其家属的生存现状在片中多次出现。这部国家形象片看似一些“视觉碎片”,但能够让世人从中读出一个比较真实、鲜活的当代中国。}

\begin{flushright}\kai{摘编自 新华网(2011年2月2日)}\end{flushright}

\hei{材料3}

\qquad \kai{从2009年“中国制造 世界合作”的广告片,到2011年的国家形象宣传片《人物篇》和《角度篇》,如此几种地向西方展示中国国家形象还是第一次。有评论认为,这几部短片展示了中国人民的勤劳智慧和精神风貌,每个人的笑容都那么阳光,让人油然而生到中国看看的愿望;还有人认为,这是“中国国际形象公关”的又一次主动出击,愈趋自信的中国主动向世界展示自己,中国开始步入“国家公关时代”;也有报道说,中国的宣传片是中国国家领导人出访之前的“盛大派对”,展示了中国的软实力。}

\begin{flushright}\kai{摘编自 人民网(2011年1月20日)}\end{flushright}

(1)分析“中国开始步入‘国家公关时代’”的原因。(4分)

(2)国家形象宣传对提升中国软实力有何作用?(4分)