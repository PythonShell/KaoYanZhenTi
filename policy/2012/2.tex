17. “沉舟侧畔千帆过,病树前头万木春。“辩证法认为发展的实质是新事物的产生和旧事物的灭亡。新生事物必然取代旧事物,从根本上说,是因为
\begin{choices}
	\choice0 新生事物产生于旧事物之后,是新出现的事物
	\choice0 新生事物具有新的结构和功能,能适应已经变化了的环境和条件
	\choice0 新生事物是对旧事物的扬弃,并添加了旧事物所不能容纳的新内容
	\choice0 在社会历史领域内,新生事物符合广大人民群众的根本利益和要求
\end{choices}

18. 2011年4月,耶鲁大学出版了《马克思为什么是对的》一书,书中列举了当前西方社会10个典型的歪曲马克思主义的观点。其中一种观点认为:马克思主义将世间万物都归结于经济因素,艺术、宗教,政治、法律、道德等都被简单地视为经济的反映,对人类历史错综复杂的本质视而不见,而试图建立一种非黑即白的单一历史观,上述观点是对马克思主义关于经济基础和上层建筑辩证关系思想的严重歪曲,其表现为
\begin{choices}
	\choice0 把社会历史发展多重因素的综合作用歪曲为单一因素决定论
	\choice0 把上层建筑与经济基础的相互作用歪曲为机械的单向作用
	\choice0 把经济作为社会的“基础”所具有的归根到底的决定作用歪曲为唯一决定作用
	\choice0 把意识形态对社会历史始终具有的积极能动作用歪曲为消极被动作用
\end{choices}

19. 人们往往将汉语中的“价”、“值”二字与金银财宝等联系起来,而这两字的偏旁却都是“人”,示意价值在“人”。马克思劳动价值论透过商品交换的物与物的关系,揭示了商品价值的科学内涵,其主要观点有
\begin{choices}
	\choice0 劳动是社会财富的唯一源泉
	\choice0 具体劳动是商品价值的实体
	\choice0 价值是凝结在商品中的一般人类劳动
	\choice0 价值在本质上体现了生产者之间的社会关系
\end{choices}

20. 关于共产主义理想实现的必然性,马克思主义除了从社会形态更替规律上作了一般性的历史观论证外,还通过对资本主义社会的深入实证的剖析,科学地论证了
\begin{choices}
	\choice0 资本主义的历史暂时性
	\choice0 资本主义发展的自我否定的趋势
	\choice0 资本主义灭亡的具体途径和方式
	\choice0 工人阶级推翻旧世界建设新世界的历史使命
\end{choices}

21. 从中华人民共和国成立到社会主义改造基本完成,是我国从新民主主义到社会主义的过渡时期。这一时期中国社会的阶级构成主要包括
\begin{choices}
	\choice0 工人阶级
	\choice0 农民阶级
	\choice0 民族资产阶级
	\choice0 城市小资产阶级
\end{choices}

22. 加快转变经济发展方式是推动科学发展的必由之路,是我国经济社会领域的一场深刻变革,贯穿经济社会发展全过程和各领域。在当前和今后一个时期,转变经济发展方式的基本思路是,促进经济增长
\begin{choices}
	\choice0 由主要依靠投资、出口拉动向依靠消费、投资、出口协调拉动转变
	\choice0 由主要依靠第二产业带动向依靠第一、第二、第三产业协同带动转变
	\choice0 由主要依靠国有企业推动向依靠国有企业、民营企业、外资企业协调推动转变
	\choice0 由主要依靠增加物质资源消耗向主要依靠科技进步、劳动者素质提高、管理创新转变
\end{choices}

23. 基层民主是我国广大工人、农民、知识分子和各阶层人士在城乡基层政权机关、企事业单位和基层自治组织中依法直接行使民主权利。发展基层民主
\begin{choices}
	\choice0 有利于提高全民的民族素养,为进一步发展民主创造了条件
	\choice0 是发展社会主义民主的基础性工程
	\choice0 为基层群众直接参与国家事务的管理提供了更多机会
	\choice0 为基层群众管理基层公共事务和公益事业创造了条件
\end{choices}

24. 十七大以来,党对兴起社会主义文化建设新高潮,推动社会主义文化大发展大繁荣作出战略部署,这是基于
\begin{choices}
	\choice0 文化已经成为经济社会发展的强大动力
	\choice0 文化已经成为国家核心竞争力的重要因素
	\choice0 文化产业已经成为国家经济的支柱性产业
	\choice0 文化已经成为民族凝聚力和创造力的重要源泉
\end{choices}

25. 随着我国改革开放的不断深入和社会主义市场经济的不断发展,各种社会矛盾日益凸显。解决这些社会矛盾,改革创新社会管理体制,需要
\begin{choices}
	\choice0 健全社会管理格局
	\choice0 健全基层社会管理体制
	\choice0 创新社会管理理念
	\choice0 创新社会管理方式
\end{choices}

26. 中英《南京条约》签订后,美、法趁火打劫,相继逼迫清政府签订的不平等条约有
\begin{choices}
	\choice0 《虎门条约》
	\choice0 《望厦条约》
	\choice0 《黄埔条约》
	\choice0 《天津条约》
\end{choices}

27. 第二次鸦片战争后,清朝统治集团内部一部分人震惊于列强的“船坚炮利”,主张学习西方以求“自强”,洋务运动由此兴起。洋务运动的一个重要内容就是创办新式学堂,主要有
\begin{choices}
	\choice0 翻译学堂
	\choice0 工艺学堂
	\choice0 军事学堂
	\choice0 法政学堂
\end{choices}

28. 一般说来,游击战争是个战术问题。但是,在抗日战争中,游击战争具有战略地位,是因为它
\begin{choices}
	\choice0 主要是在外线单独作战,而不是在内线配合正规军作战
	\choice0 是抗日战争的主要作战方式,而不是次要作战方式
	\choice0 是大规模的,而不是小规模的
	\choice0 是进攻战,而不是防御战
\end{choices}

29. 1957年2月,毛泽东在扩大的最高国务会议上发表《关于正确处理人民内部矛盾的问题》的讲话,强调指出
\begin{choices}
	\choice0 社会主义社会充满着矛盾
	\choice0 社会主义社会的基本矛盾仍然是生产关系和生产力之间、上层建筑和经济基础之间的矛盾
	\choice0 社会主义社会的矛盾可以通过社会主义制度本身得到解决
	\choice0 把正确处理人民内部矛盾作为国家政治生活的主题
\end{choices}

30. 社会主义法治理念反映和指引着社会主义法治的性质、功能、目标方向、价值取向和实现途径,是社会主义法治的精髓和灵魂。其基本内涵包括依法治国、执法为民和
\begin{choices}
	\choice0 公平正义
	\choice0 自由平等
	\choice0 服务大局
	\choice0 党的领导
\end{choices}

31. 爱国主义体现了人民群众对自己祖国的深厚感情,反映了个人对祖国的依存关系,是人们对自己故土家园、民族和文化的归属感、认同感、尊严感与荣誉感的统一。在我国,爱国主义
\begin{choices}
	\choice0 既是道德要求,又是法律规范
	\choice0 既继承了优良传统,又具有时代特征
	\choice0 体现了爱国主义与爱社会主义的一致性
	\choice0 体现了爱国主义与拥护祖国统一的一致性
\end{choices}

32. 胡锦涛总书记《在庆祝中国共产党成立90周年大会上的讲话》中指出,经过90年的奋斗、创造、积累,党和人民必须倍加珍惜、长期坚持、不断发展的成就是
\begin{choices}
	\choice0 开辟了中国特色社会主义道理
	\choice0 形成了中国特色社会主义理论体系
	\choice0 确立了中国特色社会主义制度
	\choice0 建成了中国特色社会主义现代化国家
\end{choices}

33. 2011年11月28日-12月1日,《联合国气候变化框架公约》缔约方会议在南非德班举行。尽管对部分焦点议题分歧严重,但在各方共同努力下,大会取得了一些重要成果,包括
\begin{choices}
	\choice0 达成涵盖所有缔约方的“国际法律框架”
	\choice0 成立“德班增强行动平台特设工作组”
	\choice0 继续《京都议定书》第二承诺期
	\choice0 正式启动“绿色气候基金”
\end{choices}
