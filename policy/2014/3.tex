34题、结合材料回答问题:

\begin{center}\kai{巧用大循环,处理不再难}\end{center}

\qquad \kai{山东某地采用循环经济的理念,将秸秆“吃干榨尽”,对秸秆利用进行了有益探索。}

\qquad \kai{一、秸秆种蘑菇}

\qquad \kai{该地小麦种植面积为60万亩,按亩产500公斤秸秆计算,每年产生30万吨秸秆。虽然粉碎还田、压块做燃料、青储养殖等消化了大量秸秆,但一些农户为图方便,仍然偷偷焚烧秸秆,当地禁烧压力很大。}

\qquad \kai{2009年,该地通过招商引资引进了一家蘑菇种植企业,该企业以小麦秸秆加鸡粪为原料培育双孢菇,从当地收到小麦秸秆不够用,还在周边100公里范围的县市收集,鸡粪则由当地一家大型养鸡场提供。自蘑菇厂建起来后,蘑菇厂对秸秆的大量需要,让原本难以处理而成为“包袱”的秸秆摇身一变,不仅成了香饽饽,而且还成为农民增收的渠道。}

\qquad \kai{二、延长产业链}

\qquad \kai{然而,蘑菇厂每年产生的6万吨菌渣,四处堆积,臭气难闻,也引来周边群众的投诉,由此,该蘑菇厂开始寻找下游菌渣处理企业,开展产业链条的招商引资。}

\qquad \kai{山东某生物科技有限公司得知消息后主动前来,并把厂于建在该蘑菇厂旁边,他们将买来的菌渣加上猪粪,经过发酵,制成了很好的有机复合肥。这不仅解决了菌渣问题,而且也附带解决了让周边养猪户头痛的猪粪问题,该公司将生产出来的有机复合肥直接卖给周边的有机蔬菜种植基地,种植户以及果农等,由于减少了销售中间环节,价格合理。而很受欢迎。该公司也因之而获利颇丰。}

\qquad \kai{三、“链接”到山林}

\qquad \kai{秸秆经过种植蘑菇,变成了有机复合肥,最后拿到市场上销售,算是完成了一个标准的循环利用过程,然后,如果将有机复合肥集中用于生态修复工程。再次推动一个新的生态产业发展,岂不是更好? }

\qquad \kai{该地又动起脑筋,将秸秆利用产业与退耕还林工程对接,该地的山区丘陵面积占全市总面积的2/3,其中林荒山地有6万多亩,这些山地土壤贫瘠。含沙量大,农作物产量低,经济效益差。}

\qquad \kai{在深入调研的基础上,该地从2011年开始,由市财政投入数亿元,实施为期5年的“自主退耕还林生态富民”工程,打算将这些山地改造成高产的大枣、大樱桃等经济果林,大力推进农林业转型。}

\qquad \kai{而要发展高产高效的有机果业,所面临的突出问题是有机肥从何而来?这时,秸秆等有机肥料又成了人们惦记的宝贝。为了种植出优质林果,当地农民在山地种植果林时,都开始垫秸秆、放菌渣有机复合肥等。大片经济果林的种植,不仅大大地改善了当地生态环境,从而实现了秸秆利用的大循环,而且也大大地提高了农民收入。}

\begin{flushright}\kai{摘编自《人民日报》(2013年6月22日)}\end{flushright}

(1)从唯物辩证法的角度分析“巧用大循环,处理不再难”中“巧”在何处?(6分)

(2)当你在生活中遇到难题和矛盾时,上述事例对你有何启示?(4分)

\clearpage

35题、结合材料回答问题:

材料1

\qquad \kai{1978年我国作出改革开放的战略决策时,美国《时代》杂志曾质疑说:“他们的目标几乎不可能按期实现,甚至不可能实现。”经过三十多年的改革开放,我国国内生产总值,外贸进出口总额均已达到世界第二位,经济总量占世界经济的份额提升到10\%左右,对世界经济增长的贡献率年平均超过20\%。据世界银行统计,我国已进入中高收入国家行列。}

\qquad \kai{在物质文化生活得到提高之后,人民群众对未来期待更高,过去施工建厂,首先考虎的是经济利益,今天引进项目,担心的却是环境污染;过去期盼吃饱穿暖,今天却追求吃的健康、安全检查过去梦想有车有房,现在则忧虑PM2.5排放,城乡居民收入整体都有提高,但城乡区域发展差距和居民收入分配差距依然较大,近10年来中国基尼系数始终处于0.4以上,超出国际公认“警戒线”……这个经济飞速发展、财富不断积累的世界第二大经济体,在创造着“中国式奇迹”的同时,仍有一些“中国式难题”丞待破解。 }

\qquad \kai{“改革开放是我们党的历史上一次伟大觉醒,正是这个伟大觉醒孕育了新时期从理论到实践的伟大创造。”习近平在党的十八大之后首次到地方调研就选择了广东,并向深圳莲花山顶的邓小平钢像敬献了花篮。习近平表示,之所以到广东来,就是要到在我国改革开放中得风气之先的地方,现场回顾我国改革开放的历史进程,将改革开放继续推向前行。我们来瞻爷邓小平钢像。就是要表明我们将坚定不移推进改革开放,奋力推进改革开放和现代化建设取得新进展、实现新突破、迈上新台阶。}

\begin{flushright}\kai{摘编自《人民日报》(2013年3月22日)、新华网(2012年12月11日)等}\end{flushright}

材料2

\qquad {1992年,邓小平同志在南方谈话中说:“不坚持社会主义,不改革开放,不发展经济,不改善人民生活,只能是死路一条。”回过头来看,我们对邓小平同志这番话就有更深的理解了。所以,我们讲,只有社会主义才能救中国,只有改革开放才能发展中国、发展社会主义、发展马克思主义。}

\qquad \kai{正是从历史经验和现实需要的高度,党的十八大以来,中央反复强调,改革开放是决定当代中国命运的关键一招,也是决定实现“两个一百年”奋斗目标、实现中华民族伟大复兴的关键一招,实践发展永无止境,解放思想永无止境,改革开放也永无止境,停顿和倒退没有出路,改革开放只有进行时、没有完成时。}

\begin{flushright} \kai{摘自习近平《关于<中共中央关于全面深化改革若干重大问题的决定>的说明》}\end{flushright}

(1)如何看待改革开放进程中的“中国式奇迹”与“中国式难题”?(4分)

(2)运用社会基本矛盾原理分析为什么“改革开放只有进行时、没有完成时”?(6分)

\clearpage

36题、结合材料回答问题:

材料1

\qquad \kai{1980年8月,邓小平会见意大利记者奥琳娜、法拉奇。法拉奇问:“天安门上的毛主席像,是否要永远保留下去?”邓小平回答说:“永远要保留下去。过去毛主席像挂得太多,到处都挂,并不是一种严肃的事情,也并不能表明对毛主席的尊重。”邓小平又说:“毛主席一生中大部分时间是做了非常好的事情的,他多次从危机中把党和国家挽救过来,没有毛主席,至少我们中国人民还要在黑暗中摸索更长的时间。毛主席最伟大的功绩是把马列主义的原理同中国革命的实际结合起来,指出了中国夺取革命胜利的道路、应该说,在六十年代以前或五十年代后期以前,他的许多思想给我们带来了胜利,他提出的一些根本的原理是非常正确的。”}

\begin{flushright}\kai{摘自《邓小平文选》第2卷 }\end{flushright}

材料2

\qquad \kai{2013年1月5日,习近平在新进中央委员会的委员、候补委员学习贯彻党的十八大精神研讨班开班式上发表重要讲话。他强调指出,我们党领导人民进行社会主义建设,有改革开放前和改革开放后两个时期,这是两个相互联系又有重大区别的时期,虽然这两个历史时期在进行社会主义建设的思想指导、方针政策、实际工作上有很大差别,但两者决不是彼此割裂的,更不是根本对立的,不能用改革开放后的历史时期否定改革开放前的历史时期,也不能用改革开放前的历史时期否定改革开放后的历史时期。要坚持实事求是的思想路线,分清主流和支流,坚持真理,修正错误,发扬经验,吸取教训,在这上基础上把党和人民事业继续推向前进。}

\begin{flushright}\kai{摘自《人民日报》(2013年1月6日)}\end{flushright}

(1)1980年,邓小平为什么强调天安门上的毛主席像“永远要保留下去”?(5分)

(2)如何理解习近平总书记提出的“两个不能否定”的深刻内涵及其意义?(5分)

\clearpage

37题、结合材料回答问题:

\qquad \kai{鹦哥岭是海南省陆地面积最大的自然保护区,区内分布着完整的垂直带谱。在我国热带雨林生态系统保存上独占鳌头。这里山高路远,条件艰苦,一直难以招聘到具有较高专业素质的工作人员。}

\qquad \kai{一、鹦哥岭来了大学生}

\qquad \kai{自2007年起,先后有27名大学毕业生(2名博士、4名硕士、21名本科生)放弃大城市的优越生活,陆续从全国各地来到鹦哥岭保护区工作,山脚下一排破旧平房中的两间就是他们的家。“孩子们,这里的黎苗兄弟说是以种田为生,实际上就是种些橡胶,靠山吃山……你们来任务重啊!在关爱森林的同时,还要想法帮这里的百姓致富!”老站长的一席话,像重锤一样敲击着大家。“我们不会让鹦哥岭失望着的!”大家不约而同地喊出声。 }

\qquad \kai{二、鹦哥岭有了“档案馆”}

\qquad \kai{到底鹦哥岭有多少种动植物?这是摆在大学生们面前最直接的课题、也是鹦哥岭自然保护区要完成的首要工作。大学生们背着睡袋。锅碗瓢盆和监测仪上山了,他们聚精会神地做着记录,天黑了,架起锅巴煮成米饭,和着辣酱吃了实在太困了支起帐蓬钻进去睡一觉……经过4年多的艰辛努力,鹦哥岭自然保护区终于有了自己的“档案馆”;记录到城管来植物2197种、脊椎动物431种、鹦哥岭树蛙等14种科学新种以及26个中国新记录种等。}

\qquad \kai{三、鹦哥岭有了护林员}

\qquad \kai{鹦哥岭周边有103个自然村,近2万村民。看到村民大片砍代雨林种山芝、香蕉、作为环境保护者,大学生们痛心疾首。但习惯靠山吃山的当地百姓说。“让我们放下砍刀、放下猎枪绝对不行!”大学生们克服阻力,用真诚和智慧动员招募了270名护林员,并与他们一起,用一个多月时间,走遍了209公里长的界线,埋下了近400根桩和50多块界碑,为鹦哥岭保护区筑起了一道看得见的保护网。}

\qquad \kai{四、鹦哥岭有了农业示范田}

\qquad \kai{鹦哥岭是海南的贫困山区,为帮助当地黎苗族百姓脱贫致富,大学生们特地去外地取经,在鹦哥岭通过试点而大面积推广“稻鸭共育”的方法,带动当地人致富,农户们在稻田里骄傲地插上了“农业示范田”的牌子。接着大学生们又推广林下经济,在橡胶树下种菜、种瓜、养鸡;并帮助当地人建起了环保厕所,发行了猪圈,改善了居住的环境。当地百姓手里有了钱,靠上山砍树卖钱的人越来越少了。看到这一切,大学生们说,“我们感到由衷的幸福和快乐,也深切地感受到,这就是我们工作的意义和存在的价值。”}

\qquad \kai{5年过去了,27名大学生一直坚守在鹦哥岭,他们甘于寂寞,乐于奉献仪式发现新物种,是敬业的科研工作者;引来环保理念,是先进理念的传播者;心系百姓喜忧,是黎苗族兄弟的贴心人!一份职业,背负三份责任。三个角色的完美融合,让我们看到了甘于寂寞的坚守力量和不甘于寂寞的奋斗精神,也让我们懂得了自已手中的笔、脚下的路、心中的秤要靠什么来指引,他们选择了一种有远见的生活方式。}

\qquad \kai{每到毕业季,总有一些大学生毕业生发出“理想很丰满,现实很骨感”的感慨。究竟如何看待理想与现实的关系,鹦哥岭的大学生们用他们的实际行动给出了最响亮的回答。}

\begin{flushright}\kai{摘编自《光明日报》(2012年4月9日、2013年6月7日)}\end{flushright}

(1)为什么说鹦哥岭的大学生选择的是“一种有远见的生活方式”?(6分)

(2)怎样看待“理想很丰满,现实很骨感”这种说法?

\clearpage

38题、结合材料回答问题:

材料1

\qquad \kai{航海家哥伦布完成了他的前无古人的探险活动后、向支持他探险的西班牙国王和王后汇报他的发现时说:“地球是圆的。”他因为这一伟大的发现而名垂后世。但是,时隔500多年后,美国《纽约时报》中东事务专栏作家、普利策奖获得者托马斯?弗里德曼沿着哥伦布的航程,从美国乘飞机出发,经由法兰克福一直向东飞行,来到了印度的“硅谷”——班加罗尔,经过一段时间的观察,他有一一个破天荒的发现。他回到美国后,悄悄地对他的太太说了一句话:“亲爱的,我发现这个世界是平的。”}

\qquad \kai{“世界是平的”,并不是说地球已改变了它的物理形态,但这个论点的提出却有着划时代的意义。它揭示出当今世界正在发生的深刻而又令人激动的的一个变化——全球化的趋势。它以高科技发展为动力,在地球各处勇往直前、势不可挡,世界也因此从一个球体变得平坦。}

\qquad \kai{“世界是平的”,意味着在今天这样一个因信息技术而紧密、方便的互联世界中,全球市场、劳动力和产品都可以被整个世界共享,一切都有可能以最有效率和最低成本的方式实现。}

\qquad \kai{“世界是平的”,改变着每一个人的工作方式、生活方式和思想方式,乃至一个人的生存方式。因此,生活在当今时代的每一个人,都面临着平坦的世界这样巨大的变化,我们将如何自处?看来,在这个世界里,要想脱疑而出,最重要的一点是不断强化自己的竞争力,首先要培养“学习如何学习”的能力——不断学习和教会自己处理旧事物和新事物的新方式。}

\begin{flushright}\kai{摘编自《人民日报》(2007年7月26日)、《人民日报》(海外版)(2012年11月1日)}\end{flushright}

材料2

\qquad \kai{弗里德曼在《世界是平的》一书中,以丰富生动的语言描述了全球化带来的挑战和益处。其中一段话颇令人回味:“小时候父母常常说,儿子阿,乖乖把饭吃完,因为中国和印度的小孩没饭吃。现在,父母会对孩子说,女儿啊,乖乖把书念完,因为中国和印度的小孩正在等着抢你的饭碗。”}

\begin{flushright}\kai{摘编自《美》托马斯?弗里德曼《世界是平的》}\end{flushright}

(1)在“世界变平”的时代,为什么每个人“要培养 ‘学习如何学习’的能力”?

(2)从“抢饭”到“抢饭碗”的变化说明了什么?