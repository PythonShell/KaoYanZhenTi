17.  海的“贯通东流”水系的形成年代。如果说上游的沉积物从青藏高原、四川盆地顺廷而下能到达下游,这就表是胀江贯通了,这就是物源示踪。我国科学家采用这一方法以,研究长江中下游盆地沉积物的来源,从而判别长江上游的物质何时到达下游,间接指示了长江贯通东流的时限。他们经过10多年的研究,提出长江贯通东流的时间距今约2300多万年。这一研究成果从一个侧面显示出
\begin{choices}
	\choice0  时间和空间是有限的,物质运动是永恒的
	\choice0  时间和空间的通过物质运动的变化表现出来的
	\choice0  时间和空间是指标示物质运动的观念形式
	\choice0  时间和空间是物质运动的存在形式
\end{choices}

18. 作家史铁生在《奶奶的星星》中讲道,奶奶告诉他的故事与通常的说法不同:一般人说,地上死一个人,天上就熄灭了一颗星星;而奶奶说,地上死一个人,天上又多了一个星星,人死了就会升到天空,变成星星给走夜道的人照个亮了。于是他“慢慢相信,每一个活过的人,都能给后人的路途上添些光亮,也许是一颗巨星,也许是一把火炬,也许只是一支含泪的烛光……”这对我们理解个人在社会历史的作用的启示有
\begin{choices}
	\choice0  历史是无数个人相互作用的合力的结果
	\choice0  杰出个人决定历史发展的走向
	\choice0  人人都是历史的创造者
	\choice0  每个人对社会发展都有或大或小的作用
\end{choices}

19. 1918年,马寅初在一次演讲时,有一位老农问他:“马教授,请问什么是经济学?”马寅初笑着说:“我给这位朋友讲个故事吧:有个赶考的书生到旅店投宿,拿出十两银子,挑了该旅店标价十两银子的最好房间,店主立刻用它到隔壁的米店付了欠单,米店老板转身去屠夫处还了肉钱,屠夫马上去付清了赊欠的饲料款,饲料商赶紧到旅店还了房钱。就这样,十两银子又到了店主的手里。这时书生来说,房间不合适,要回银子就走了。你看,店主一文钱也没赚到,大家却把债务都还清了,所以,钱的流通越快越好,这就是经济学。”在这个故事中,货币所发挥的职能有
\begin{choices}
	\choice0  支付手段
	\choice0  流通手段
	\choice0  价值尺度
	\choice0  贮藏手段
\end{choices}

20. 第二次世界大战结束以来,随着国家垄断资本主义的形成和发展,资产阶级国家对经济进行的干预明显加强,从而使得资本主义社会的经济调节机制发生了显著变化。与这种变化相适应,经济危机形态也发生了很大变化。其主要表现是
\begin{choices}
	\choice0  经济危机更多地表现为金融危机的频繁发生
	\choice0  经济危机通常由国家间的贸易失衡直接引发
	\choice0  经济危机各阶段的交替过程已不十分明显
	\choice0  经济危机的破坏作用只局限于发达资本主义国家
\end{choices}

21. 1926~1927年初,邓小平在莫斯科中山大学留学一年。此时正值列宁的新经济政策在莫斯科和整个苏联燎原般发展,国家经济全面开花,市场上商品丰富、品类繁多,商店、饭馆、咖啡馆随处可见。邓小平到中山大学第一天就收到了一大堆日用品,一日三餐也非常丰富。在此期间,邓小平还认真阅读和摘抄了苏联领导人关于新经济政策的许多论述。这一段经历对邓小平后来思考建设“有中国特色的社会主义”具有一定的启示。邓小平与列宁在如何建设社会主义的探索中有许多相通之处,主要有
\begin{choices}
	\choice0  优先发展重工业,快速实现从农业国到工业国的转变
	\choice0  把大力发展生产力、提高劳动生产率放在首要地位
	\choice0  学习和利用资本主义的文明成果
	\choice0  在多种经济成分并存在条件下,利用商品、货币和市场发展经济
\end{choices}

22. 坚持和完善社会主义初级阶段基本经济制度,必须毫不动摇巩固和发展公有制经济,必须毫不动摇鼓励、支持、引导非公有制经济发展。这是因为,公有制经济和非公有制经济都是我国
\begin{choices}
	\choice0  经济社会发展的重要基础
	\choice0  社会主义市场经济的重要组成部分
	\choice0  社会主义经济的重要组成部分
	\choice0  社会主义经济制度的基础
\end{choices}

23. 2013年9月7日,国家主席习近平在哈萨克斯坦纳扎尔耶夫大学发表演讲并回答学生提问时说,“我们既要绿水青山,也要金山银山。宁要绿水青山,不要金山银山,而且绿水青山就是金山银山。”这段话生动地反映了生态文明建设与经济建设之间的关系,即
\begin{choices}
	\choice0  生态环境是经济发展的重要基础
	\choice0  生态文明建设应与经济建设协同发展
	\choice0  生态文明建设可以取代经济建设
	\choice0  生态优势可以转化为经济优势
\end{choices}

24. 2013年6月6日,《财富》全球论坛首次在中国西部内陆城市成都举行。这次论坛以“中国的新未来”为主题,集中讨论了中国西部发展对中国未来发展的重要意义。“优先推进西部大开发”是党的十八大提出的重大战略部署,把西部大开发放在区域发展总体战略的优先位置,是因为西部发展有利于
\begin{choices}
	\choice0  扩大国有资本在西部地区社会总资产中的比重
	\choice0  增强西部地区的经济实力
	\choice0  缩小区域发展差距
	\choice0  形成优势互补、良性互动、协调有序的区域发展格局
\end{choices}

 25.《中共中央关于全国 》
\begin{choices}
	\choice0  增强公民对社会的认同感
	\choice0  降低政府治理成本
	\choice0  扩大政府管理权限
	\choice0  提高社会治理水平
\end{choices}

 26.近年来,我国企业“走出去”的步伐明显加快。非金融类对外直接投资从2007年的248亿美元上升到2012年的773亿美元,年均增长25.5\%,??身对外投资大国行列。我国企业“走出去”战略的重要意义是
\begin{choices}
	\choice0  充分利用国外资源
	\choice0  增强我国企业国际化经营能力
	\choice0  培育我国具有世界水平的跨国公司
	\choice0  拓展我国经济发展空间
\end{choices}

 27.1912年3月中华民国临时参议既颁布的《中华民间临时约法》是中国历史上第一部具有资产阶级共和国宪法性质的法典。毛泽东曾称赞它“带有革命性、民主性”。其“革命性、民主性”主要体现在
\begin{choices}
	\choice0  它不承认清政府与列强签订的一切不平等条约
	\choice0  它规定中华民国国民一律平等
	\choice0  它规定中华民国之主权属于国民全体
	\choice0  它以根本大法的形式废除了封建君主专制制度
\end{choices}

 28.钓鱼岛及其附属岛屿是中国领土不可分割的一部分。中国最早发现、命名、利用和管辖钓鱼岛。1895年,请朝在甲午战争中战败,被迫与日本签署不平等的《马关条约》,割让“台湾全岛及所有附属各岛屿”。钓鱼岛等作为台湾“附属岛屿”一并被割让给日本。1941年12月,中国政府正式对日宣战,宣布废除中日之间的一切条约。日本投降后,依据有关国际文件规定,钓鱼岛作为台湾的附属岛屿应与台湾一并归还中国。这些国际文件是
\begin{choices}
	\choice0  《日本投降书》
	\choice0  《波茨坦公告》
	\choice0  《开罗宣言》
	\choice0  《德黑兰宣言》
\end{choices}

 29.抗日战争结束后,中国共产党为避免内战,实现和平建国,采取的主要措施有
\begin{choices}
	\choice0  参加政协会议并维护政协协议
	\choice0  赴重庆与国民党当局进行谈判
	\choice0  在国统区开辟第二条战线
	\choice0  在解放区开展土地改革运动
\end{choices}

 30.柏拉图说:“法律有一部分是为有美德的人制定的,如果他们愿意和平善良地生活,那么法律可以教会他们在与他人的交往中所要遵循的准则;法律也有一部分是为那些不接受教诲的人制定的,这些人顽固不化,没有任何办法能使他们摆脱罪恶。”这段话所凸显的法律的规范作用是
\begin{choices}
	\choice0  教育作用
	\choice0  保障作用
	\choice0  预测作用
	\choice0  强制作用
\end{choices}

31. 爱国主义优良传统源远流长,内涵数极为丰富。下列诗句中反映爱国主义优良传统的有
\begin{choices}
	\choice0 位卑未敢忘忧国,事定犹须持★棺
	\choice0 四万万人齐下泪,天涯何处是神州
	\choice0 寄意寒星荃不察,我以我血荐轩辕
	\choice0 苟利国家生死以,岂因祸福避趋之
\end{choices}

32. 2013年9月29日,中国(上海)自由贸易试验区正式启动运作,36家中外企业和金融机构颁布证照,首批入驻试验区,建设该试验区的主要任务是
\begin{choices}
	\choice0 促进转变经济增长方式和优化经济结构
	\choice0 推动加快转变政府职能和行政体制改革
	\choice0 为全面深化改革和扩大开放探索新途径、积累新经验
	\choice0 推动构建更加公平合理的市场经济体制
\end{choices}

33. 应中国总理李克强的邀请,俄罗斯总理梅德韦杰夫、印度总理辛格和蒙古国总理阿勒坦呼亚格于2013年10月22日开始分别对中国进行正式访问。来自中国三个陆上邻国的领导人,在同一天开启中国之行,这样密集的双边访问在中国外交史上实属罕见。这一外交动向
\begin{choices}
	\choice0 体现了中国经济发展的吸引力
	\choice0 深化了中国与俄印蒙三国间的盟友关系
	\choice0 反映了中国周边外交行动的延续和加速
	\choice0 顺应了互利共赢的时代潮流
\end{choices}