%!Tex Program = xelatex
\documentclass[a4paper]{article}

% 设置页边距
\usepackage{geometry}
\geometry{left=2cm, right=2cm, top=2.5cm, bottom=1cm}

% 中文断行
\XeTeXlinebreaklocale "zh"
\XeTeXlinebreakskip = 0pt plus 1pt

%% code include by PythonShell
%% From http://tex.stackexchange.com/questions/140923/how-to-automatically-align-the-four-choices-of-a-multiple-choice-question-in-exa
%% thanks to the author ollydbg23 @ stackexchange
%% some tiny modifies from PythonShell
%% 2014-01-08

\usepackage{ifthen}
\usepackage{calc}
\setlength\parindent{0pt}

    %usage \choice{ }{ }{ }{ }
    %(A)(B)(C)(D)
    \newcommand{\fourch}[4]{
	%\par
            \begin{tabular}{*{4}{@{}p{0.23\textwidth}}}
            [A] ~#1 & [B] ~#2 & [C] ~#3 & [D] ~#4
            \end{tabular}
    }

    %(A)(B)
    %(C)(D)
    \newcommand{\twoch}[4]{
	%\par
            \begin{tabular}{*{2}{@{}p{0.46\textwidth}}}
            [A] ~#1 & [B] ~#2
            \end{tabular}
    \par
            \begin{tabular}{*{2}{@{}p{0.46\textwidth}}}
            [C] ~#3 & [D] ~#4
            \end{tabular}
    }

    %(A)
    %(B)
    %(C)
    %(D)
    \newcommand{\onech}[4]{
	%\par
            [A] ~#1 \par [B] ~#2 \par [C] ~#3 \par [D] ~#4
    }

    \newlength\widthcha
    \newlength\widthchb
    \newlength\widthchc
    \newlength\widthchd
    \newlength\widthch
    \newlength\tabmaxwidth

    \setlength\tabmaxwidth{0.96\textwidth}
    \newlength\fourthtabwidth
    \setlength\fourthtabwidth{0.24\textwidth}
    \newlength\halftabwidth
    \setlength\halftabwidth{0.48\textwidth}

    \newcommand{\choice}[4]{
            \settowidth\widthcha{AM.#1}\setlength{\widthch}{\widthcha}
            \settowidth\widthchb{BM.#2}    
            \ifthenelse{\widthch<\widthchb}{\setlength{\widthch}{\widthchb}}{}
            \settowidth\widthchb{CM.#3}    
            \ifthenelse{\widthch<\widthchb}{\setlength{\widthch}{\widthchb}}{}
            \settowidth\widthchb{DM.#4}    
            \ifthenelse{\widthch<\widthchb}{\setlength{\widthch}{\widthchb}}{}     
            \ifthenelse{\widthch<\fourthtabwidth}{\fourch{#1}{#2}{#3}{#4}}
                               {\ifthenelse{\widthch<\halftabwidth\and\widthch>\fourthtabwidth}{\twoch{#1}{#2}{#3}{#4}}
                               {\onech{#1}{#2}{#3}{#4}}}
    }

\usepackage{fontspec}
\setmainfont{SimSun}	% 设置正文默认字体为SimSun

\newcommand\fontnamekai{楷体}	% 设置楷体
\newfontinstance\KAI {\fontnamekai}
\newcommand{\kai}[1]{{\KAI#1}}

\newcommand\fontnamehei{黑体}	% 设置黑体
\newfontinstance\HEI{\fontnamehei}  
\newcommand{\hei}[1]{{\HEI#1}} 

% 设置页眉
\pagestyle{myheadings}
\markright{2014年考研政治答案——PythonShell 工作室}

% 取消缩进
\setlength{\parindent}{0pt}

\begin{document}
\begin{tabbing}
一、单选题\\
\= 01. B \qquad \= 02. C \qquad \= 03. B \qquad \= 04. C \qquad \= 05. B \qquad \= 06. A \qquad \= 07. B \qquad \= 08. A \qquad \= \\
\> 09. D \> 10. A \> 11. C \> 12. D \> 13. A \> 14. A \> 15. D \> 16. C \\
二、多选题\\
\> 17. ABD\> 18. AD \> 19. ABC \> 20. AC \> 21. BCD \> 22. AB \> 23. ABD \> 24. BCD \\
\> 25. ABD\> 26. ABCD \> 27. BCD \> 28. BC \> 29. AB \> 30. AD \> 31. ABCD \> 32. ABCD \> 33. ACD \\
\end{tabbing}

34.

答:(1)矛盾是事物的普遍本质,矛盾普遍存在,我们应当正视矛盾,承认秸秆、菌渣以及山林等问题的存在;矛盾又具有特殊性,我们要分析矛盾的特殊性,要在事物发展过程中分清主次、善于利用不同的方法解决不同的矛盾,集中有限精力重点解决主要问题、突出问题,同时也要兼顾次要问题,具体问题具体分析。“巧用大循环”过程中,从秸秆到菌菇,从菌渣、猪粪到有机肥,再从有机肥到有机果业,因地制宜实现了合理循环。

矛盾双方就有同一性,在一定条件下可以相互转化。材料中人们在尊重客观规律的基础上,充分发挥了主观能动性,创造了实现矛盾双方相互转化的有利条件,促进了最有利于事物发展的状态。

矛盾推动事物的发展。内因是事物发展的根本原因,外因是事物变化的条件,内外因共同作用推动事物发展。材料中秸秆变成菌菇培育的原料再变成有机肥,有其内在关联,同时也离不开人们发挥能动性、招商引资加大投入这一重要条件。

(2)矛盾分析法唯物辩证法的根本方法。我们应该全面而深刻地分析事物的矛盾,善于分析矛盾的特殊性,看到矛盾双方的同一性,创造有利条件解决矛盾。一方面,我们要正确对待事物发展整个过程中的不同矛盾,全面协调解决事物发展过程中的各种矛盾;另一方面,我们又要准确把握解决问题的关键。

主观能动性与客观规律性是人们认识和实践的重大问题。发挥人的主观能动性必须从事物的客观实际出发、以尊重客观规律为前提,只有深刻而正确地认识、掌握和利用事物的发展规律,才能有效地发挥主观能动性。我们只有在认真分析事物发展规律的基础上,充分发挥主观能动性,将革命的热情和科学的态度结合起来,才能不断解决问题。

35.

答:(1)在改革开放的进程中,我们既取得了辉煌的“中国式奇迹”,又面临着诸多“中国式难题”。对此,我们应辩证对待,清醒认识。

十一届三中全会以来,我国通过改革开放取得了“中国式奇迹”:经济平稳较快发展。综合国力大幅提升,改革开放取得重大进展。农村综合改革、集体林权制度改革、国有企业改革不断深化,非公有制经济健康发展。开放型经济达到新水平,进出口总额跃居世界第二位。 人民生活水平显著提高。 民主法制建设迈出新步伐。 文化建设迈上新台阶。 社会建设取得新进步。这些成就都是改革开放取得的辉煌成果。

同时,必须清醒看到,我们工作中还存在许多不足,前进道路上还有不少困难和问题。主要是:发展中不平衡、不协调、不可持续问题依然突出,科技创新能力不强,产业结构不合理,农业基础依然薄弱,资源环境约束加剧,制约科学发展的体制机制障碍较多,深化改革开放和转变经济发展方式任务艰巨;城乡区域发展差距和居民收入分配差距依然较大;社会矛盾明显增多,教育、就业、社会保障、医疗、住房、等关系群众切身利益的问题较多,部分群众生活比较困难;一些领域存在道德失范、诚信缺失现象;少数党员干部理想信念动摇、宗旨意识淡薄,形式主义、官僚主义问题突出,奢侈浪费现象严重,反腐败斗争形势依然严峻。对这些困难和问题,我们必须高度重视,通过进一步改革认真加以解决。

(2)我国社会主义改造完成以后,社会主义社会的基本矛盾仍然是生产力和生产关系之间的矛盾、上层建筑和经济基础之间的矛盾,它们表现在社会生活的各个方面,是推动社会主义社会不断前进的根本动力。这就决定了我们必须通过改革推动社会发展。

社会主义社会的基本矛盾性质是非对抗性的,具有“又相适应又相矛盾”的特点,可以通过社会主义制度本身即改革解决社会基本矛盾。也就是说,我们既不能能封闭僵化的老路,也不能走改旗易帜的邪路,只能走中国特色社会主义道路。改革必须坚持社会主义方向,是社会主义制度的自我完善和发展。全面深化改革,必须立足于我国长期处于社会主义初级阶段这个最大实际,坚持发展仍是解决我国所有问题的关键这个重大战略判断,以经济建设为中心,发挥经济体制改革牵引作用,推动生产关系同生产力、上层建筑同经济基础相适应,推动经济社会持续健康发展。

问题就是矛盾。社会主义社会的基本矛盾是推动社会主义社会不断前进的根本动力。改革是由问题倒逼而产生,又在不断解决问题中得以深化。改革开放是坚持和发展中国特色社会主义的必由之路。没有改革开放,就没有中国的今天,也就没有中国的明天。“改革开放是一项长期的、艰巨的、繁重的事业,必须一代又一代接力干下去。旧的问题解决了,新的问题又会产生”,发展永无止境、实践永无止境,认识也永无止境。“改革开放只有进行时、没有完成时” 。

36.

答:(1)习近平总书记指出,我们党领导人民进行社会主义建设,有改革开放前和改革开放后两个历史时期,这是两个相互联系又有重大区别的时期,但本质上都是我们党领导人民进行社会主义建设的实践探索。

中国特色社会主义是在改革开放新时期开创的,但也是在新中国已经建立起社会主义基本制度、并进行了20多年建设的基础上开创的。虽然这两个历史时期在进行社会主义建设的思想指导、方针政策、实际工作上有很大差别,但两者决不是彼此割裂的,更不是根本对立的。不能用改革开放后的历史时期否定改革开放前的历史时期,也不能用改革开放前的历史时期否定改革开放后的历史时期。

(2)“两个不能否定” 体现了党在新的历史时期对当前国际形势、国内发展所面临的形势作出的又一个准确务实的判断和清晰的论述。

在新民主义革命时期,中国共产党领导人民经过28年艰苦卓绝的斗争推翻了长期压在中国人民头上的帝国主义、封建主义、官僚资本主义三座大山,结束了旧中国长期受外国列强欺凌的历史,真正实现了中华民族的独立。这是中华民族开始走向复兴的一个重要标志,也是中国人民追求民族独立、实现国家富强的历史起点。

改革开放前期,共产党领导中国人民探索的社会主义道路,是要在一穷二白的基础上,建设一个伟大的社会主义现代化强国。这是一项前无古人的艰巨工作,既没有现成的书本答案,又不能照抄外国经验。因此,改革开放以前的探索出现了许多的曲折,甚至是弯路。然而中国却依然在西方国家实行外交孤立、经济封锁、军事包围的极端困难的国际环境下,发展“两弹一星”等高端战略后盾并逐步形成了相对独立的工业体系和国民经济体系,为后来的改革开放奠定了一定的物质基础。

37.

答:(1)个人理想要符合社会理想。并不是要排斥和抹杀个人理想,而是要摆正个人理想同社会理想的关系。个人理想只有同国家的前途、民族的命运相结合,个人的向往和追求只有同社会的需要和人民的利益相一致,才是有意义的。

社会理想是个人理想的凝聚和深化,代表和反映着人们的共同愿望和根本利益,归根到底要靠全体社会成员的共同努力来是此案,并具体体现在每个社会成员为实现个人理想而进行的活生生的实践中。当社会理想同个人理想有矛盾冲突的时候,有志气、有抱负的人可以作出最大的自我牺牲,使个人的理想服从于全社会的共同理想。

人的社会性决定了人生的社会价值是人生价值的最基本内容。一个人的价值主要是以他对社会所作的贡献为标准的,贡献越大,人生价值越大。

社会主义集体主义强调集体利益高于个人利益。在实际生活中,个人利益与集体利益难免发生矛盾,个人应当以大局为重,使个人利益服从集体利益,在必要时为集体的共同利益作出牺牲。

(2)“理想很丰满,现实很骨感”,这就是要求当代年轻人要有勇气有信心,在实践中化理想为现实。第一,正确认识理想与现实的关系是实现理想的思想基础。第二,坚定的信念是实现理想的重要条件。追求理想需要有执着的信念。没有对理想的执著,要想实现宏伟的理想是不可能的。第三,勇于实践、艰苦奋斗是实现理想的根本途径。理想必须通过实践才能转变为现实。再好的理想,如果不付诸行动,就没有实际意义。艰苦奋斗是我们的传家宝。艰苦奋斗始终是激励我们为实现国家富强、民族振奋而共同奋斗的强大精神力量。

38.

答:(1)人才和人的素质的竞争成为综合国力竞争的基础内容。除了少数战略性资源外,一般的物质资源在国家实力中的地位下降,而人才、人的素质作为一种人力资源,其作用和地位上升。较之以往,人才的流动性、包括在世界范围的流动性,大大加强,所以人才的竞争在当今世界表现得越来越突出,许多国家都把教育作为国家发展和振兴的基础,大力培养人才。当今世界是一个不断开放和发展的世界,特别是在信息化日益增强的过程中,生产力水平在不断的提高,要求人们必须不断学习,增强个人知识,提高学习能力,进而适应日益发展的时代要求。当今世界是不断变化的,新的生活方式和思维方式的涌现,需要人们不断学习,不断进步,从而学习适应社会发展的需求。

在日益竞争激烈的社会当中,人们唯有不断学习不断进步,才能培养和锻炼自己的生活技能,提高自己认识新事物,处理新问题的能力。

(2)从“抢饭到抢饭碗”说明随着经济全球化的不断发展,以中国和印度为代表的发展中国家在国际舞台上正发挥着日益重要的作用,特别是以中国为代表的发展中国家的崛起深刻改变着世界的面貌,成为改革 不合理国际经济旧秩序的基本力量,发展中国家地域辽阔,资源丰富,人口众多,发展潜力巨大。20世纪90年代以来,发展中国家的总体实力增强,“二十国集团”取代“八国集团”,“金砖国家”的兴起,表明发展中国家在世界地位与作用日益增大。

\end{document}

