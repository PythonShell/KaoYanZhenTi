1. 爱迪生在发明电灯之前做了两千多实验,有个年轻的记者曾经问他为什么遭遇这么多次失败。爱迪生回答:“我一次都没有失败。我发明了电灯。这只是一段经历了两千步的历程。”爱迪生之所以说“我一次都没有失败”,是因为他把每一次实验都看作
\begin{choices}
	\choice0 认识中所获得的相对真理
	\choice0 整个实践过程中的一部分
	\choice0 对事物规律的正确反映
	\choice0 实践中可以忽略不计的偶然挫折
\end{choices}

2. 俄国早期马克思主义理论家普列汉诺夫说,绝不会有人去组织一个“月食党”以促进或阻止月食的到来,但要进行社会革命就必须组织革命党,这是因为社会规律与自然规律有所不同,它是
\begin{choices}
	\choice0 不具有重复性的客观规律
	\choice0 由多数人的意志决定的
	\choice0 通过人的有意识的活动实现的
	\choice0 比自然规律更易于认识的规律
\end{choices}

3. 社会生产是连续不断进行的,这种连续不断重复的生产就是再生产。每次经济危机发生期间,总有许多企业或因产品积压、或因订单缺乏等致使其无法继续进行再生产而被迫倒闭。那些因产品积压而倒闭的企业主要是由于无法实现其生产过程中的
\begin{choices}
	\choice0 劳动补偿
	\choice0 价值补偿
	\choice0 实物补尝
	\choice0 增殖补偿
\end{choices}

4. 与第二次世界大战前的资本主义相比,当代资本主义在许多方面已经并正在发生着深刻的变化。正确分析这些新变化发生的原因,有利于我们科学而全面地认识当代资本主义社会。导致当代资本主义新变化发生的根本推动力量是
\begin{choices}
	\choice0 改良主义政党对资本主义制度的改革
	\choice0 工人阶级争取自身权利的斗争
	\choice0 科学技术革命和生产力的发展
	\choice0 社会主义制度的优越性对资本主义的影响
\end{choices}

5. 1992年,党的十四大提出了我国经济体制改革的目标是建立社会主义市场经济体制。经过十四大到十八届三中全会20多年的实践,党对政府和市场的关系有了新的科学定位,提出使市场在资源配置中起
\begin{choices}
	\choice0 辅助性作用
	\choice0 决定性作用
	\choice0 基础性作用
	\choice0 补充性作用
\end{choices}

6. 改革开放以来,人民代表大会制度建设和人民代表大会的工作得到不断推进。全国和地方各级人民代表大会的代表
\begin{choices}
	\choice0 实行差额选举
	\choice0 按党派分配名额
	\choice0 按单位分配名额
	\choice0 实行等额选举
\end{choices}

7. 劳动、资本、技术、管理等生产要素是社会生产不可或缺的因素。在我国社会主义初级阶段,实行按生产要素分配的必要性和根据是
\begin{choices}
	\choice0 生产要素可以转化为生产力
	\choice0 我国社会存在着生产要素的多种所有制
	\choice0 按生产要素分配是按劳分的补充
	\choice0 生产要素是价值的源泉
\end{choices}

8. 文化强则中国强。建设社会主义文化强国是实现中华民族伟大复兴的必然要求,其关键是
\begin{choices}
	\choice0 增强全民族文化创造活力
	\choice0 发展新型文化业态
	\choice0 提高全民族思想道德素质和科学文化素质
	\choice0 提升国家文化软实力
\end{choices}

9. 1915年9月,陈独秀在上海创办《青年杂志》。他在该刊发刊词中宣称,“盖改造青年之思想,辅导青年之修养,为本志之天职。批评时政,非其旨也。”此时陈独秀把主要注意力倾注于思想变革的原因是
\begin{choices}
	\choice0 他认为批评时政不利于改造青年思想
	\choice0 他对资本阶级民主主义产生了怀疑
	\choice0 他对政治问题不感兴趣
	\choice0 他认定改造国民性是政治变革的前提
\end{choices}

10. 1924年1月,中国国民党第一次全国代表大会在广州召开,大会通过的宣言对三民主义作出了新的解释。新三民主义成为第一次国共合作的政治基础,究其原因,是由于新三民主义的政纲
\begin{choices}
	\choice0 同中国共产党在民主革命阶段的纲领基本一致
	\choice0 把斗争的矛头直接指向北洋军阀
	\choice0 体现了联俄、联共、扶助农工三大革命政策
	\choice0 把民主主义概括为“平均地权”
\end{choices}

11. 1930年1月,毛泽东在《星星之火,可以燎原》一文中写道:“我所说的中国革命高潮快要到来,决不是如有些人所谓‘有到来之可能’那样完全没有行动意义的、可望而不可即的一种空的东西。它是站在海岸遥望海中已经看得见桅杆尖头了的一只航船,它是立于高山之巅远看东方已见光芒四射喷薄欲出的一轮朝日,它是躁动于母腹中的快要成熟了的一个婴儿。”这段话是针对当时党内和红军中存在的
\begin{choices}
	\choice0 “在全国范围内先争取群众后建立政权”的理论
	\choice0 “御敌于国门之外”的主张
	\choice0 “红旗到底打得多久”的疑问
	\choice0 “一省或数省的首先胜利”的设想
\end{choices}

12.  “房子是应该经常打扫的,不打扫就会积满了灰尘,脸是应该经常洗的,不洗也就会灰尘满面。我们同志的思想、我们党的工作,也会沾染灰尘的,也应该打扫和洗涤。”这段话形象地反映了中国共产党在长期革命实践中历形成的
\begin{choices}
	\choice0 密切联系群众的优良作风
	\choice0 艰苦奋斗的优良作风
	\choice0 理论联系实际的优良作风
	\choice0 批评与自我批评的优良作风
\end{choices}

13. 中国特色社会主义法治理念包含“依法治国、执法为民、公平正义、服务大局、党的领导”五个方面的基本内涵,它们是相辅相成、不可分割的有机整体,构成了社会主义法治理念的完整理论体系。其中,公平正义是
\begin{choices}
	\choice0 社会主义法治的价值追求
	\choice0 社会主义法治的本质要求
	\choice0 社会主义法治的核心内容
	\choice0 社会主义法治的重要使命
\end{choices}

14. 近年来,从“彭宇案”掀起的轩然大波,到“扶老被诬伤老,好人败诉赔钱”等事件的一再发生,使历来推崇“助人为乐”的国人遭遇考验。2013年8月1日,《深圳特区救助人权益保护规定》的正式实施,填补了国内公民救助行为立法的空白。为此,有媒体撰文《“好人法”释放道德正能量》,认为该规定无疑会释放出挺好人、做好人的正能量,对社会风气的净化不无益处。法律之所以能释放道德正能量,是因为
\begin{choices}
	\choice0 法律是道德的归宿
	\choice0 法律是道德的基础
	\choice0 活动是道德的前提
	\choice0 法律是道德的支撑
\end{choices}

15. 党群关系,关乎党和国家的存亡大计。为了实现党的十八大确定的奋斗目标,中共中央部署并在全党开展了党的群众路线教育实践活动。这次活动的主要内容是
\begin{choices}
	\choice0 建设学习型党组识
	\choice0 保持共产党员先进性
	\choice0 讲学习、讲政治、讲正气
	\choice0 为民务实清廉
\end{choices}

16. 2013年6月,中国国家主席习近平与美国总统奥巴马在美国加州安纳伯格庄园会晤时,将中美新型大国关系的内涵概括为
\begin{choices}
	\choice0 共同发展、合作共赢、友好伙伴、相互尊重
	\choice0 加强对话、增加互信、发展合作、管控分歧
	\choice0 不冲突、不对抗、相互尊重、合作共赢
	\choice0 相互尊重、平等互利、密切协作、相互支持
\end{choices}