%!Tex Program = xelatex
\documentclass[a4paper]{article}

% 设置页边距
\usepackage{geometry}
\geometry{left=2cm, right=2cm, top=2.5cm, bottom=1cm}

% 中文断行
\XeTeXlinebreaklocale "zh"
\XeTeXlinebreakskip = 0pt plus 1pt

%% code include by PythonShell
%% From http://tex.stackexchange.com/questions/140923/how-to-automatically-align-the-four-choices-of-a-multiple-choice-question-in-exa
%% thanks to the author ollydbg23 @ stackexchange
%% some tiny modifies from PythonShell
%% 2014-01-08

\usepackage{ifthen}
\usepackage{calc}
\setlength\parindent{0pt}

    %usage \choice{ }{ }{ }{ }
    %(A)(B)(C)(D)
    \newcommand{\fourch}[4]{
	%\par
            \begin{tabular}{*{4}{@{}p{0.23\textwidth}}}
            [A] ~#1 & [B] ~#2 & [C] ~#3 & [D] ~#4
            \end{tabular}
    }

    %(A)(B)
    %(C)(D)
    \newcommand{\twoch}[4]{
	%\par
            \begin{tabular}{*{2}{@{}p{0.46\textwidth}}}
            [A] ~#1 & [B] ~#2
            \end{tabular}
    \par
            \begin{tabular}{*{2}{@{}p{0.46\textwidth}}}
            [C] ~#3 & [D] ~#4
            \end{tabular}
    }

    %(A)
    %(B)
    %(C)
    %(D)
    \newcommand{\onech}[4]{
	%\par
            [A] ~#1 \par [B] ~#2 \par [C] ~#3 \par [D] ~#4
    }

    \newlength\widthcha
    \newlength\widthchb
    \newlength\widthchc
    \newlength\widthchd
    \newlength\widthch
    \newlength\tabmaxwidth

    \setlength\tabmaxwidth{0.96\textwidth}
    \newlength\fourthtabwidth
    \setlength\fourthtabwidth{0.24\textwidth}
    \newlength\halftabwidth
    \setlength\halftabwidth{0.48\textwidth}

    \newcommand{\choice}[4]{
            \settowidth\widthcha{AM.#1}\setlength{\widthch}{\widthcha}
            \settowidth\widthchb{BM.#2}    
            \ifthenelse{\widthch<\widthchb}{\setlength{\widthch}{\widthchb}}{}
            \settowidth\widthchb{CM.#3}    
            \ifthenelse{\widthch<\widthchb}{\setlength{\widthch}{\widthchb}}{}
            \settowidth\widthchb{DM.#4}    
            \ifthenelse{\widthch<\widthchb}{\setlength{\widthch}{\widthchb}}{}     
            \ifthenelse{\widthch<\fourthtabwidth}{\fourch{#1}{#2}{#3}{#4}}
                               {\ifthenelse{\widthch<\halftabwidth\and\widthch>\fourthtabwidth}{\twoch{#1}{#2}{#3}{#4}}
                               {\onech{#1}{#2}{#3}{#4}}}
    }

\usepackage{fontspec}
\setmainfont{SimSun}	% 设置正文默认字体为SimSun

\newcommand\fontnamekai{楷体}	% 设置楷体
\newfontinstance\KAI {\fontnamekai}
\newcommand{\kai}[1]{{\KAI#1}}

\newcommand\fontnamehei{黑体}	% 设置黑体
\newfontinstance\HEI{\fontnamehei}  
\newcommand{\hei}[1]{{\HEI#1}} 

% 设置页眉
\pagestyle{myheadings}
\markright{2013年考研政治答案——PythonShell 工作室}

% 取消缩进
\setlength{\parindent}{0pt}

\begin{document}
\begin{tabbing}
一、单选题\\
\= 01. C \qquad \= 02. D \qquad \= 03. D \qquad \= 04. D \qquad \= 05. D \qquad \= 06. A \qquad \= 07. B \qquad \= 08. A \qquad \= \\
\> 09. D \> 10. D \> 11. B \> 12. C \> 13. B \> 14. C \> 15. C \> 16. D \\
二、多选题\\
\> 17. ACD \> 18. ABCD \> 19. ACD \> 20. ABD \> 21. ABCD \> 22. ACD \> 23. ABCD \> 24. ABCD \\
\> 25. ACD \> 26. ABCD \> 27. ACD \> 28. ABC \> 29. ACD \> 30. ABD \> 31. BC \> 32. BCD \> 33. ACD \\
\end{tabbing}

34.

(1)认识主体性主要表现在:认识可以使主体了解、把握规律性,认识事物的本质,而不只是停留在表面的感性认识层面;认识可以使主体在实践活动之前,对实践活动作出预测和规划;认识可以使主体根据变化了的情况及时调整自己的行动;认识可以指导主体将局部经验上升为理论;认识还可以使主体实现对自身的认识,并自觉调整自己的活动,以适应改造客体的需要。“用孩子的眼睛看世界”和“设计前把自己想象成为老人”这两句富有哲理的话充分了体现了认识主体的能动作用。当主体站在自己的角度无法认识客体或认识不全面时,就应该从客体出发,根据变化了的情况,调整自己的角度以实现对客体正确的认识,并且在认识的过程中不断总结经验并上升为理论,以实现对客体系统的认识。在教育孩子时,用孩子的眼睛看世界,在设计老年人使用的产品时,把自己想象成为老人,只有这样才能真正实现对客体完整、系统和正确的认识。

(2)“用自己的设计改变生活,让生活更加美好”给我们的实践活动具有重大的启示作用。具体表现在:第一,实践具有直接现实性,能够把我们的意识转化为客观世界的现实,因此只有勇于实践才能实现对客观世界的改造,创造美好的生活。第二,我们的实践必须要受认识的指导,尤其是作为认识高级形式的理论对于实践具有巨大的指导作用。因此,在实践过程中,应该首先实现对事物的认识,特别是理性认识,并用它来指导实践,才能取得成功。第三,正确的设计和实践还必须要符合主体的需要,不断创造价值,为群众谋利。总之,实践和认识是辩证统一的,我们在看到实践决定认识方面的同时,也要充分看到认识的主体性及其对实践的指导作用。在认识的指导下,勇于实践,尊重群众的需要,让生活更加美好。

35.

(1)创新是一个民族进步的灵魂,是一个国家兴旺发达的不竭动力,也是一个党政永葆生机的源泉。实践基础上的理论创新是社会发展和变革的先导,要通过理论创新推动制度创新、科技创新、文化创新、以及其他方面的创新,为各个方面的创新提供指导。制度创新的核心内容是社会政治、经济和管理等制度的革新,是支配人们行为和相互关系的规则的变更,是组织与其外部环境相互关系的变更,其直接结果是激发人们的创造性和积极性,促使不断创造新的知识和社会资源的合理配置及社会财富源源不断的涌现,最终推动社会的进步。

(2)基层群众自制制度的主要内容目前中国已经建立了以农村村民委员会、城市居民委员会和企业职工代表大会为主要内容的基层民主自制体系。广大人民在城乡基层群众性自治组织中,依法直接行使民主选举、民主决策、民主管理和民主监督的权利,对所在基层组织的公共事务和公益事业实行民主自治,已经成为当代中国最直接、最广泛的民主实践。

36.

(1)鸦片战争以来,中国逐步沦为半殖民地半封建社会,近代中华民族面临的两大历史任务就是争取民族独立、人民解放和实现国家富强、人民富裕。中国共产党领导人民进行了新民主主义革命,实现了民族独立、人民解放;完成了社会主义革命,确立了社会主义基本制度;进行了改革开放,开创和发展了中国特色社会主义。近代以来的奋斗告诉我们,落后就要挨打,发展才能自强,因此必须实现中华民族的伟大复兴。实现中华民族伟大复兴,就是中华民族近代以来最伟大的梦想。

(2)经过九十多年艰苦奋斗,我们党团结带领全国各族人民,把贫穷落后的旧中国变成日益走向繁荣富强的新中国,中华民族伟大复兴展现出光明的前景。改革开放以来,我们总结历史经验,不断艰辛探索,终于找到了实现中华民族伟大复兴的正确道路,取得了举世瞩目的成果。中国共产党领导人民开辟了中国特色社会主义道路,形成了中国特色社会主义理论体系,确立了中国特色社会主义制度,高举中国特色社会主义伟大旗帜,使中国人民更有信心、有能力实现中华民族伟大复兴的目标。


37.

(1)从法律角度的理解如下:人人生而平等,我国宪法规定:中华人民共和国公民在法律面前一律平等,都平等地受到同等的待遇,不因性别、民族、宗教信仰、等其他因素而区别对待。宪法尊重和保护人权。

自由平等观念是社会主义法制观念的重要组成部分,社会主义法律体系正体现了公民的权利平等地受到法律保护。不能因为是乞丐或者拾荒者而剥夺了他们进入图书馆的权利。公民的权利受法律保护,图书馆属于公共文化事业领域,公民有享受基本文化权益的权利。同样,公民的自由权利也受法律保护,乞丐或者拾荒者有权利选择离开。

(2)人与人的和谐相处时构建和谐社会的基本要求,个人与他人的和谐相处应符合四个基本原则:平等、诚信、宽容、互助。要把自尊和尊重他人结合起来,要学会换位思考,平等待人;相互将往中要重诚信,相互理解,相互信任;要协调好个人与他人相处的原则,严于律己、宽于待人;要学会体谅他人,乐于帮助他人。

材料中的图书馆的读者应该体谅,允许乞丐和拾荒者进入不是对其他读者的不尊重,相反正是允许他们进入才真正体现了人与人的平等,人与人应该保有的和谐关系。

综上所述:社会中的每一位公民都有受教育、享受公共文化设施的权利。和谐社会需要每个人与他人都能够和谐相处,我们应本着平等、诚信、宽容、互助的原则,正确处理好与他人的利益关系。

38.

(1)美国战略东移的主要目的是平衡中国在东南亚和东北亚的影响力。随着中国的不断发展,国际影响力日趋上升,美国此时的战略东移策略主要是为了重新夺回其在亚洲的霸权,通过对传统盟友日本、韩国、菲律宾等的影响,压缩中国的发展空间,打破地区和平稳定的状态,阻止中国的可持续发展。

国际上,随着美国反恐战略接近尾声,需要中国的支持减少。国内方面,经济危机对美国影响仍在持续,国内经济的复苏低迷,失业率的攀升等问题,使得美国在人权、国际贸易往来、人民币升值等问题上会对中国继续施压,要将国内自身存在的问题转嫁给中国。

(2)美国战略重心转移是美国称霸全球野心的表现,这一行为第一会恶化中国周边的安全环境;第二使得中国和周边国家在诸如钓鱼岛、南海诸岛等争议海域和岛屿的已有危机进一步恶化;第三会进一步激化中国和周边国家的矛盾,使得美国从中渔利。

中国和美国是世界上最大的发展中国家和发达国家,两国关系是当今世界最重要的双边关系之一;两国有着广泛而深入的共同利益,两国合则两利、斗则俱伤;两国携手合作,不仅有利于造福两国人民,而且有利于世界的和平、繁荣和稳定。所以两国要走出一条新型大国关系之路。一是需要创新思维,以创新的思维、切实的行动,探索经济全球化时代发展大国关系新路径。二是需要相互信任,三是需要平等互谅,尊重和照顾彼此利益关切,四是需要积极行动,五是需要厚植友谊,积极推进两国。


\end{document}

