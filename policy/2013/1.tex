1. 有一副对联,上联是“桔子洲,洲旁舟,舟行洲不行”,下联是“天心阁,阁中鸽,鸽飞阁不飞。”这形象地说明了运动和静止是相互联系的。静止是%
\begin{choices}
	\choice0 运动的普遍状态
	\choice0 运动的内在原因
	\choice0 运动的衡量尺度
	\choice0 运动的存在方式
\end{choices}

2. 一位机械工程专家讲过这样一件事:“文革”中,他在某地劳动,有一天公社派他去割羊草。他没养过羊,怎么认得羊草呢?但终于一个办法出来了。他把羊牵出去,看羊吃什么就割什么。不到半天就割回了羊草。这位专家之所以这样做是因为他认识到,“羊吃草”与“割羊草”两者之间存在着
\begin{choices}
	\choice0 因果联系
	\choice0 必然联系
	\choice0 主观联系
	\choice0 本质联系
\end{choices}

3. 《资本论》中有这样的表述:“对上衣来说,无论是裁缝自己穿还是他的顾客穿,都是一样的。”这主要是因为无论谁穿
\begin{choices}
	\choice0 上衣都是抽象劳动的结果
	\choice0 上衣都起着价值的作用
	\choice0 上衣都起着使用价值的作用
	\choice0 上衣都是社会劳动的结果
\end{choices}
4. 某资本家投资100万元,每次投资所获得的利润为25万元,假定其资本有机构成为4:1,那么该资本家每次投资所实现的剩余价值率为
\begin{choices}
	\choice0 100\%
	\choice0 75\%
	\choice0 50\%
	\choice0 125\%
\end{choices}
5. 当今世界是开放的世界,中国的发展离不开世界,实行对外开放是我国的一项基本国策。坚持这一基本国策的立足点是
\begin{choices}
	\choice0 相互借鉴,求同存异
	\choice0 多元平衡,共同发展
	\choice0 内外联动,互惠互利
	\choice0 独立自主,自力更生
\end{choices}
6. 公益性文化事业是保障公民基本文化权益的重要途径,大力发展公益性文化事业,始终坚持放在首位的是
\begin{choices}
	\choice0 繁荣文化市场
	\choice0 经济效益
	\choice0 社会效益
	\choice0 创新文化体制
\end{choices}
7. 近年来,为了缩小我国居民在收入分配方面存在的差距,党和政府做出了巨大努力,如提高个税起征点、提高企业退休人员基本养老金、提高国家扶贫标准和城乡低保补助水平等,这些举措体现了
\begin{choices}
	\choice0 劳动报酬在初次分配中比重提高
	\choice0 再分配更加注重公平 
	\choice0 初次分配注重效率
	\choice0 各种生产要素参与分配
\end{choices}
8.党的十八大报告提出为确保实现全面建成小康社会的宏伟目标到2020年,在实现国内生产总值比2010年翻一番的同时,还要实现翻一番的是
\begin{choices}
	\choice0 城乡居民人均收入
	\choice0 城乡居民可支配收入
	\choice0 国民收入
	\choice0 财政收入
\end{choices}
9. 甲午战争后,维新运动迅速兴起,针对洋务派提出的维新派指出,“体”与“用”是不可分的。中学有中学的“体”与,的“体”与“用”,把中学之“体”与西学之“用”凑在一起,就用”一样荒谬。维新派与洋务派分歧的实质是
\begin{choices}
	\choice0 要不要社会革命
	\choice0 要不要以革命手段推翻清政府
	\choice0 要不要在中国兴办近代企业
	\choice0 要不要学习西方的
\end{choices}
10. 1948年10月2日,刘少奇同志在同华北记者团谈话时,故事:巨人安泰是地神之子,他在同对手搏斗时,只要大地母亲那里不断吸取力量,所向无敌;但是,只要他毫无力量。他的对手赫拉克勒斯发现了他的这一特征,把他举到半空中将他扼死。刘少奇借用这一神话故事始终要
\begin{choices}
	\choice0 坚持力量联系实际
	\choice0 保持党的方针政策的正确 
	\choice0 保持对敌人的高度警惕
	\choice0 保持同人民群众的
\end{choices}
11.全面提高公民道德素质,要坚持依法治国,以德治国 公德,职业道德,家庭美德,个人品德教育,弘扬中华传统 新风。下列选项中,即是道德规范又是法律原则的是
\begin{choices}
	\choice0 爱岗敬业
	\choice0 诚实守信
	\choice0 助人为乐
	\choice0 勤俭持家
\end{choices}
12.我们要大力弘扬的时代精神是当代人民精神风貌的集中 力的强大力量。时代精神内涵十分丰富,其核心是
\begin{choices}
	\choice0 国际主义
	\choice0 集体主义
	\choice0 改革创新
	\choice0 开拓进取
\end{choices}
13.个体的人生活动不仅具有满足自我需要的价值属性,还必然包含着满足社会需要的价值属性。个人的需要能不能从社会中得到满足没,在多大程度上得到满足,取决于他的
\begin{choices}
	\choice0 社会影响
	\choice0 社会价值
	\choice0 社会地位
	\choice0 社会理想
\end{choices}
14. “和为贵”是中华民族的传统美德,采用调解的方法解决纠纷,有利于社会和谐。调解可以在诉讼程序外进行,也可以在诉讼程序内进行,诉讼中调解是指
\begin{choices}
	\choice0 人民调解
	\choice0 行政调解
	\choice0 司法调解
	\choice0 仲裁调解
\end{choices}
15. 2012年6月27日,中国宣布在南海地区对外开放九个海上区块,供与外国公司合作勘探开发。此外,海南省宣布将西沙群岛的四个区域划定为文化遗产保护区。这些决定联同设立三沙市构成了中国加强在南海地区维护主权的“组合行动”。中国在南海问题上的一贯主张是
\begin{choices}
	\choice0 使南海问题长期化
	\choice0 推动南海问题国际化
	\choice0 搁置争议,共同开发
	\choice0 建立“南海和平自由友谊合作区”
\end{choices}
16. 2012年9月25日,第67届联合国大会一般性辩论在纽约联合国总部开幕。针对错综复杂的国际形势和此起彼伏的地区动荡及热点问题,本次辩论的主题是
\begin{choices}
	\choice0 改善全球经济治理
	\choice0 携手推动各国普遍安全与共同发展
	\choice0 增强联合国维和作用
	\choice0 以和平方式调解或解决国际争端或局势
\end{choices}
\vspace{6pt}
