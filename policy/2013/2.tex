17. 马克思主义是关于无产阶级和人类解放的科学,实现共产主义是全人类解放的根本体现。人类解放包括
\begin{choices}
	\choice0 从自然的压迫下解放出来
	\choice0 从客观规律的制约下解放出来
	\choice0 从旧的社会关系的束缚下解放出来
	\choice0 从旧的传统观念的禁锢下解放出来
\end{choices}
18. 唯物史观第一次科学地解决了历史创造者的问题,认为人民群众是历史的创造者。人民群众
\begin{choices}
	\choice0 从量上说是指社会人的绝大多数
	\choice0 从质上说是对社会历史发展起推动作用的人们
	\choice0 在任何历史时期都不包括剥削阶级
	\choice0 最稳定的主体部分始终是从事物质资料生产的劳动群众及其知识分子
\end{choices}
19. 美国导演迈克尔。穆尔在他的最新记录片《资本主义:一个爱情故事》问世以来,一直颇受关注。“资本主义”为何与“爱情故事”联系起来呢?穆尔解释说,这是一种“贪欲之爱”,喜爱财富的人不仅爱他们自己的钱,也爱你口袋中的钱……很多人不敢说出它的名字,真见鬼,就说出来吧。这就是“资本主义”。对金钱的“贪欲”与资本主义连为一体,是因为
\begin{choices}
	\choice0 资本就是人格化的资本
	\choice0 赚钱体现了人的天然本性
	\choice0 资本的生命在于不断运动和不断增值
	\choice0 追逐剩余价值是资本主义生产方式的绝对规律
\end{choices}
20. 伴随着生产力发展,科技进步及阶级关系调整,当代资本主义社会的劳资关系和分配关系发生了很大变化。其中资本家及其代理人为缓和劳资关系所采取的激励制度有
\begin{choices}
	\choice0 职工终身雇佣制度
	\choice0 职工参与决策制度
	\choice0 职工选举管理着制度
	\choice0 职工持股制度
\end{choices}
21. 自第一个社会主义国家建立以来,社会主义事业的发展并不是一帆风顺的,社会主义发展道路的多样性以及发展过程中的前进性和曲折性的时间告诉我们
\begin{choices}
	\choice0 发展社会主义,不等于不学习西方资本主义的文明成果
	\choice0 坚持社会主义,不等于要坚持某种单一的社会主义模式
	\choice0 改革或抛弃某种社会主义模式,不等于改掉或抛弃社会主义
	\choice0 某种社会主义模式的失败,不等于整个社会主义事业的失败
\end{choices}
22. 党的十八大把科学发展观同马克思主义列宁主义、毛泽东思想、邓小平理论、“三个代表”重要思想一道确立为党必须长期坚持的指导思想。科学发展观是
\begin{choices}
	\choice0 中国特色社会主义理论体系的最新成果
	\choice0 中国革命、建设、改革经验的科学总结
	\choice0 指导党和国家全部工作的强大思想武器
	\choice0 中国共产党集体智慧的结晶
\end{choices}
23. 国务院新闻办公室发表的《西藏和平解放60年》白皮书指出,目前,西藏共有各类宗教活动场所1700余处,僧尼约4. 6万人,藏传佛教特有的活佛转世的传承方式得到充分尊重,寺庙学经、辩经、受戒、灌顶、修行等传统宗教活动和寺庙学经考核晋升学位活动正常进行,每年到拉萨朝佛敬香的信教群众达百万人次以上。上述事实表明,我国的宗教政策得到了充分贯彻。我国宗教政策主要有
\begin{choices}
	\choice0 尊重和保护公民的宗教信仰自由
	\choice0 积极引导宗教与社会
	\choice0 独立自主自办教会
	\choice0 依法管理宗教事务
\end{choices}
24. 统筹区域发展,促进区域协调发展,是我国经济社会发展的一个重点原则,坚持这一原则有利于
\begin{choices}
	\choice0 扩大内需,拉动经济增长
	\choice0 区域间优势互补,促进经济共同发展
	\choice0 不同区域人民共享改革发展成果
	\choice0 生产要素在区域间合理流动和配置
\end{choices}
25. 1992年初,邓小平在南方谈话中指出:“社会主义的本质是,解放生产,发展生产力。消除剥削,消除两极分化,最终达到共同富裕。“这一概括对社会主义传统认识的突破主要体现在
\begin{choices}
	\choice0 破除了脱离生产力水平抽象谈论社会主义的认识
	\choice0 否定了社会主义必须坚持公有制和按劳分配原则的认识
	\choice0 摆脱了长期以来忽视建设社会主义根本目的和目标的认识
	\choice0 Test
\end{choices}
26. PM2.5(细粒颗粒)这个大家原本很陌生的专有名词,因为2011年10月我国多地灰霾天气造成严重大气污染而迅速成为社会热词。2012年2月修订的《环境空气质量标准》增加了PM2.5指标,该指标随后又被写入政府工作报告。这既折射出当前我国环境污染的严重性,同时也反映了党和政府治理环境污染、建设生态文明的决心。十八大提出的生态文明建设新要求是
\begin{choices}
	\choice0 加大自然生态系统和环境保护力度
	\choice0 全面促进资源节约
	\choice0 优化国土空间开发格局
	\choice0 加强生态文明制度建设
\end{choices}
27. 1925至1927年的大革命规模宏伟,内涵丰富,与辛亥革命相比较,其不同点在于
\begin{choices}
	\choice0 它广泛而深刻地发动了工农群众
	\choice0 它的主要斗争形式是武装斗争
	\choice0 它的革命对象是帝国主义和封建军阀
	\choice0 它是在以国共合体为基础的统一战线的组织下进行的
\end{choices}
28.1931年1月至1935年1月,以王明为代表的“左”倾错误给中国革命带来严重危害,其主要错误有
\begin{choices}
	\choice0 排斥和打击中国民族资本主义势力
	\choice0 将反帝反封建与反资产阶级并列
	\choice0 集中力量攻打大城市
	\choice0 主张“一切经过统一战线”
\end{choices}
29. 抗日战争是近代以来中华民族反抗外敌入侵第一次取得完全胜利的民族解放战争,中国赢得抗日战争胜利的主要原因是
\begin{choices}
	\choice0 中国共产党发挥了中流砥柱的作用
	\choice0 中国的国力空前强大
	\choice0 得到了国际反法西斯力量的同情和支持
	\choice0 中国实现了空前的民族觉醒和民族团结
\end{choices}
30. 以毛泽东同志为核心的党的第一代中央领导集体带领全党全国和各族人民完成了新民主主义革命,进行了社会主义改造,确立了社会主义基本制度,这一基本制度的确立
\begin{choices}
	\choice0 为当代中国一切发展进步奠定了根本政治前提和制度基础
	\choice0 是中国历史上最深刻最伟大的社会变革
	\choice0 标志着马克思主义同中国实际第二次结合的完成
	\choice0 使广大劳动人民真正成为国家的主人
\end{choices}
31. 一位社会学家发现大楼的一块玻璃坏了,起初他没太当回事,没过多久,他发现许多处窗户都破损了,经过调研后,他得出结论:一样东西如果有点破损,人们就会有意无意地加快它的破损速度,一样东西如果完好无损,或是及时维护,人们就会精心的护理。这就是著名的“破窗定律”。下列关于道德修养的名言与“破窗定律”内涵相近的是
\begin{choices}
	\choice0 非知之难,行之惟难,非行之难,终之斯难。
	\choice0 善不可谓小而无益,不善不可谓小而无伤。
	\choice0 小善虽无大益,而不可不为。细恶虽无近祸,而不可不去。
	\choice0 见贤思齐焉,见不贤而内自省也。
\end{choices}
32. “可上九天揽月,可下五洋捉鳖”,2012年6月24日在航天员的手动操作下,我国“神航号”宇宙飞船与“天宫一号”交互对接成功,向未来建设宇宙空间站迈向坚实一步。就在同一天,我国的“蛟龙号”载人潜水器成功下潜7020米的深度,再次创造历史记录。这两次成就的里程碑意义在于
\begin{choices}
	\choice0 完成了“近地空间长期载人飞行”和“南海深部计划”
	\choice0 丰富了人类认识和开发宇宙的梦想和实践
	\choice0 为中国在太空和海底参与国际合体提供了更多机会
	\choice0 为中国在太空和大洋这两个尚未充分利用空间的活动创造了新可能性
\end{choices}
33. 2012年6.6至7日,上海合作组织成员国元首理事会第十二次会议在京举行。这是该组织发展进入第二个十年的首次峰会,与会各国元首就深化成员国友好合作以及重大国际和地区问题深入交换了意见,达成了新的重要共识。此次峰会的主要成果有
\begin{choices}
	\choice0 首次制定了中长期战略规划
	\choice0 签订了军事同盟条约
	\choice0 签署了首个人文合作宣言
	\choice0 增加了新的观察员国和对话伙伴国
\end{choices}
\vspace{6pt}
