\hei{34题、结合材料回答问题:}

材料1

\qquad \kai{小学老师雷夫.爱斯基斯在其所著的热门教育畅销书《第56号教室的奇迹》中讲过这样一个故事:}

\qquad \kai{一位从事特殊教育的优秀教师获得一个宝贵的签名球,上面有美国著名棒球队――红袜队1967年全体队员的签名,这些球员都是他的偶像,对这样一个签名球,这位教师别提有多珍爱了。当年幼的儿子找他一起玩球时,他理所当然地警告儿子:绝对不能拿签名球来玩。儿子问他理由时,他觉得儿子还太小,对球和球员一无所知,说多了儿子也不会明白,于是,他没有解释原委,只对儿子说,不能用那颗球,是因为“球上写满了字”。}

\qquad \kai{过了几天,儿子又找他一起玩球,当老爸再次提醒儿子不要拿写满字的球来玩时,小男孩满不在乎说:我已经把问题解决了,爸爸问怎么回事,儿子说:我把球上所有的字都擦掉了。老爸气的想痛打儿子,但他转念一想,觉得儿子根本没有做错事。因为自己并没有告诉儿子上面的字有什么意义。从那天起,他无论去什么地方,都带着那颗空白的签名球。这颗球提醒他,不管是教导学生还是子女,一定要时时从孩子的角度去看事情。}

\qquad \kai{不论家长还是教师,常常用成人的眼光看待孩子,用成人的思维理解孩子,用成人的保准要求孩子。岂不知,从孩子的角度看事情,用孩子的眼睛看世界,正是儿童教育应当遵循的基本规律。}

\begin{flushright}\kai{摘编自《人民日报》(2012年3月16日)}\end{flushright}

材料2

\qquad \kai{某大学一研究生凭借着设计“醒目药瓶”,摘得了素有“设计界奥斯卡”美誉的2011年度“国际红点奖”概念设计类奖。}

\qquad \kai{在他提供的设计图上,常见的塑料瓶盖的顶上一圈变身为一块圆圆的玻璃。“这是一面凹凸镜,有放大的功能”。他解释说,有了这个药瓶盖,老年人不需要带上老花镜来区别药的类别、服用量等。他的灵感来源于生活中对中老年人群体的关注。有一天,有位老人要吃药,可是药瓶上的字太小了,原本挂在脖子上的老花镜却不见了,急的这位老人团团转。就这样,该同学很长一段时间沉浸在老人世界中。突然有一天灵感迸发,想到“醒目药瓶”这个点子。}

\qquad \kai{有了灵感后,从设计,带写英文翻译说明,再到制作动画,一共才三天时间。也许有人要问,这样的设计看上去很简单,为什么能拿“国际红点奖”呢?他坦然,设计很简单,关键在于设计前把自己想象成老人,这一设计胜在实用。按照测算,不会给药品本身带来额外的成本,推广起来很容易,实用方便。“希望将来这款设计能推向市场,让更多人得到帮助。”}

\qquad \kai{这位研究生说他没有想当名人的“野心”。只期望能从生活中的小处入手,用自己的设计改变生活,让生活更加美好。正如“红点”主席PeterZec博士在颁奖晚会上说的那样:从同学们优秀的设计中,他高兴地看到的是他们所描绘的未来更加美好的世界。}

\begin{flushright}\kai{摘编自《扬子晚报》(2012年3月17日)}\end{flushright}

(1)分析“用孩子的眼睛看世界”和“设计前把自己想象成老人”两事例所体现的认识主体的能动作用。(6分)

(2)“用自己的设计改变生活,让生活更美好”对我们从事实践活动有何意义?(4分)

\clearpage

\hei{35题、结合材料回答问题:}

\qquad \kai{浙江武义县后陈村是全县经济条件较好的村,但由于财务使用不透明,村民从上世纪年代末以来连续向县纪委、街道反映村里的问题,却长期得不到彻底解决。2003年,连续两任村支书因经济问题被查处。2004年初,村里有1000多亩获补偿金1900多万元。在人均发放7000元后,还剩3000多万元,如何处理本资产成为村内的一个难题。为帮助村民寻找从根本上解决问题的办法,由县纪委  的村务监督改革指导小组进驻后陈村。指导组在大量听取村民意见的基础上,决定成立一个相当独立于村委会及村党支部的监督委员会,真正能从根本导航能让村民有效制定全部的权力。6月,后陈村在海选村委会基础上由群众选举产生了全国第一个村级监督组织---村务监督委员会,与村党支部、村委会一起称为“三委会”。}

\qquad \kai{村务监督委员会成立不久,即对村里两口池塘的承包进行了全程监督,结果每口池塘三年从2. 8万元升至5.8万元。2005年,后陈村举行垃圾清理投标会,村委会主任  监委会主任到场监督,不到5分钟结果就出来了,没有人对投标会公正性表示质疑。}

\qquad \kai{监委会成立后,后陈村每年的创收情况,包括出租土地给广告公司做广告牌、旧粮站出租、经营沙场、村留土地上的杉树出售,以及向上级部门申请到的资金补助等,每一笔都要经过监委会审核后公布。2004年,当年的招待开支是23909元,比前些年下降近一半,村干部再不能拿着发票随便报销了。在村财务公开栏前,村民告诉记者:"过去简直是胡来,集体的钱像是干部自己的,现在不一样了。"监委会主任说:"我的职责就是看他们有没有按程序办事,有没有搞暗箱操作。"监委会成立以来,后陈村的固定收入逐年增加,村干部连续8年实现零违纪,村民连续8年实现零上访。村两委已顺利完成了3次换届。最近的一次换届,村两委成员一个没动,全部高票当选,一次通过。}

\qquad \kai{目前,浙江省3万多个行政村,村村建立了村务监督委员会,实现了村级监督组织全覆盖,村务监督委员会这一有效而不需要太大监督成本的权力制衡制度,对建立乡村"阳光权力体系"共建和谐社会带来重要启迪。村务监督委员会这一制度创新已被写进《中华人民共和国村民委员会组织法》,并在全国推行。}

\begin{flushright}\kai{摘编自《人民日报》2012年5月14日。}\end{flushright}

(1)后陈村是如何通过制度创新来保障权利在阳光下运行的?(5分)

(2)全国第一个村务监督委员会的建立对推进基层民主制度建设有何启示?(5分)

\clearpage

\hei{36. 结合材料回答问题:}

材料1

\qquad \kai{1910年,上海人陆士谔在幻想小说《新中国》里记载了一个神奇的梦,梦中主人公随时光穿梭,看到“万国博览会”在上海浦东举行,为方便市民参观,上海滩建成了浦东大铁桥和越江隧道,还造了地铁,工厂中的机器有鬼斧神工之妙,租界的治外法权已经收回,汉语成了世界通用的流行语言……最后梦中人一跤跌醒,却言道:“休说是梦,到那时,真有这景象也未可知。”}

\qquad \kai{1920年,孙中山先生完成《建国方略》一书,书中提出了修建三峡水利、建设高原铁路系统等宏伟设想,构想了工厂遍地、机器轰鸣、高楼大厦矗立城乡、火车轮船繁忙往返的现代化景象,描绘了“万众一心,急起直追,以我五千年文明优秀之民族,应世界之潮流,而建设一政治最修明、人民最安乐之国家”的愿景。}

\qquad \kai{1935年,方志敏在《可爱的中国》中写道:“中国一定有个可赞美的光明前途……到那时候,到处都是活跃跃的创造,到处都是日新月异的进步,欢歌将代替了悲叹,笑脸将代替了哭脸,富裕将代替了贫穷,康健将代替了疾苦,智慧将代替了愚昧,友爱将代替了仇杀,生之快乐将代替了死之悲哀,明媚的花园,将代替了凄凉的荒地!这时,我们民族就可以无愧色的立在人类的面前,而生育我们的母亲,也会最美丽地装饰起来,与世界上各位母亲平等的携手了。”“这么光荣的一天,决不在辽远的将来,而在很近的将来。”}

\begin{flushright}\kai{摘编自《经济日报》(2012年12月12日)、《方志敏文集》}\end{flushright}

材料2

\qquad \kai{2012年11月29日,中共中央总书记习近平到国家博物馆参观<复兴之路》展览,在十九世纪末列强割占领土、设立租借地、划定势力范围示意图前,在鸦片战争期间虎门的大炮前,在反映辛亥革民的文物和照片前,在《共产党宣言》第一个中文全译本前,在《中国共产党的第一个纲领》等反映中国共产党成立的文物和照片前,在李大钊狱中亲笔自述前,在中华人民共和国第一面五星红旗前,在党的十一届三中全会照片前,习近平不时停下脚步,认真观看,仔细询问和了解有关情况,在参观过程中,习近平发表了重要讲话,他提出,每个人都有理想和追求,都有自己的梦想,实现中华民族伟大复兴,就是中华民族近代以来最伟大的梦,中华民族的昨天,可以说是“雄关漫道真如铁”;中华民族的今天,可谓“人间正道是沧桑”;中华民族的明天,可以说是“长风破浪会有事”。经过鸦片战争以来170多年的持续奋斗,中华民族伟大复兴展现出光明的前景,现在,我们比历史上任何时期都更接近中华民族伟大复兴的目标,比历史上任何时期都更有信息、有能力实现这个目标。}

\begin{flushright}\kai{摘编自《人民日报》(2012年11月30日)}\end{flushright}

(1)为什么“实现中华民族伟大复兴,就是中华民族近代以来最伟大复兴的目标”?(4分)

(2)为什么说“现在,我们比历史任何时期都更接近中华民族伟大复兴的目标”?(6分)

\clearpage

\hei{37. 结合材料回答问题:}

\qquad \kai{某图书馆向所有读者免费开放。乞丐、拾荒者和衣衫褴褛的民工小心翼翼进来了,无人阻拦。于是他们便堂而皇之地在馆内读书、看报。有读者对此表示不满,向馆长抱怨说:“图书馆是大雅之堂,如果允许乞丐和拾荒者进入阅读,就是对其他读者的不尊重。”馆长回答说: “我们无权拒绝他们入内阅读,但你有权选择离开。”}

\qquad \kai{此事被发在微博上,顿时触动了社会的神经,引发人们对人文精神的关注和思考。中央电视台等主流媒体对此事进行了报道。一场图书馆办馆理念的大讨论由此引发。}

\qquad \kai{公共图书馆一向更愿意向体面的文化人敞开,常在门口凛然告示:“衣衫不整,拒绝入内”!把读者分成三六九等,拒绝部分人进入。其公益性大打折扣,而该馆馆长希望图书馆成为“每一个读者的天堂”,“无论任何人,只有走进了图书馆,在知识面前都有着同等的权利,不得有高低贵贱之分”。为此,该馆在全国同行中率先推出免费阅读制度,任何人进馆借阅书籍都不需要证件和费用,以体现人道、人文的公共图书馆办馆理念和人性化的服务。}

\qquad \kai{对于图书馆免费开放可能带来的问题,该馆有关负责人感触颇深,自图书馆实行零门槛后,我们不仅没有感到压力增加,反而感觉开放的时间越长,不尊重这种权利的读者越少。我们和读者都被这种和谐的环境所改变,至于进馆要先洗洗手,馆内并没有硬性规定,耳濡目染的时间长了,谁都会自觉地先洗手,然后再阅读。}

\qquad \kai{“如果有天堂,天堂应该是图书馆的模样。”这是文学大师、曾担任阿根廷国立图书馆馆长的博尔赫斯的一句名言。该图书馆向乞丐和拾荒者免费开放,不啻于一轮明亮的太阳,让乞丐和拾荒者在得到温暖的同时,也净化了我们的心灵。}

\begin{flushright}\kai{摘编自《中国青年》(2011年第5期)、《光明日报》(2012年5月10日)}\end{flushright}

(1)从法律角度如何理解“我无权拒绝他们入内阅读,但你有权选择离开”?(5分)

(2)我们处理人际关系有何启示?(5分)

\clearpage

\hei{38题、结合材料回答问题:}

材料一

\qquad \kai{2012年5月,美国发布奥巴马上台后第一个《四年防务评估报告》和《国建安全战略报告》。报告指出,美国国家利益由安全、繁荣、价值和国际失序四个方面组成;美国通过对这些利益的追求,实现国家复兴和全球领导地;相比世界其他地区,亚洲是美国最有所作为的地区。同时,报告将南亚国家归为三个类别:“战略伙伴”和“可预期的战略伙伴”,虽然,美国准备让东盟所有国家成为美国的盟友或是伙伴。}

\qquad \kai{美国卡内基国际和平基金会副总裁包道格说:“过去的10年来,中国在南亚地区不断扩大其利益,取得了有效的成果,这是美国没有做到的。}

\begin{flushright}\kai{摘编自 新华网(2011年5月28日)}\end{flushright}

材料二

\qquad \kai{2011年10月以来,美国高层不断访问亚太地区的国家,参加相关的国际会议。10月下旬,国防部长帕内塔出访印尼,日本和韩国,强调美国将加大在军事部署。11月下旬到12月初,国务卿希拉里。克林顿,先后访问菲律宾、泰国,并对缅甸进行了“历史性访问”,这是1955年以来美国国务卿首次访问缅甸。同时,奥巴马也是展开亚太之行,他参加在夏威夷举行的亚太经合组织第十九次领导人非正式会议,随后出访澳大利亚并前往印尼出席东亚多边峰会,成为参加东亚峰会的首位美国总统。奥巴马在亚太之行中高调宣示,美国是“太平洋大国”,讲“留驻”亚太,通过“坚持核心原则”和与盟友及伙伴的紧密合作,在“塑造”亚太地区未来中发挥更大,更长远的作用。2012年11月8日,奥巴马连任成功不到48小时,即宣布他的首次出访选在东南亚。17日至20日,他不仅访问了泰国,而且对缅甸和柬埔寨进行了历史性首访。在这次东南亚之行中,奥巴马再次强调,美国是一个太平洋国家,亚太地区对美国创造就业机会以及塑造其安全与繁荣至关重要。}

\begin{flushright}\kai{摘编自《人民日报》(2011年12月23日)、《参考消息》(2012年11月21日)等}\end{flushright}

材料三

\qquad \kai{第十一届香格里拉对话暨亚洲安全会议于2012年6月1日至3日在新加坡举行。会议期间,美国国防部长帕内塔发表了题为《美国对亚太的再平衡》的演讲,重点阐释了“亚太再平衡战略”的军事计划,其中包括2020年前在亚太地区保持6个航母舰队,以及将60\%的海军力量部署到亚太地区。国际舆论普遍怀疑,美国战略重心向亚太地区转移的意图是为了遏制中国。}

\begin{flushright}\kai{摘编自 新华网(2012年6月4日)}\end{flushright}

(1) 美国将其全球战略中心转向亚太的原因何在?(5分)

(2)如何看待美国战略中心东移对中国周边安全的影响?(5分)