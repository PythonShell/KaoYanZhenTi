\hei{34题}

\qquad \kai{最近,四川省搞了一次“医患换位体验”活动,让医生以患者的身份挂号、排队、看病、拿药……结果,医生跑前跑后,既受累又受气,一名全程体验了“患者”的医生感慨道:“医生就像拿着个遥控器,把患者指挥得团团转,当患者确实很苦。”}

\qquad \kai{美国医生爱德华·罗森邦行医50年,忽然患上了喉癌,当他重新审视医学、医院和医生时,感慨地说:“站在病床边和躺在病床上所看到的角度完全不同。”他后来在《亲尝我自己的药方》一书中写道:“如果我能从头来过的话,我会以完全不同的方式行医,很不幸的是,生命不给人这种重新来过的机会。”}

\qquad \kai{多年前,有位年轻医生患上甲状腺病,中国医学科学院著名头颈外科专家屠规益为他主刀。当手术结束时,屠教授低下身来说:“对不起,让您受苦了!”这是屠教授术后经常对病人说的一句话,虽然简短,却让这位年轻医生深感震撼。}

\qquad \kai{著名医学家袁法祖早年从医,曾在老师的带领下,为一名中年妇女进行开腹手术。术后没几天,那名妇女就去世了,经解剖发现,患者的死亡与手术并无关系,当时,袁法祖的老师轻轻说了句,“她是4个孩子的妈妈”,就是这句简单的话,让袁法祖至今念念不忘,他知道这句话包含了多少情感,懂得了医生的责任有多重大:医生不仅要看到人身上的病,更要看到生病的人。}

\begin{flushright}\kai{根据《人民日报》有关文章整理}\end{flushright}

结合材料回答问题:

(1)“医生换位体验”活动中蕴含着何种哲理?

(2)从人的本质属性说明为什么“医生不仅要看到人身上的病,更要看到生病的人”。

(3)你在现实生活中遇到类似医患关系的矛盾,按照矛盾辩证法该如何对待和处理?

\clearpage

\hei{35题}

\qquad \kai{IBM公司是世界上最大的信息工业跨国公司之一,从上世纪50年代其致力于计算机行业,并很快在大型计算机业务上占据了统治地位。IBM生产的计算机在技术上常常是最先进的,在某些情况下,他们即使不是最好的,但由于出色的服务和技术支持,他们仍有卓越的信誉。}

\qquad \kai{在整个60年代和70年代,虽然有Control data,Honeywell、Sperry Univac、Burroughs和NCR等企业的竞争,但这些公司都不是其对手,到1980年为止,IBM仍占据全球大型计算机市场超过80\%的份额。大型计算机是IBM的“金母鸡”,毛利高达70\%。}

\qquad \kai{80年代,随着个人计算机和工作站所连接成的网络逐渐取代大型机,日本、欧洲共同体和美国国内许多资本、技术雄厚的企业纷纷涉足这一高风险、高效益的领域。在与苹果、康柏、东芝、戴尔等企业激烈竞争中IBM公司开始走下坡路。}

\qquad \kai{迫于竞争的压力,90年代IBM公司进行了组织改造以降低成本、进行资产重组和资本运营,使公司的股票价格扶摇直上;进行经营战略转型,在保持计算机硬件领域领先地位的同时,成功地实现了向软件服务等高利润领域的转移;实施竞争战略调整,全面提升企业竞争力,重塑其昔日的辉煌。}

\qquad \kai{IBM确立的战略目标是:在所处产业的所有领域都能实现高增长率;在所有领域都有技术和质量卓越的产品,并发挥领导作用;在生产、销售、服务和管理的所有业务活动上,实现最高的效率;确保企业成长所需要的高利润,以便在产业中具有不可动摇的地位。}

\qquad \kai{目前,计算机技术正在向更加“开放型系统”的方向发展。往往主机是一个公司制造的,显示器是另一个公司的;打印机又是第三个公司的,软件是第四个公司的,这些组合起来使整个系统得以运行。在新的技术基础上,计算机行业的企业组织趋向网络化发展,IBM公司面临着新的竞争挑战。IBM公司在垄断和竞争中寻求着未来的发展。}

\begin{flushright}\kai{摘编自【美】J.E. 斯蒂格利茨:《<经济学>小品和案例》及新华网有关资料}\end{flushright}

结合资料回答问题:

(1) 用IBM的案例说明垄断和竞争的关系。

(2) 从IBM公司的发展过程总结垄断资本条件下竞争的新特点。

\clearpage

\hei{36题}

\qquad \kai{从党的建立到抗日时期,中间有北伐战争和十年土地革命战争,我们经过了两次胜利,两次失败。北伐战争胜利了,但是到一九二七年,革命遭到了失败。土地革命战争曾经取得了很大的胜利,红军发展到三十万人,后来又遭到挫折,经过长征,这三十万人缩小到两万多人……在民主革命时期,经过胜利、失败,再胜利、再失败,两次比较,我们才认识了中国这个客观世界。在抗日战争前夜和抗日战争时期,我写了一些论文,例如《中国革命战争的战略问题》、《论持久战》、《新民主主义论》、《<共产党人>发刊词》,替中央起草过一些关于政策、策略的文件,都是革命经验的总结。那些论文和文件,只有在那个时候才能产生,在以前不可能,因为没有经过大风大浪,没有两次胜利和两次失败的比较,还没有充分的经验,还不能充分认识中国革命的规律。}

\qquad \kai{……过去,特别是开始时期,我们只是一股劲儿要革命,至于怎么革法,革些什么,哪些先革,哪些后革,哪些要到下一阶段才革,在一个相当长的时间内,都没有弄清楚,或者说没有完全弄清楚。}

\begin{flushright}\kai{毛泽东《在扩大的中央工作会议上的讲话》(1962年1月30日)}\end{flushright}

结合材料回答问题:

(1) 毛泽东在20世纪60年代初回顾中国共产党在民主革命时期艰难地但是成功地认识中国革命规律的这段历史,是要说明什么问题?

(2) 在改革开放和社会主义现代化建设取得举世瞩目成就的今天,如何看待以毛泽东为主要代表的中国共产党人在社会主义建设方面的艰辛探索?

\clearpage

\hei{37题}

\qquad \kai{在《人民日报》"说句心里话"栏目,重庆市城乡统筹综合改革先行示范区的一位农民说出了这样的心里话:}

\qquad \kai{这些年,党和政府在想办法给农民更多实惠,直补种粮农民,免除了农业税,让我们参加了新型农村合作医疗,这些以前真是想都不敢想啊!}

\qquad \kai{现如今,我们这个村先搞了“农民转市民”试点,全村131人今年全部将农村户口转为城镇户口,由农民开始变为市民啦!}

\qquad \kai{我家承包的土地自愿流转给集体统一经营,每亩补贴我们青苗费4880元,以后每年按照亩产1000斤粮食的市场价补偿我们,村里把流转出来的土地集中起来,引进一些现代农业项目,经营赚了钱,我们可以分红。这些项目优先从村里招聘劳动力,我儿子就可以回来打工,离我们更近了。以前大家都出去打工,地荒在那里,流转以后可以提高土地利用效率,我们又能从中受益,对村里经济也有好处。}

\qquad \kai{我们现在住的房子是17年前盖的,已经破旧了,根据农民转市民的政策,房子拆迁以后,会补偿给我们两套75平方米的搂房,新房子离这不远,政府承诺我们明年9月搬家。}

\qquad \kai{最高兴的是变市民以后,参加了基本养老保险,像我们这样的老人,一次性缴4320元钱,男的从60岁起,女的从55岁起,就可以每个月领156元养老金。}

结合材料回答问题:

(1)结合我国农村改革发展的历史进程,说明为什么一些过去农民“想都不敢想”的问题现在已经解决或正在解决。

(2)通过该示范区的变化,指出建设我国社会主义新农村的主要途径。

\clearpage

\hei{38题}、本题为选做题,请在Ⅰ、Ⅱ两道试题中选取其中一首作答,若两题都回答,只按第Ⅰ道试题的成绩计入总分。

\vspace{6pt}

\hei{选做题Ⅰ}

\hei{材料1}

\qquad \kai{2007年是“卢沟桥事变”70周年,也是中日邦交正常化35周年,温家宝总理应邀于4月中旬对日本进行了正式访问,在两国发表的《中日联合新闻公报》中,确认双方将继续遵循《中日联合声明》、《中日和平友好条约》和《中日联合宣言》的各项原则,努力构筑“基于共同战略利益的互惠关系”。}

\qquad \kai{温总理在日本国会众议院发表的演讲中,引用日本的谚语“尽管风在呼啸,山却不会移动”形容中日关系,引起日本国会议员们的广泛共鸣。}

\qquad \kai{日本防卫大臣石破茂在为中国军舰“深圳”号访日举行的招待会上致词说,实现两国军舰互访,必将促进两国防务领域的深入交流,进一步提高彼此之间的信任关系,推进双方战略互惠关系向前发展。}

\begin{flushright}\kai{摘自《人民日报》、新华网}\end{flushright}

\hei{材料2}

\qquad \kai{日本首相福田康夫就职后明确表示,他作为首相不会去参拜靖国神社,他在就中日邦交正常化35周年致温家宝的贺信中说:日中两国在地理上是无法迁移的“一衣带水”的邻邦,不论今后国际形势如何变化,日中关系对两国而言乃为最重要的双边关系之一却是不会改变的,我愿意致力于构筑日中战略互惠关系。}

\qquad \kai{曾480余次访华的日中协会理事长白西绅一郎认为,发展日中战略互惠关系,除了要“政治、经济两个轮子一起转”之外,还应特别注重扩大日中民间关系,这样才能夯实日中战略互惠关系的基石。}

\begin{flushright}\kai{摘自《人民日报》、《东方早报》}\end{flushright}

\hei{材料3}

\qquad \kai{据日本海关统计,2007年1月-9月,日中双边贸易额为1715.3亿美元,同比增长12.1\%,其中,日本向中国出口786.3亿美元,增长17.8\%,日本自中国进口929.0亿美元,增长7.7\%,日本贸易逆差142.7亿美元,减少26.7\%,中国继续保持日本第二大出口目的地和第一大进口来源国的地位。}

\begin{flushright}\kai{摘自商务部两国别数据库}\end{flushright}

结合材料回答问题:

(1)中日两国“战略互惠关系”的基本精神是什么?

(2)分析温家宝总理用“尽管风在呼啸,山却不会移动”形容中日关系的寓意。

\vspace{6pt}

\hei{选做题Ⅱ}

\hei{材料1}

\qquad \kai{联合国政府间气候变化专门委员会在2007年2月2日就气候问题发出了警告:从现在开始到2100年,全球平均气温的“最可能升高幅度”是1.8摄氏度至4摄氏度,海平面升高幅度是18厘米至58厘米。美国著名智库国际战略研究所的报告认为,"如果温室气体排放仍得不到控制,其灾难性后果不亚于发生一场核战争。"}

\hei{材料2}

\qquad \kai{现在国际上担心中国很快就会成为世界头号污染物排放国:而且再过25年,中国温室气体排放量将超过其他发达国家总各。……中国的高速崛起,会用掉全球大半的能源,加重能源危机;由于巨大的污染物和温室气体的排放量,中国将成为全球最大的污染源,中国是气候变化的主要威胁。}

\hei{材料3}

\qquad \kai{从1950年到2002年,中国化石燃料排放的二氧化碳占世界同期累计排放量的9.33\%(同期发达国家排放量占77\%,而此前的200年间,发达国家更是占到95\%);1950年到2002年的50多年间,中国人均排放量居世界第92位,从单位GDP二氧化碳排放的弹性系数看,1990年到2004年的15年间,单位GDP每增长1\%,世界平均二氧化碳排放增长0.6\%,中国增长0.38\%。}

\begin{flushright}\kai{摘自国家发改委主任马凯在国务院新闻办新闻发布会上的讲话}\end{flushright}

\qquad \kai{2007年12月14日,刚刚参加完印尼巴厘岛联合国气候变化大会的世界银行行长佐利克来到中国,针对近年来中国为节能减排所付出的努力,佐利克说,中国已经形成强烈共识,在发展经济的同时更关注环境保护,并提出了科学发展观。他认为,中国政府在降低能耗、提高车辆能将标准,以及发展全球碳市场等方面发挥了重要作用。……这不仅对中国本身发展意义重大,也将为全球应对气候变化挑战做出贡献。}

\begin{flushright}\kai{摘自中国广播网有关报道}\end{flushright}

结合材料回答问题:

(1)上述材料中所反映的气候变化的严峻事实对我们理解自然环境在社会发展中的作用有何启示?

(2)评析“中国气候威胁论”并指出中国应对气候变化问题的战略选择。
