\qquad In 1924 American' National Research Council sent to engineers to supervise a series of industrial experiments at a large telephone-parts factory called the Hawthorne Plant near Chicago. It hoped they would learn how stop-floor lignting \underline{\quad 1\quad} workers productivity. Instead, the studies ended \underline{\quad 2\quad} giving their name to the ``Hawthorne effect'', the extremely influential idea that the very \underline{\quad 3\quad} to being experimented upon changed subjects' behavior.

\qquad The idea arose because of the \underline{\quad 4\quad} behavior of the women in the Hawthorne plant. According to \underline{\quad 5\quad} of the experiments, their hourly output rose when lighting was increased, but also when it was dimmed. It did not \underline{\quad 6\quad} what was done in the experiment; \underline{\quad 7\quad} someting was changed, productivity rose. A(n) \underline{\quad 8\quad} that they were being experimented upon seemed to be \underline{\quad 9\quad} to alter workers' behavior \underline{\quad 10\quad} itself.

\qquad After several decades, the same data were \underline{\quad 11\quad} to econometric the analysis. Hawthorne experiments has another surprise store \underline{\quad 12\quad} the descriptions on record, no systematic \underline{\quad 13\quad} was found that levels of productivity were related to changes in lighting.

\qquad It turns out that peculiar way of conducting the experiments may be have let to \underline{\quad 14\quad} interpretation of what happed. \underline{\quad 15\quad}, lighting was always changed on a Sunday . When work started again on Monday, output \underline{\quad 16\quad} rose compared with the previous Saturday and \underline{\quad 17\quad} to rise for the next couple of days. \underline{\quad 18\quad}, a comparison with data for weeks when there was no experimentation showed that output always went up on Monday, workers \underline{\quad 19\quad} to be diligent for the first few days of the week in any case, before \underline{\quad 20\quad} a plateau and then slackening off. This suggests that the alleged ``Hawthorne effect'' is hard to pin down.

01. \choice{  affected }{  achieved }{  extracted  }{  restored}
02. \choice{  at  }{ up}{  with      }{  off}
03. \choice{ truth   }{ sight    }{  act  }{  proof}
04. \choice{  controversial    }{  perplexing    }{ mischievous   }{  ambiguous}
05. \choice{ requirements     }{ explanations   }{  accounts   }{  assessments}
06. \choice{  conclude    }{  matter}{  indicate     }{  work}
07. \choice{  as far as     }{  for fear that   }{  in case that    }{  so long as}
08. \choice{  awareness}{  expectation  }{  sentiment     }{  illusion}
09. \choice{  suitable}{  excessive     }{  enough  }{  abundant}
10. \choice{  about     }{  for}{  on    }{  by}
11. \choice{  compared  }{ shown   }{  subjected   }{  conveyed}
12. \choice{  contrary to  }{  consistent with }{  parallel with   }{  pealliar to}
13. \choice{  evidence }{ guidance      }{ implication   }{ source}
14. \choice{  disputable }{ enlightening   }{ reliable    }{ misleading}
15. \choice{  In contrast     }{  For example   }{  In consequence }{  As usual}
16. \choice{  duly     }{ accidentally   }{  unpredictably  }{  suddenly}
17. \choice{ failed        }{ ceased      }{ started     }{ continued}
18. \choice{ Therefore        }{ Furthermore      }{ However     }{ Meanwhile}
19. \choice{ Attempted       }{ tended      }{ chose     }{ intenced}
20. \choice{ breaking }{ climbing   }{ surpassing     }{ hiting}