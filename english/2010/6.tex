%% Content from http://zhenti.kaoyan.eol.cn/
%% Format by PythonShell
%% 2014-01-08

\qquad [A] The first and more important is the consumer's growing preference for eating out; the consumption of food and drink in places other than homes has risen from about 32 percent of total consumption in 1995 to 35 percent in 2000 and is expected to approach 38 percent by 2005. This development is boosting wholesale demand from the food service segment by 4 to 5 percent a year across Europe, compared with growth in retail demand of 1 to 2 percent. Meanwhile, as the recession is looming large, people are getting anxious. They tend to keep a tighter hold on their purse and consider eating at home a realistic alternative.

\qquad [B] Retail sales of food and drink in Europe's largest markets are at a standstill, leaving European grocery retailers hungry for opportunities to grow. Most leading retailers have already tried e-commerce, with limited success, and expansion abroad. But almost all have ignored the big, profitable opportunity in their own backyard: the wholesale food and drink trade, which appears to be just the kind of market retailers need.

\qquad [C] Will such variations bring about a change in the overall structure of the food and drink market? Definitely not. The functioning of the market is based on flexible trends dominated by potential buyers. In other words, it is up to the buyer, rather than the seller, to decide what to buy .At any rate, this change will ultimately be acclaimed by an ever-growing number of both domestic and international consumers, regardless of how long the current consumer pattern will take hold.

\qquad [D] All in all, this clearly seems to be a market in which big retailers could profitably apply their scale, existing infrastructure and proven skills in the management of product ranges, logistics, and marketing intelligence. Retailers that master the intricacies of wholesaling in Europe may well expect to rake in substantial profits thereby. At least, that is how it looks as a whole. Closer inspection reveals important differences among the biggest national markets, especially in their customer segments and wholesale structures, as well as the competitive dynamics of individual food and drink categories. Big retailers must understand these differences before they can identify the segments of European wholesaling in which their particular abilities might unseat smaller but entrenched competitors. New skills and unfamiliar business models are needed too.

\qquad [E] Despite variations in detail, wholesale markets in the countries that have been closely examined-France, Germany, Italy, and Spain-are made out of the same building blocks. Demand comes mainly from two sources: independent mom-and-pop grocery stores which, unlike large retail chains, are two small to buy straight from producers, and food service operators that cater to consumers when they don't eat at home. Such food service operators range from snack machines to large institutional catering ventures, but most of these businesses are known in the trade as ``horeca'': hotels, restaurants, and caf\'{e}s. Overall, Europe's wholesale market for food and drink is growing at the same sluggish pace as the retail market, but the figures, when added together, mask two opposing trends.

\qquad [F] For example, wholesale food and drink sales come to \$168 billion in France, Germany, Italy, Spain, and the United Kingdom in 2000-more than 40 percent of retail sales. Moreover, average overall margins are higher in wholesale than in retail; wholesale demand from the food service sector is growing quickly as more Europeans eat out more often; and changes in the competitive dynamics of this fragmented industry are at last making it feasible for wholesalers to consolidate.

\qquad [G] However, none of these requirements should deter large retailers (and even some large good producers and existing wholesalers) from trying their hand, for those that master the intricacies of wholesaling in Europe stand to reap considerable gains.

\[41. \to 42. \to 43. \to 44. \to E \to 45.\]