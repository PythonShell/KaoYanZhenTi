\begin{center}\textbf{Text 3}\end{center}

\qquad In his book The Tipping Point, Malcolm Aladuell argues that social epidemics are driven in large part by the acting of a tiny minority of special individuals, often called influentials, who are unusually informed, persuasive, or well-connected. The idea is intuitively compelling, but it doesn't explain how ideas actually spread.

\qquad The supposed importance of influentials derives from a plausible sounding but largely untested theory called the ``two step flow of communication'': Information flows from the media to the influentials and from them to everyone else. Marketers have embraced the two-step flow because it suggests that if they can just find and influence the influentials, those selected people will do most of the work for them. The theory also seems to explain the sudden and unexpected popularity of certain looks, brands, or neighborhoods. In many such cases, a cursory search for causes finds that some small group of people was wearing, promoting, or developing whatever it is before anyone else paid attention. Anecdotal evidence of this kind fits nicely with the idea that only certain special people can drive trends

\qquad In their recent work, however, some researchers have come up with the finding that influentials have far less impact on social epidemics than is generally supposed. In fact, they don't seem to be required of all.

\qquad The researchers' argument stems from a simple observing about social influence, with the exception of a few celebrities like Oprah Winfrey-whose outsize presence is primarily a function of media, not interpersonal, influence-even the most influential members of a population simply don't interact with that many others. Yet it is precisely these non-celebrity influentials who, according to the two-step-flow theory, are supposed to drive social epidemics by influencing their friends and colleagues directly. For a social epidemic to occur, however, each person so affected, must then influence his or her own acquaintances, who must in turn influence theirs, and so on; and just how many others pay attention to each of these people has little to do with the initial influential. If people in the network just two degrees removed from the initial influential prove resistant, for example from the initial influential prove resistant, for example the cascade of change won't propagate very far or affect many people.

\qquad Building on the basic truth about interpersonal influence, the researchers studied the dynamics of populations manipulating a number of variables relating of populations, manipulating a number of variables relating to people's ability to influence others and their tendency to be influenced. Our work shows that the principal requirement for what we call ``global cascades''--the widespread propagation of influence through networks--is the presence not of a few influentials but, rather, of a critical mass of easily influenced people, each of whom adopts, say, a look or a brand after being exposed to a single adopting neighbor. Regardless of how influential an individual is locally, he or she can exert global influence only if this critical mass is available to propagate a chain reaction.

31.By citing the book The Tipping Point, the author intends to\par
	\choice{ analyze the consequences of social epidemics}{ discuss influentials’ function in spreading ideas}{ exemplify people’s intuitive response to social epidemics}{ describe the essential characteristics of influentials.}

32.The author suggests that the “two-step-flow theory”\par
	\choice{ serves as a solution to marketing problems}{ has helped explain certain prevalent trends}{ has won support from influentials}{ requires solid evidence for its validity}

33.what the researchers have observed recently shows that\par
	\choice{  the power of influence goes with social interactions}{  interpersonal links can be enhanced through the media}{  influentials have more channels to reach the public}{  most celebrities enjoy wide media attention}

34.The underlined phrase “these people” in paragraph 4 refers to the ones who\par
	\choice{ stay outside the network of social influence}{have little contact with the source of influence}{  are influenced and then influence others}{  are influenced by the initial influential}

35.what is the essential element in the dynamics of social influence?\par
	\choice{ The eagerness to be accepted}{ The impulse to influence others}{ The readiness to be influenced}{ The inclination to rely on others}