%% Content from http://zhenti.kaoyan.eol.cn/
%% Format by PythonShell
%% 2014-01-08

\qquad One basic weakness in a conservation system based wholly on economic motives is that most members of the land community  have no economic value. Yet these creatures are members of the biotic community and, if its stability depends on its integrity, they are entitled to continuance.

\qquad When one of these non-economic categories is threatened and, if we happen to love it. We invert excuses to give it economic importance. At the beginning of century songbirds were supposed to be disappearing. \ul{(46) Scientists jumped to the rescue with some distinctly shaky evidence to the effect that insects would eat us up if birds failed to control them.} The evidence had to be economic in order to be valid.

\qquad It is painful to read these round about accounts today. We have no land ethic yet, \ul{(47) but we have at least drawn near the point of admitting that birds should continue as a matter of intrinsic right, regardless of the presence or absence of economic advantage to us.}

\qquad A parallel situation exists in respect of predatory mammals and fish-eating birds. \ul{(48) Time was when biologists somewhat overworked the evidence that these creatures preserve the health of game by killing the physically weak, or that they prey only on ``worthless'' species.} Here again, the evidence had to be economic in order to be vaild. It is only in recent years that we hear the more honest argument that predators are members of the community, and that no special interest has the right to exterminate them for the sake of benifit, real or fancied, to itself.

\qquad Some species of tree have been ``read out of the party'' by economics-minded foresters because they grow too slowly, or have too low a sale vale to pay as timber crops. \ul{(49) In Europe, where forestry is ecologically more advanced, the non-commercial tree species are recognized as members of the native forest community, to be preserved as such, within reason.} Moreover some have been found to have a variable function in building up soil fertility. The interdependence of the forest and its constituent tree species, groud flora, and fauna is taken for granted.

\qquad To sum up: a system of conservation based solely on economic self-interest is hopelessly lopsided. \ul{(50) It tends to ignore, and thus eventually to eliminate, many elements in the land community that lack commercial value, but that are essential to its healthy functioning.} It assumes, falsely, I think, that the economic parts of the biotic clock will function without the uneconomic parts.