\begin{center}\textbf{Text 1}\end{center}

\qquad Of all the changes that have taken place in English-language newspapers during the past quarter-century, perhaps the most far-reaching has been the inexorable decline in the scope and seriousness of their arts coverage. Not only have many newspapers done away with their book-review sections, but several major papers, including the Chicago Sun-Times and the Minneapolis Star-Tribune, no longer employ full-time classical-music critics. Even those papers that continue to review fine-arts events are devoting less space to them, while the “think pieces” on cultural subjects that once graced the pages of big-city Sunday papers are becoming a thing of the past.

\qquad It is, I suspect, difficult to the point of impossibility for the average reader under the age of forty to imagine a time when high-quality arts criticism could be found in most big-city newspapers. Yet a considerable number of the most significant collections of criticism published in the 20th century, including Virgil Thomson’s The Musical Scene (1945), Edwin Denby’s Looking at the Dance (1949), Kenneth Tynan’s Curtains (1961), and Hilton Kramer’s The Age of the Avant-Garde (1973) consisted in large part of newspaper reviews. To read such books today is to marvel at the fact that their erudite contents were once deemed suitable for publication in general-circulation dailies.

\qquad We are even farther removed from the discursive newspaper reviews published in England between the turn of the 20th century and the eve of World War II, at a time when newsprint was dirt-cheap and stylish arts criticism was considered an ornament to the publications in which it appeared. In those far-off days, it was taken for granted that the critics of major papers would write in detail and at length about the events they covered.1 Theirs was a serious business, and even those reviewers who wore their learning lightly, like George Bernard Shaw and Ernest Newman, could be trusted to know what they were about. These men (for they were all men) believed in journalism as a calling, and were proud to be published in the daily press. “So few authors have brains enough or literary gift enough to keep their own end up in journalism,” Newman wrote, “that I am tempted to define ‘journalism’ as ‘a term of contempt applied by writers who are not read to writers who are.’”

\qquad Why, then, are virtually all of these critics forgotten? Neville Cardus, who wrote for the Manchester Guardian from 1917 until shortly before his death in 1975, is now known solely as a writer of essays on the game of cricket. During his lifetime, though, he was also one of England’s foremost classical-music critics, a stylist so widely admired that his Autobiography (1947) became a best-seller. He was knighted in 1967, the first music critic to be so honored. Yet only one of his books is now in print, and his vast body of writings on music is unknown save to specialists. How is it possible that so celebrated a critic should have slipped into near-total obscurity?

21. It is indicated in Paragraphs 1 and 2 that\par
	\choice{arts criticism has disappeared from big-city newspapers.}{English-language newspapers used to carry more arts reviews.}{high-quality newspapers retain a large body of readers.}{young readers doubt the suitability of criticism on dailies.}

22. Newspaper reviews in England before World War 2 were characterized by\par
	\choice{free themes.}{casual style.}{elaborate layout.}{radical viewpoints.}

23. Which of the following would shaw and Newman most probably agree on?\par
	\choice{It is writers' duty to fulfill journalistic goals.}{It is contemptible for writers to be journalists.}{Writers are likely to be tempted into journalism.}{Not all writers are capable of journalistic writing.}

24. What can be learned about Cardus according to the last two paragraphs?\par
	\choice{His music criticism may not appeal to readers today.}{His reputation as a music critic has long been in dispute.}{His style caters largely to modern specialists.}{His writings fail to follow the amateur tradition.}

25. What would be the best title for the text?\par
	\choice{Newspapers of the Good Old Days}{The Lost Horizon in Newspapers}{Mournful Decline of Journalism}{Prominent Critics in Memory}