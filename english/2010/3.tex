%% Content from http://zhenti.kaoyan.eol.cn/
%% Format by PythonShell
%% 2014-01-08

\begin{center}\textbf{Text 2}\end{center}

\qquad Over the past decade, thousands of patents have been granted for what are called business methods. Amazon.com received one for its ``one-click'' online payment system. Merrill Lynch got legal protection for an asset allocation strategy. One inventor patented a technique for lifting a box.

\qquad Now the nation's top patent court appears completely ready to scale back on business-method patents, which have been controversial ever since they were first authorized 10 years ago. In a move that has intellectual-property lawyers abuzz, the U.S. court of Appeals for the Federal Circuit said it would use a particular case to conduct a broad review of  business-method patents. In \emph{re Bilski}, as the case is known, is ``a very big deal'', says Dennis D. Crouch of the University of Missouri School of law. It ``has the potential to eliminate an entire class of patents.''

\qquad Curbs on business-method claims would be a dramatic about-face, because it was the Federal Circuit itself that introduced such patents with its 1998 decision in the so-called state Street Bank case, approving a patent on a way of pooling mutual-fund assets. That ruling produced an explosion in business-method patent filings, initially by emerging Internet companies trying to stake out exclusive rights to specific types of online transactions. Later, move established companies raced to add such patents to their files, if only as a defensive move against rivals that might beat them to the punch. In 2005, IBM noted in a court filing that it had been issued more than 300 business-method patents despite the fact that it questioned the legal basis for granting them. Similarly, some Wall Street investment films armed themselves with patents for financial products, even as they took positions in court cases opposing the practice.

\qquad The Bilski case involves a claimed patent on a method for hedging risk in the energy market. The Federal Circuit issued an unusual order stating that the case would be heard by all 12 of the court's judges, rather than a typical panel of three, and that one issue it wants to evaluate is whether it should ``reconsider'' its State Street Bank ruling.

\qquad The Federal Circuit's action comes in the wake of  a  series of recent decisions by the Supreme  Count that has narrowed the scope of protections for patent holders. Last April, for example the justices signaled that too many patents were being upheld for ``inventions'' that are obvious. The judges on the Federal Circuit are ``reacting to the anti-patent trend at the Supreme Court'', says Harole C. Wegner, a partend attorney and professor at George Washington University Law School.

\vspace{6pt}

26. Business-method patents have recently aroused concern because of\par
	\choice{ their limited value to business}{ their connection with asset allocation}{ the possible restriction on their granting}{ the controversy over authorization}

27. Which of the following is true of the Bilski case?\par
	\choice{ Its ruling complies with the court decisions}{ It involves a very big business transaction}{ It has been dismissed by the Federal Circuit}{ It may change the legal practices in the U.S.}

28. The word ``about-face'' (Line 1, Para 3) most probably means\par
	\choice{ loss of good will}{ increase of hostility}{ change of attitude}{ enhancement of dignity}

29. We learn from the last two paragraphs that business-method patents\par
	\choice{ are immune to legal challenges}{ are often unnecessarily issued}{ lower the esteem for patent holders}{ increase the incidence of risks}

30. Which of the following would be the subject of the text?\par
	\choice{ A looming threat to business-method patents}{ Protection for business-method patent holders}{ A legal case regarding business-method patents}{ A prevailing trend against business-method patents}