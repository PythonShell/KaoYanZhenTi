%% Content from http://zhenti.kaoyan.eol.cn/
%% Format by PythonShell
%% 2014-01-08

\begin{center}\textbf{Text 4}\end{center}

\qquad Bankers have been blaming themselves for their troubles in public. Behind the scenes, they have been taking aim at someone else: the accounting standard-setters. Their rules, moan the banks, have forced them to report enormous losses, and it's just not fair. These rules say they must value some assets at the price a third party would pay, not the price managers and regulators would like them to fetch.

\qquad Unfortunately, banks' lobbying now seems to be working. The details may be unknowable, but the independence of standard-setters, essential to the proper functioning of capital markets, is being compromised. And, unless banks carry toxic assets at prices that attract buyers, reviving the banking system will be difficult.

\qquad After a bruising encounter with Congress, America's Financial Accounting Standards Board (FASB) rushed through rule changes. These gave banks more freedom to use models to value illiquid assets and more flexibility in recognizing losses on long-term assets in their income statement. Bob Herz, the FASB's chairman, cried out against those who ``question our motives.'' Yet bank shares rose and the changes enhance what one lobby group politely calls ``the use of judgment by management.''

\qquad European ministers instantly demanded that the International Accounting Standards Board (IASB) do likewise. The IASB says it does not want to act without overall planning, but the pressure to fold when it completes it reconstruction of rules later this year is strong. Charlie McCreevy, a European commissioner, warned the IASB that it did ``not live in a political vacuum'' but ``in the real word'' and that Europe could yet develop different rules.

\qquad It was banks that were \ul{on the wrong planet}, with accounts that vastly overvalued assets. Today they argue that market prices overstate losses, because they largely reflect the temporary illiquidity of markets, not the likely extent of bad debts. The truth will not be known for years. But bank's shares trade below their book value, suggesting that investors are skeptical. And dead markets partly reflect the paralysis of banks which will not sell assets for fear of booking losses, yet are reluctant to buy all those supposed bargains.

\qquad To get the system working again, losses must be recognized and dealt with. America's new plan to buy up toxic assets will not work unless banks mark assets to levels which buyers find attractive. Successful markets require independent and even combative standard-setters. The FASB and IASB have been exactly that, cleaning up rules on stock options and pensions, for example, against hostility form special interests. But by giving in to critics now they are inviting pressure to make more concessions.

\vspace{6pt}

36. Bankers complained that they were forced to\par
	\choice{  follow unfavorable asset evaluation rules}{ collect payments from third parties}{ cooperate with the price managers}{ reevaluate some of their assets.}

37. According to the author, the rule changes of the FASB may result in\par
	\choice{ the diminishing role of management}{ the revival of the banking system}{ the banks' long-term asset losses}{ the weakening of its independence}

38. According to Paragraph 4, McCreevy objects to the IASB's attempt to\par
	\choice{ keep away from political influences.}{ evade the pressure from their peers.}{ act on their own in rule-setting.}{ take gradual measures in reform.}

39. The author thinks the banks were ``on the wrong planet'' in that they\par
	\choice{ misinterpreted market price indicators}{ exaggerated the real value of their assets}{ neglected the likely existence of bad debts.}{ denied booking losses in their sale of assets.}

40. The author's attitude towards standard-setters is one of\par
	\choice{ satisfaction.}{ skepticism.}{ objectiveness}{ sympathy}