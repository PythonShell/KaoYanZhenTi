46. 科学家们急忙介入,但提供的证据显然站不住脚,其大意是,如果鸟类不能控制昆虫的数量,昆虫便会吞噬我们人类。

47. 但是我们至少近乎承认,无论鸟类能否带给我们经济价值,它们自有生存下去的权利。

48. 有证据表明,这些生物杀死体弱者来保持种群的健康,或者说它们仅仅捕食“没有价值”的物种。曾经有段时间,生物学家或多或少滥用这一证据。

49. 在林业生态更为发达的欧洲,没有商业价值的树种被合理地看成是当地森林群落的成员,并得到相应的保护。

50. 这种保护体系往往忽视陆地群落中诸多缺乏商业价值但对其健康运作至关重要的物种,而最终导致它们的灭绝。