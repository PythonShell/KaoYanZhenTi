%% Content from http://zhenti.kaoyan.eol.cn/
%% Format by PythonShell
%% 2014-01-08

\begin{center}\textbf{Text 4}\end{center}

\qquad The most thoroughly studied intellectuals in the history of the New World are the ministers and political leaders of seventeenth-century New England. According to the standard history of American philosophy, nowhere else in colonial America was ``So much important attached to intellectual pursuits'' According to many books and articles, New England's leaders established the basic themes and preoccupations of an unfolding, dominant Puritan tradition in American intellectual life.

\qquad To take this approach to the New Englanders normally mean to start with the Puritans' theological innovations and their distinctive ideas about the church-important subjects that we may not neglect. But in keeping with our examination of southern intellectual life, we may consider the original Puritans as carriers of European culture adjusting to New world circumstances. The New England colonies were the scenes of important episodes in the pursuit of widely understood ideals of civility and virtuosity.

\qquad The early settlers of Massachusetts Bay included men of impressive education and influence in England. Besides the ninety or so learned ministers who came to Massachusetts churches in the decade after 1629, there were political leaders like John Winthrop, an educated gentleman, lawyer, and official of the Crown before he journeyed to Boston. These men wrote and published extensively, reaching both New World and Old World audiences, and giving New England an atmosphere of intellectual earnestness.

\qquad We should not forget, however, that most New Englanders were less well educated. While few craftsmen or farmers, let alone dependents and servants, left literary compositions to be analyzed, it is obvious that their views were less fully intellectualized. Their thinking often had a traditional superstitions quality. A tailor named John Dane, who emigrated in the late 1630s, left an account of his reasons for leaving England that is filled with signs. Sexual confusion, economic frustrations, and religious hope-all name together in a decisive moment when he opened the Bible, told his father that the first line he saw would settle his fate, and read the magical words: ``Come out from among them, touch no unclean thing, and I will be your God and you shall be my people.'' One wonders what Dane thought of the careful sermons explaining the Bible that he heard in Puritan churched.

\qquad Meanwhile, many settles had slighter religious commitments than Dane's, as one clergyman learned in confronting folk along the coast who mocked that they had not come to the New world for religion. ``Our main end was to catch fish. ''

\vspace{6pt}

36. The author holds that in the seventeenth-century New England\par
	\choice{Puritan tradition dominated political life.}{intellectual interests were encouraged.}{politics benefited much from intellectual endeavors.}{intellectual pursuits enjoyed a liberal environment.}

37. It is suggested in Paragraph 2 that New Englanders\par
	\choice{experienced a comparatively peaceful early history.}{brought with them the culture of the Old World}{paid little attention to southern intellectual life}{were obsessed with religious innovations}

38. The early ministers and political leaders in Massachusetts Bay\par
	\choice{were famous in the New World for their writings}{gained increasing importance in religious affairs}{abandoned high positions before coming to the New World}{created a new intellectual atmosphere in New England}

39. The story of John Dane shows that less well-educated New Englanders were often\par
	\choice{influenced by superstitions}{troubled with religious beliefs}{puzzled by church sermons}{frustrated with family earnings}

40. The text suggests that early settlers in New England\par
	\choice{were mostly engaged in political activities}{were motivated by an illusory prospect}{came from different backgrounds.}{left few formal records for later reference}