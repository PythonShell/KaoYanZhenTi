\begin{center}\textbf{Text 3}\end{center}

\qquad The relationship between formal education and economic growth in poor countries is widely misunderstood by economists and politicians alike progress in both area is undoubtedly necessary for the social, political and intellectual development of these and all other societies; however, the conventional view that education should be one of the very highest priorities for promoting rapid economic development in poor countries is wrong. We are fortunate that is it, because new educational systems there and putting enough people through them to improve economic performance would require two or three generations. The findings of a research institution have consistently shown that workers in all countries can be trained on the job to achieve radical higher productivity and, as a result, radically higher standards of living.

\qquad Ironically, the first evidence for this idea appeared in the United States. Not long ago, with the country entering a recessing and Japan at its pre-bubble peak. The U.S. workforce was derided as poorly educated and one of primary cause of the poor U.S. economic performance. Japan was, and remains, the global leader in automotive-assembly productivity. Yet the research revealed that the U.S. factories of Honda Nissan, and Toyota achieved about 95 percent of the productivity of their Japanese countere pants a result of the training that U.S. workers received on the job.

\qquad More recently, while examing housing construction, the researchers discovered that illiterate, non-English- speaking Mexican workers in Houston, Texas, consistently met best-practice labor productivity standards despite the complexity of the building industry’s work.

\qquad What is the real relationship between education and economic development? We have to suspect that continuing economic growth promotes the development of education even when governments don’t force it. After all, that’s how education got started. When our ancestors were hunters and gatherers 10,000 years ago, they didn’t have time to wonder much about anything besides finding food. Only when humanity began to get its food in a more productive way was there time for other things.

\qquad As education improved, humanity’s productivity potential, they could in turn afford more education. This increasingly high level of education is probably a necessary, but not a sufficient, condition for the complex political systems required by advanced economic performance. Thus poor countries might not be able to escape their poverty traps without political changes that may be possible only with broader formal education. A lack of formal education, however, doesn’t constrain the ability of the developing world’s workforce to substantially improve productivity for the forested future. On the contrary, constraints on improving productivity explain why education isn’t developing more quickly there than it is.

31. The author holds in paragraph 1 that the important of education in poor countries  .\par
	\choice{ is subject groundless doubts}{ has fallen victim of bias}{ is conventional downgraded}{ has been overestimated}

32. It is stated in paragraph 1 that construction of a new education system  .\par
	\choice{challenges economists and politicians}{takes efforts of generations}{ demands priority from the government}{ requires sufficient labor force}

33.A major difference between the Japanese and U.S workforces is that .\par
	\choice{ the Japanese workforce is better disciplined }{ the Japanese workforce is more productive}{the U.S workforce has a better education }{ the U.S workforce is more organize}

34. The author quotes the example of our ancestors to show that education emerged .\par
	\choice{ when people had enough time}{ prior to better ways of finding food}{ when people on longer went hung }{ as a result of pressure on government}

35. According to the last paragraph , development of education .\par
	\choice{ results directly from competitive environments                                 }{ does not depend on economic performance}{ follows improved productivity                            }{ cannot afford political changes }