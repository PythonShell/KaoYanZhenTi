%% Content from http://zhenti.kaoyan.eol.cn/
%% Format by PythonShell
%% 2014-01-08

\qquad Coinciding with the groundbreaking theory of biological evolution proposed by British naturalist Charles Darwin in the 1860s, British social philosopher Herbert Spencer put forward his own theory of biological and cultural evolution. Spencer argued that all worldly phenomena, including human societies, changed over time, advancing toward perfection. \ul{(41) \qquad\qquad\qquad\qquad}.

\qquad American social scientist Lewis Henry Morgan introduced another theory of cultural evolution in the late 1800s. Morgan helped found modern anthropology. In his work, he attempted to show how all aspects of culture changed together in the evolution of societies. \ul{(42) \qquad\qquad\qquad\qquad}.

\qquad In the early 1900s in North America, German-born American anthropologist Franz Boas developed a new theory of culture known as historical particularism. Historical particularism, which emphasized the uniqueness of all cultures, gave new direction to anthropology. \ul{(43) \qquad\qquad\qquad\qquad}.

\qquad Boas felt that the culture of any society must be understood as the result of a unique history and not as one of many cultures belonging to a broader evolutionary stage or type of culture. \ul{(44) \qquad\qquad\qquad\qquad}.

\qquad Historical particularism became a dominant approach to the study of culture in American anthropology, largely through the influence of many students of Boas. But a number of anthropologists in the early 1900s also rejected the particularist theory of culture in favor of diffusionism. Some attributed virtually every important cultural achievement to the inventions of a few, especially gifted peoples that, according to diffusionists, then spread to other cultures. \ul{(45) \qquad\qquad\qquad\qquad}.

\qquad Also in the early 1900s, French sociologist \'{E}mile Durkheim developed a theory of culture that would greatly influence anthropology. Durkheim proposed that religious beliefs functioned to reinforce social solidarity. An interest in the relationship between the function of society and culture became a major theme in European, and especially British, anthropology.

\vspace{6pt}

\qquad [A] Other anthropologists believed that cultural innovations, such as inventions, had a single origin and passed from society to society. This theory was known as diffusionism.

\qquad [B] In order to study particular cultures as completely as possible, Boas became skilled in linguistics, the study of languages, and in physical anthropology, the study of human biology and anatomy.

\qquad [C] He argued that human evolution was characterized by a struggle he called the ``survival of the fittest,'' in which weaker races and societies must eventually be replaced by stronger, more advanced races and societies.

\qquad [D] They also focused on important rituals that appeared to preserve a people's social structure, such as initiation ceremonies that formally signify children's entrance into adulthood.

\qquad [E] Thus, in his view, diverse aspects of culture, such as the structure of families, forms of marriage, categories of kinship, ownership of property, forms of government, technology, and systems of food production, all changed as societies evolved.

\qquad [F] Supporters of the theory viewed as a collection of integrated parts that work together to keep a society functioning.

\qquad [G] For example, British anthropologists Grafton Elliot Smith and W. J. Perry incorrectly suggested, on the basis of inadequate information, that farming, pottery making, and metallurgy all originated in ancient Egypt and diffused throughout the world. In fact, all of these cultural developments occurred separately at different times in many parts of the world.
