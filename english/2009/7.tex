%% Content from http://zhenti.kaoyan.eol.cn/
%% Format by PythonShell
%% 2014-01-08

\qquad There is a marked difference between the education which every one gets from living with others, and the deliberate educating of the young. In the former case the education is incidental; it is natural and important, but it is not the express reason of the association. \ul{(46) It may be said that the measure of the worth of any social institution is its effect in enlarging and improving experience; but this effect is not a part of its original motive.} Religious associations began, for example, in the desire to secure the favor of overruling powers and to ward off evil influences; family life in the desire to gratify appetites and secure family perpetuity; systematic labor, for the most part, because of enslavement to others, etc. \ul{(47) Only gradually was the by-product of the institution noted, and only more gradually still was this effect considered as a directive factor in the conduct of the institution.} Even today, in our industrial life, apart from certain values of industriousness and thrift, the intellectual and emotional reaction of the forms of human association under which the world's work is carried on receives little attention as compared with physical output.

\qquad But in dealing with the young, the fact of association itself as an immediate human fact, gains in importance. \ul{(48) While it is easy to ignore in our contact with them the effect of our acts upon their disposition, it is not so easy as in dealing with adults.} The need of training is too evident; the pressure to accomplish a change in their attitude and habits is too urgent to leave these consequences wholly out of account. \ul{(49) Since our chief business with them is to enable them to share in a common life we cannot help considering whether or not we are forming the powers which will secure this ability.} If humanity has made some headway in realizing that the ultimate value of every institution is its distinctively human effect we may well believe that this lesson has been learned largely through dealings with the young.

\qquad \ul{(50) We are thus led to distinguish, within the broad educational process which we have been so far considering, a more formal kind of education -- that of direct tuition or schooling.} In undeveloped social groups, we find very little formal teaching and training. These groups mainly rely for instilling needed dispositions into the young upon the same sort of association which keeps the adults loyal to their group.