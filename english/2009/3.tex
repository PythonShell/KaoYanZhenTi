\begin{center}\textbf{Text 2}\end{center}

\qquad It is a wise father that knows his own child, but today a man can boost his paternal (fatherly) wisdom – or at least confirm that he's the kid's dad. All he needs to do is shell our \$30 for paternity testing kit (PTK) at his local drugstore – and another \$120 to get the results.

\qquad More than 60,000 people have purchased the PTKs since they first become available without prescriptions last years, according to Doug Fog, chief operating officer of Identigene, which makes the over-the-counter kits. More than two dozen companies sell DNA tests Directly to the public, ranging in price from a few hundred dollars to more than \$2500.

\qquad Among the most popular: paternity and kinship testing, which adopted children can use to find their biological relatives and latest rage a many passionate genealogists-and supports businesses that offer to search for a family's geographic roots.

\qquad Most tests require collecting cells by webbing saliva in the mouth and sending it to the company for testing.  All tests require a potential candidate with whom to compare DNA.

\qquad But some observers are skeptical, “There is a kind of false precision being hawked by people claiming they are doing ancestry testing,” says Trey Duster, a New York University sociologist. He notes that each individual has many ancestors-numbering in the hundreds just a few centuries back. Yet most ancestry testing only considers a single lineage, either the Y chromosome inherited through men in a father’s line or mitochondrial DNA, which a passed down only from mothers. This DNA can reveal genetic information about only one or two ancestors, even though, for example, just three generations back people also have six other great-grandparents or, four generations back, 14 other great-great-grandparents.

\qquad Critics also argue that commercial genetic testing is only as good as the reference collections to which a sample is compared. Databases used by some companies don’t rely on data collected systematically but rather lump together information from different research projects. This means that a DNA database may differ depending on the company that processes the results. In addition, the computer programs a company uses to estimate relationships may be patented and not subject to peer review or outside evaluation.

26.In paragraphs 1 and 2 , the text shows PTK’s .\par
	\choice{ easy availability}{ flexibility in pricing}{ successful promotion}{ popularity with households}

27. PTK is used to .\par
	\choice{locate one’s birth place}{ promote genetic research}{ identify parent-child kinship}{ choose children for adoption}

28. Skeptical observers believe that ancestry testing fails to.\par
	\choice{trace distant ancestors                                               }{  rebuild reliable bloodlines}{ fully use genetic information                                            }{ achieve the claimed accuracy}

29. In the last paragraph ,a problem commercial genetic testing faces is .\par
	\choice{disorganized data collection  }{  overlapping database building}{C}{D}

30. An appropriate title for the text is most likely to be.\par
	\choice{Fors and Againsts of DNA testing                               }{  DNA testing and It’s problems}{DNA testing outside the lab                    }{ lies behind DNA testing }