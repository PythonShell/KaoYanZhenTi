%% Content from http://zhenti.kaoyan.eol.cn/
%% Format by PythonShell
%% 2014-01-08

\begin{center}\textbf{Text 3}\end{center}

\qquad In the early 1960s Wilt Chamberlain was one of only three players in the National Basketball Association (NBA) listed at over seven feet. If he had played last season, however, he would have been one of 42.  The bodies playing major professional sports have changed dramatically over the years, and managers have been more than willing to adjust team uniforms to fit the growing numbers of bigger, longer frames.

\qquad The trend in sports, though, may be obscuring an unrecognized reality: Americans have generally stopped growing. Though typically about two inches taller now than 140 years ago, today's people - especially those born to families who have lived in the U.S. for many generations - apparently reached their limit in the early 1960s. And they aren't likely to get any taller. ``In the general population today, at this genetic, environmental level, we've pretty much gone as far as we can go,'' says anthropologist William Cameron Chumlea of Wright State University. In the case of NBA players, their increase in height appears to result from the increasingly common practice of recruiting players from all over the world.

\qquad Growth, which rarely continues beyond the age of 20, demands calories and nutrients - notably, protein - to feed expanding tissues. At the start of the 20th century, under-nutrition and childhood infections got in the way. But as diet and health improved, children and adolescents have, on average, increased in height by about an inch and a half every 20 years, a pattern known as the secular trend in height. Yet according to the Centers for Disease Control and Prevention, average height - 5'9'' for men, 5'4'' for women - hasn't really changed since 1960. 

\qquad Genetically speaking, there are advantages to avoiding substantial height. During childbirth, larger babies have more difficulty passing through the birth canal. Moreover, even though humans have been upright for millions of years, our feet and back continue to struggle with bipedal posture and cannot easily withstand repeated strain imposed by oversize limbs. ``There are some real constraints that are set by the genetic architecture of the individual organism,'' says anthropologist William Leonard of Northwestern University.

\qquad Genetic maximums can change, but don't expect this to happen soon. Claire C. Gordon, senior anthropologist at the Army Research Center in Natick, Mass., ensures that 90 percent of the uniforms and workstations fit recruits without alteration. She says that, unlike those for basketball, the length of military uniforms has not changed for some time. And if you need to predict human height in the near future to design a piece of equipment, Gordon says that by and large, ``you could use today's data and feel fairly confident.''

\vspace{6pt}

31. Wilt Chamberlain is cited as an example to\par
	\choice{ illustrate the change of height of NBA players.}{ show the popularity of NBA players in the U.S..}{ compare different generations of NBA players.}{ assess the achievements of famous NBA players.}

32. Which of the following plays a key role in body growth according to the text?\par
	\choice{ Genetic modification.}{ Natural environment.}{ Living standards.}{ Daily exercise.}

33. On which of the following statements would the author most probably agree?\par
	\choice{ Non-Americans add to the average height of the nation.}{ Human height is conditioned by the upright posture.}{ Americans are the tallest on average in the world.}{ Larger babies tend to become taller in adulthood.}

34. We learn from the last paragraph that in the near future\par
	\choice{ the garment industry will reconsider the uniform size.}{ the design of military uniforms will remain unchanged.}{ genetic testing will be employed in selecting sportsmen.}{ the existing data of human height will still be applicable.}

35. The text intends to tell us that\par
	\choice{ the change of human height follows a cyclic pattern.}{ human height is becoming even more predictable.}{ Americans have reached their genetic growth limit.}{ the genetic pattern of Americans has altered.}