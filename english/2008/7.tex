%% Content from http://zhenti.kaoyan.eol.cn/
%% Format by PythonShell
%% 2014-01-08

\qquad In his autobiography, Darwin himself speaks of his intellectual powers with extraordinary modesty. He points out that he always experienced much difficulty in expressing himself clearly and concisely, but \ul{(46) he believes that this very difficulty may have had the compensating advantage of forcing him to think long and intently about every sentence, and thus enabling him to detect errors in reasoning and in his own observations.} He disclaimed the possession of any great quickness of apprehension or wit, such as distinguished Huxley. \ul{(47) He asserted, also, that his power to follow a long and purely abstract train of thought was very limited, for which reason he felt certain that he never could have succeeded with mathematics.} His memory, too, he described as extensive, but hazy. So poor in one sense was it that he never could remember for more than a few days a single date or a line of poetry. \ul{(48) On the other hand, he did not accept as well founded the charge made by some of his critics that, while he was a good observer, he had no power of reasoning.} This, he thought, could not be true, because the ``Origin of Species'' is one long argument from the beginning to the end, and has convinced many able men. No one, he submits, could have written it without possessing some power of reasoning. He was willing to assert that ``I have a fair share of invention, and of common sense or judgment, such as every fairly successful lawyer or doctor must have, but not, I believe, in any higher degree.'' \ul{(49) He adds humbly that perhaps he was ``superior to the common run of men in noticing things which easily escape attention, and in observing them carefully.''}

\qquad Writing in the last year of his life, he expressed the opinion that in two or three respects his mind had changed during the preceding twenty or thirty years. Up to the age of thirty or beyond it poetry of many kinds gave him great pleasure. Formerly, too, pictures had given him considerable, and music very great, delight. In 1881, however, he said: ``Now for many years I cannot endure to read a line of poetry. I have also almost lost my taste for pictures or music.'' \ul{(50) Darwin was convinced that the loss of these tastes was not only a loss of happiness, but might possibly be injurious to the intellect, and more probably to the moral character.}