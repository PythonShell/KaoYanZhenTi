%% Content from http://zhenti.kaoyan.eol.cn/
%% Format by PythonShell
%% 2014-01-08

\begin{center}\textbf{Text 1}\end{center}

\qquad While still catching up to men in some spheres of modern life, women appear to be way ahead in at least one undesirable category. ``Women are particularly susceptible to developing depression and anxiety disorders in response to stress compared to men,'' according to Dr. Yehuda, chief psychiatrist at New York's Veteran's Administration Hospital.

\qquad Studies of both animals and humans have shown that sex hormones somehow affect the stress response, causing females under stress to produce more of the trigger chemicals than do males under the same conditions. In several of the studies, when stressed-out female rats had their ovaries (the female reproductive organs) removed, their chemical responses became equal to those of the males.

\qquad Adding to a woman's increased dose of stress chemicals, are her increased ``opportunities'' for stress. ``It's not necessarily that women don't cope as well. It's just that they have so much more to cope with,'' says Dr. Yehuda. ``Their capacity for tolerating stress may even be greater than men's,'' she observes, ``it's just that they're dealing with so many more things that they become worn out from it more visibly and sooner.''

\qquad Dr. Yehuda notes another difference between the sexes. ``I think that the kinds of things that women are exposed to tend to be in more of a chronic or repeated nature. Men go to war and are exposed to combat stress. Men are exposed to more acts of random physical violence. The kinds of interpersonal violence that women are exposed to tend to be in domestic situations, by, unfortunately, parents or other family members, and they tend not to be one-shot deals. The wear-and-tear that comes from these longer relationships can be quite devastating.''

\qquad Adeline Alvarez married at 18 and gave birth to a son, but was determined to finish college. ``I struggled a lot to get the college degree. I was living in so much frustration that that was my escape, to go to school, and get ahead and do better.'' Later, her marriage ended and she became a single mother. ``It's the hardest thing to take care of a teenager, have a job, pay the rent, pay the car payment, and pay the debt. \ul{I lived from paycheck to paycheck.}''

\qquad Not everyone experiences the kinds of severe chronic stresses Alvarez describes. But most women today are coping with a lot of obligations, with few breaks, and feeling the strain. Alvarez's experience demonstrates the importance of finding ways to diffuse stress before it threatens your health and your ability to function.

\vspace{6pt}

21. Which of the following is true according to the first two paragraphs?\par
	\choice{ Women are biologically more vulnerable to stress.}{ Women are still suffering much stress caused by men.}{ Women are more experienced than men in coping with stress.}{ Men and women show different inclinations when faced with stress.}

22. Dr. Yehuda's research suggests that women\par
	\choice{ need extra doses of chemicals to handle stress.}{ have limited capacity for tolerating stress.}{ are more capable of avoiding stress.}{ are exposed to more stress.}

23. According to Paragraph 4, the stress women confront tends to be\par
	\choice{ domestic and temporary.}{ irregular and violent.}{ durable and frequent.}{ trivial and random.}

24. The sentence ``I lived from paycheck to paycheck.'' (Line 5, Para. 5) shows that\par
	\choice{ Alvarez cared about nothing but making money.}{ Alvarez's salary barely covered her household expenses.}{ Alvarez got paychecks from different jobs.}{ Alvarez paid practically everything by check.}

25. Which of the following would be the best title for the text?\par
	\choice{ Strain of Stress: No Way Out?}{ Responses to Stress: Gender Difference}{ Stress Analysis: What Chemicals Say}{ Gender Inequality: Women Under Stress}