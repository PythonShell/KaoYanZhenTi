\begin{center}\textbf{Text 4}\end{center}

\qquad In 1784, five years before he became president of the United States, George Washington, 52, was nearly toothless. So he hired a dentist to transplant nine teeth into his jaw - having extracted them from the mouths of his slaves.

\qquad That’s a far different image from the cherry-tree-chopping George most people remember from their history books. But recently, many historians have begun to focus on the roles slavery played in the lives of the founding generation. They have been spurred in part by DNA evidence made available in 1998, which almost certainly proved Thomas Jefferson had fathered at least one child with his slave Sally Hemings. And only over the past 30 years have scholars examined history from the bottom up. Works of several historians reveal the moral compromises made by the nation’s early leaders and the fragile nature of the country’s infancy. More significantly, they argue that many of the Founding Fathers knew slavery was wrong - and yet most did little to fight it.

\qquad More than anything, the historians say, the founders were hampered by the culture of their time. While Washington and Jefferson privately expressed distaste for slavery, they also understood that it was part of the political and economic bedrock of the country they helped to create.

\qquad For one thing, the South could not afford to part with its slaves. Owning slaves was “like having a large bank account,” says Wiencek, author of An Imperfect God: George Washington, His Slaves, and the Creation of America. The southern states would not have signed the Constitution without protections for the “peculiar institution,” including a clause that counted a slave as three fifths of a man for purposes of congressional representation.

\qquad And the statesmen’s political lives depended on slavery. The three-fifths formula handed Jefferson his narrow victory in the presidential election of 1800 by inflating the votes of the southern states in the Electoral College. Once in office, Jefferson extended slavery with the Louisiana Purchase in 1803; the new land was carved into 13 states, including three slave states.

\qquad Still, Jefferson freed Hemings’s children - though not Hemings herself or his approximately 150 other slaves. Washington, who had begun to believe that all men were created equal after observing the bravery of the black soldiers during the Revolutionary War, overcame the strong opposition of his relatives to grant his slaves their freedom in his will. Only a decade earlier, such an act would have required legislative approval in Virginia.

36.George Washington’s dental surgery is mentioned to\par
	\choice{ show the primitive medical practice in the past.}{ demonstrate the cruelty of slavery in his days.}{ stress the role of slaves in the U.S. history.}{ reveal some unknown aspect of his life.}

37.We may infer from the second paragraph that\par
	\choice{ DNA technology has been widely applied to history research.}{ in its early days the U.S. was confronted with delicate situations.}{ historians deliberately made up some stories of Jefferson’s life.}{ political compromises are easily found throughout the U.S. history.}

38.What do we learn about Thomas Jefferson?\par
	\choice{ His political view changed his attitude towards slavery.}{ His status as a father made him free the child slaves.}{ His attitude towards slavery was complex.}{ His affair with a slave stained his prestige.}

39.Which of the following is true according to the text?\par
	\choice{ Some Founding Fathers benefit politically from slavery.}{ Slaves in the old days did not have the right to vote.}{ Slave owners usually had large savings accounts.}{ Slavery was regarded as a peculiar institution.}

40.Washington’s decision to free slaves originated from his\par
	\choice{ moral considerations.}{ military experience.}{ financial conditions.}{ political stand.}