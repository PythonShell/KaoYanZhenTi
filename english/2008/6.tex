%% Content from http://zhenti.kaoyan.eol.cn/
%% Format by PythonShell
%% 2014-01-08

\qquad The time for sharpening pencils, arranging your desk, and doing almost anything else instead of writing has ended. The first draft will appear on the page only if you stop avoiding the inevitable and sit, stand up, or lie down to write. \ul{(41) \qquad\qquad}.

\qquad Be flexible. Your outline should smoothly conduct you from one point to the next, but do not permit it to railroad you. If a relevant and important idea occurs to you now, work it into the draft. \ul{(42) \qquad\qquad}. Grammar, punctuation, and spelling can wait until you revise. Concentrate on what you are saying. Good writing most often occurs when you are in hot pursuit of an idea rather than in a nervous search for errors.

\qquad \ul{(43) \qquad\qquad}. Your pages will be easier to keep track of that way, and, if you have to clip a paragraph to place it elsewhere, you will not lose any writing on the other side.

\qquad If you are working on a word processor, you can take advantage of its capacity to make additions and deletions as well as move entire paragraphs by making just a few simple keyboard commands. Some software programs can also check spelling and certain grammatical elements in your writing. \ul{(44) \qquad\qquad}. These printouts are also easier to read than the screen when you work on revisions.

\qquad Once you have a first draft on paper, you can delete material that is unrelated to your thesis and add material necessary to illustrate your points and make your paper convincing. The student who wrote ``The A \& P as a State of Mind'' wisely dropped a paragraph that questioned whether Sammy displays chauvinistic attitudes toward women. \ul{(45) \qquad\qquad}.

\qquad Remember that your initial draft is only that. You should go through the paper many times - and then again - working to substantiate and clarify your ideas. You may even end up with several entire versions of the paper. Rewrite. The sentences within each paragraph should be related to a single topic. Transitions should connect one paragraph to the next so that there are no abrupt or confusing shifts. Awkward or wordy phrasing or unclear sentences and paragraphs should be mercilessly poked and prodded into shape.

\vspace{6pt}

\qquad [A] To make revising easier, leave wide margins and extra space between lines so that you can easily add words, sentences, and corrections. Write on only one side of the paper.

\qquad [B] After you have clearly and adequately developed the body of your paper, pay particular attention to the introductory and concluding paragraphs. It's probably best to write the introduction last, after you know precisely what you are introducing. Concluding paragraphs demand equal attention because they leave the reader with a final impression.

\qquad [C] It's worth remembering, however, that though a clean copy fresh off a printer may look terrible, it will read only as well as the thinking and writing that have gone into it. Many writers prudently store their data on disks and print their pages each time they finish a draft to avoid losing any material because of power failures or other problems.

\qquad [D] It makes no difference how you write, just so you do. Now that you have developed a topic into a tentative thesis, you can assemble your notes and begin to flesh out whatever outline you have made.

\qquad [E] Although this is an interesting issue, it has nothing to do with the thesis, which explains how the setting influences Sammy's decision to quit his job. Instead of including that paragraph, she added one that described Lengel's crabbed response to the girls so that she could lead up to the A \& P ``policy'' he enforces.

\qquad [F] In the final paragraph about the significance of the setting in ``A \& P,'' the student brings together the reasons Sammy quit his job by referring to his refusal to accept Lengel's store policies.

\qquad [G] By using the first draft as a means of thinking about what you want to say, you will very likely discover more than your notes originally suggested. Plenty of good writers don't use outlines at all but discover ordering principles as they write. Do not attempt to compose a perfectly correct draft the first time around.