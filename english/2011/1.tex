%% Content from http://zhenti.kaoyan.eol.cn/
%% Format by PythonShell
%% 2014-01-08

\qquad Ancient Greek philosopher Aristotle viewed laughter as ``a bodily exercise precious to health.'' But \underline{\quad 1\quad} some claims to the contrary, laughing probably has little influence on physical fitness. Laughter does \underline{\quad 2\quad} short-term changes in the function of the heart and its blood vessels, \underline{\quad 3\quad} heart rate and oxygen consumption. But because hard laughter is difficult to \underline{\quad 4\quad}, a good laugh is unlikely to have \underline{\quad 5\quad} benefits the way, say, walking or jogging does.

\qquad \underline{\quad 6\quad}, instead of straining muscles to build them, as exercise does, laughter apparently accomplishes the \underline{\quad 7\quad}. Studies dating back to the 1930's indicate that laughter \underline{\quad 8\quad} muscles, decreasing muscle tone for up to 45 minutes after the laugh dies down.

\qquad Such bodily reaction might conceivably help \underline{\quad 9\quad} the effects of psychological stress. Anyway, the act of laughing probably does produce other types of \underline{\quad 10\quad} feedback, that improve an individual's emotional state. \underline{\quad 11\quad} one classical theory of emotion, our feelings are partially rooted \underline{\quad 12\quad} physical reactions. It was argued at the end of the 19th century that humans do not cry \underline{\quad 13\quad} they are sad but they become sad when the tears begin to flow.

\qquad Although sadness also \underline{\quad 14\quad} tears, evidence suggests that emotions can flow \underline{\quad 15\quad} muscular responses. In an experiment published in 1988, social psychologist Fritz Strack of the University of W\"{u}rzburg in Germany asked volunteers to \underline{\quad 16\quad} a pen either with their teeth-thereby creating an artificial smile--or with their lips, which would produce a(n) \underline{\quad 17\quad} expression. Those forced to exercise their smiling muscles \underline{\quad 18\quad} more enthusiastically to funny catoons than did those whose months were contracted in a frown, \underline{\quad 19\quad} that expressions may influence emotions rather than just the other way around \underline{\quad 20\quad}, the physical act of laughter could improve mood.

\vspace{6pt}

01. \choice{ among  }{ except  }{ despite  }{ like}
02.   \choice{ reflect       }{ demand        }{ indicate      }{ produce}
03.   \choice{ stabilizing   }{ boosting      }{ impairing     }{ determining}
04.   \choice{ transmit      }{ sustain       }{ evaluate      }{ observe}
05.   \choice{ measurable    }{ manageable    }{ affordable    }{ renewable }
06.   \choice{ In turn       }{ In fact       }{ In addition   }{ In brief}
07.   \choice{ opposite      }{ impossible    }{ average       }{ expected}
08.   \choice{ hardens       }{ weakens       }{ tightens      }{ relaxes}
09.   \choice{ aggravate     }{ generate      }{ moderate      }{ enhance}
10.  \choice{ physical      }{ mental        }{ subconscious  }{ internal}
11.  \choice{ Except for    }{ According to  }{ Due to        }{ As for}
12.  \choice{ with          }{ on            }{ in            }{ at}
13.  \choice{ unless        }{ until         }{ if            }{ because}
14.  \choice{ exhausts      }{ follows       }{ precedes      }{ suppresses}
15.  \choice{ into          }{ from          }{ towards       }{ beyond}
16.  \choice{ fetch         }{ bite          }{ pick          }{ hold}
17.  \choice{ disappointed  }{ excited       }{ joyful        }{ indifferent}
18.  \choice{ adapted       }{ catered       }{ turned        }{ reacted}
19.  \choice{ suggesting    }{ requiring     }{ mentioning    }{ supposing}
20.  \choice{ Eventually    }{ Consequently  }{ Similarly     }{ Conversely}