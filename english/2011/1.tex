Ancient Greek philosopher Aristotle viewed laughter as ``a bodily exercise precious to health.'' But \underline{\quad 1\quad} some claims to the contrary, laughing probably has little influence on physical fitness Laughter does \underline{\quad 2\quad} short-term changes in the function of the heart and its blood vessels, \underline{\quad 3\quad} heart rate and oxygen consumption But because hard laughter is difficult to \underline{\quad 4\quad}, a good laugh is unlikely to have \underline{\quad 5\quad} benefits the way, say, walking or jogging does.

\underline{\quad 6\quad}, instead of straining muscles to build them, as exercise does, laughter apparently accomplishes the \underline{\quad 7\quad}, studies dating back to the 1930’s indicate that laughter \underline{\quad 8\quad} muscles, decreasing muscle tone for up to 45 minutes after the laugh dies down.

Such bodily reaction might conceivably help \underline{\quad 9\quad} the effects of psychological stress. Anyway, the act of laughing probably does produce other types of \underline{\quad 10\quad} feedback, that improve an individual’s emotional state. \underline{\quad 11\quad} one classical theory of emotion, our feelings are partially rooted \underline{\quad 12\quad} physical reactions. It was argued at the end of the 19th century that humans do not cry \underline{\quad 13\quad} they are sad but they become sad when the tears begin to flow.

Although sadness also \underline{\quad 14\quad} tears, evidence suggests that emotions can flow \underline{\quad 15\quad} muscular responses. In an experiment published in 1988, social psychologist Fritz Strack of the University of würzburg in Germany asked volunteers to \underline{\quad 16\quad} a pen either with their teeth-thereby creating an artificial smile--or with their lips, which would produce a(n) \underline{\quad 17\quad} expression. Those forced to exercise their enthusiastically to funny catoons than did those whose months were contracted in a frown, \underline{\quad 19\quad} that expressions may influence emotions rather than just the other way around \underline{\quad 20\quad}, the physical act of laughter could improve mood.

\begin{tabbing}
\hspace{0cm}
\=1.  \quad\= [A] among \quad\quad\quad\quad\= [B] except \quad\quad\quad\quad\= [C] despite \quad\quad\quad\quad\= [D] like\\
\>2.  \> [A] reflect      \> [B] demand       \> [C] indicate     \> [D] produce\\
\>3.  \> [A] stabilizing  \> [B] boosting     \> [C] impairing    \> [D] determining\\
\>4.  \> [A] transmit     \> [B] sustain      \> [C] evaluate     \> [D] observe\\
\>5.  \> [A] measurable   \> [B] manageable   \> [C] affordable   \> [D] renewable\\ 
\>6.  \> [A] In turn      \> [B] In fact      \> [C] In addition  \> [D] In brief\\
\>7.  \> [A] opposite     \> [B] impossible   \> [C] average      \> [D] expected\\
\>8.  \> [A] hardens      \> [B] weakens      \> [C] tightens     \> [D] relaxes\\
\>9.  \> [A] aggravate    \> [B] generate     \> [C] moderate     \> [D] enhance\\
\>10. \> [A] physical     \> [B] mental       \> [C] subconscious \> [D] internal\\
\>11. \> [A] Except for   \> [B] According to \> [C] Due to       \> [D] As for\\
\>12. \> [A] with         \> [B] on           \> [C] in           \> [D] at\\
\>13. \> [A] unless       \> [B] until        \> [C] if           \> [D] because\\
\>14. \> [A] exhausts     \> [B] follows      \> [C] precedes     \> [D] suppresses\\
\>15. \> [A] into         \> [B] from         \> [C] towards      \> [D] beyond\\
\>16. \> [A] fetch        \> [B] bite         \> [C] pick         \> [D] hold\\
\>17. \> [A] disappointed \> [B] excited      \> [C] joyful       \> [D] indifferent\\
\>18. \> [A] adapted      \> [B] catered      \> [C] turned       \> [D] reacted\\
\>19. \> [A] suggesting   \> [B] requiring    \> [C] mentioning   \> [D] supposing\\
\>20. \> [A] Eventually   \> [B] Consequently \> [C] Similarly    \> [D] Conversely
\end{tabbing}