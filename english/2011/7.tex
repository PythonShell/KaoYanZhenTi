%% Content from http://zhenti.kaoyan.eol.cn/
%% Format by PythonShell
%% 2014-01-08

\qquad With its theme that ``Mind is the master weaver,'' creating our inner character and outer circumstances, the book \emph{As a Man Thinking} by James Allen is an in-depth exploration of the central idea of self-help writing.

\qquad \ul{(46) Allen's contribution was to take an assumption we all share-that because we are not robots we therefore control our thoughts-and reveal its erroneous nature.} Because most of us believe that mind is separate from matter, we think that thoughts can be hidden and made powerless; this allows us to think one way and act another. However, Allen believed that the unconscious mind generates as much action as the conscious mind, and \ul{(47) while we may be able to sustain the illusion of control through the conscious mind alone, in reality we are continually faced with a question: ``Why cannot I make myself do this or achieve that?''}

\qquad Since desire and will are damaged by the presence of thoughts that do not accord with desire, Allen concluded : ``We do not attract what we want, but what we are.'' Achievement happens because you as a person embody the external achievement; you don't ``get'' success but become it. There is no gap between mind and matter.

\qquad Part of the fame of Allen's book is its contention that ``Circumstances do not make a person, they reveal him.'' \ul{(48) This seems a justification for neglect of those in need, and a rationalization of exploitation, of the superiority of those at the top and the inferiority of those at the bottom.}

\qquad This, however, would be a knee-jerk reaction to a subtle argument. Each set of circumstances, however bad, offers a unique opportunity for growth. If circumstances always determined the life and prospects of people, then humanity would never have progressed. In fact, \ul{(49) circumstances seem to be designed to bring out the best in us and if we feel that we have been ``wronged'' then we are unlikely to begin a conscious effort to escape from our situation.} Nevertheless, as any biographer knows, a person's early life and its conditions are often the greatest gift to an individual.

\qquad The sobering aspect of Allen's book is that we have no one else to blame for our present condition except ourselves. \ul{(50) The upside is the possibilities contained in knowing that everything is up to us; where before we were experts in the array of limitations, now we become authorities of what is possible.}
