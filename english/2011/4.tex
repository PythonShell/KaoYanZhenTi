%% Content from http://zhenti.kaoyan.eol.cn/
%% Format by PythonShell
%% 2014-01-08

\begin{center}\textbf{Text 3}\end{center}

\qquad The rough guide to marketing success used to be that you got what you paid for. No longer. While traditional ``paid'' media--such as television commercials and print advertisements--still play a major role, companies today can exploit many alternative forms of media. Consumers passionate about a product may create ``earned'' media by willing to promoting it to friends, and a company may leverage ``owned'' media by sending e-mail alerts about products and sales to customers registered with its Web site. The way consumers now approach the process of making purchase decisions means that marketing's impact stems from a broad range of factors beyond conventional paid media.

\qquad Paid and owned media are controlled by marketers promoting their own products. For earned media, such marketers act as the initiator for users' responses. But in some cases, one marketer's owned media become another marketer's paid media--for instance, when an e-commerce retailer sells ad space on its Web site. We define such sold media as owned media whose traffic is so strong that other organizations place their content or e-commerce engines within that environment. This trend, which we believe is still in its infancy, effectively began with retailers and travel providers such as airlines and hotels and will no doubt go further. Johnson \& Johnson, for example, has created BabyCenter, a stand-alone media property that promotes complementary and even competitive products. Besides generating income, the presence of other marketers makes the site seem objective, gives companies opportunities to learn valuable information about the appeal of other companies' marketing, and may help expand user traffic for all companies concerned. 

\qquad The same dramatic technological changes that have provided marketers with more (and more diverse) communications choices have also increased the risk that passionate consumers will voice their opinions in quicker, more visible, and much more damaging ways. Such hijacked media are the opposite of earned media: an asset or campaign becomes hostage to consumers, other stakeholders, or activists who make negative allegations about a brand or product. Members of social networks, for instance, are learning that they can hijack media to apply pressure on the businesses that originally created them.

\qquad If that happens, passionate consumers would try to persuade others to boycott products, putting the reputation of the target company at risk. In such a case, the company's response may not be sufficiently quick or thoughtful, and the learning curve has been steep. Toyota Motor, for example, alleviated some of the damage from its recall crisis earlier this year with a relatively quick and well-orchestrated social-media response campaign, which included efforts to engage with consumers directly on sites such as Twitter and the social-news site Digg.

\vspace{6pt}

31.Consumers may create ``earned'' media when they are\par
	\choice{ obscssed with online shopping at certain Web sites.}{ inspired by product-promoting e-mails sent to them.}{ eager to help their friends promote quality products.}{ enthusiastic about recommending their favorite products. }

32. According to Paragraph 2, sold media feature\par
	\choice{ a safe business environment.}{ random competition.}{ strong user traffic.}{ flexibility in organization.}

33. The author indicates in Paragraph 3 that earned media\par
	\choice{ invite constant conflicts with passionate consumers.}{ can be used to produce negative effects in marketing.}{ may be responsible for fiercer competition.}{ deserve all the negative comments about them. }

34. Toyota Motor's experience is cited as an example of\par
	\choice{ responding effectively to hijacked media.}{ persuading customers into boycotting products.}{ cooperating with supportive consumers.}{ taking advantage of hijacked media.}

35. Which of the following is the text mainly about?\par
	\choice{ Alternatives to conventional paid media.}{ Conflict between hijacked and earned media.}{ Dominance of hijacked media.}{ Popularity of owned media.}
