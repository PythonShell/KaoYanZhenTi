46. 爱伦的贡献在于,他拿出“我们并不是机器人,所以能控制自己思想”这一公认的假设,并揭示了其谬误所在。

47. 尽管我们或许可以仅凭借意识来维系“控制”这种错觉,现实中我们还是不断要面对一个问题:“我为什么不能让自己做这个或完成那个?”

48. 这似乎是在为忽视贫困者的行为做辩护,为剥削、为社会上层人群的优越及社会底层人群的卑微找理由。

49. 环境仿佛是为了激发我们的最大潜能而设,如果我们觉得自己遭受了“不公”,就不太可能有意识地去努力摆脱自己的处境。

50. 其正面意义在于,了解了一切都取决于我们自己,即有了诸多可能,此前我们是谙熟各种局限的专家,现在我们成了驾驭各种可能性的权威。