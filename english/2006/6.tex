\qquad On the north bank of the Ohio river sits Evansville, Ind., home of David Williams, 52, and of a riverboat casino (a place where gambling games are played). During several years of gambling in that casino, Williams, a state auditor earning \$35,000 a year, lost approximately \$175,000. He had never gambled before the casino sent him a coupon for \$20 worth of gambling.

\qquad He visited the casino, lost the \$20 and left. On his second visit he lost \$800. The casino issued to him, as a good customer, a ``Fun Card'', which when used in the casino earns points for meals and drinks, and enables the casino to track the user’s gambling activities. For Williams, those activities become what he calls ``electronic heroin''.

\qquad \ul{(41) \qquad\qquad}. In 1997 he lost \$21,000 to one slot machine in two days. In March 1997 he lost \$72,186. He sometimes played two slot machines at a time, all night, until the boat docked at 5 a.m., then went back aboard when the casino opened at 9 a.m. Now he is suing the casino, charging that it should have refused his patronage because it knew he was addicted. It did know he had a problem.

\qquad In March 1998 a friend of Williams’s got him involuntarily confined to a treatment center for addictions, and wrote to inform the casino of Williams’s gambling problem. The casino included a photo of Williams among those of banned gamblers, and wrote to him a “cease admissions” letter. Noting the medical/psychological nature of problem gambling behavior, the letter said that before being readmitted to the casino he would have to present medical/psychological information demonstrating that patronizing the casino would pose no threat to his safety or well-being.

\qquad \ul{(42) \qquad\qquad}.

\qquad The Wall Street Journal reports that the casino has 24 signs warning: “Enjoy the fun... and always bet with your head, not over it.” Every entrance ticket lists a toll-free number for counseling from the Indiana Department of Mental Health. Nevertheless, Williams’s suit charges that the casino, knowing he was “helplessly addicted to gambling,” intentionally worked to “lure” him to “engage in conduct against his will.” Well.

\qquad \ul{(43) \qquad\qquad}.

\qquad The fourth edition of the Diagnostic and Statistical Manual of Mental Disorders says “pathological gambling” involves persistent, recurring and uncontrollable pursuit less of money than of thrill of taking risks in quest of a windfall.

\qquad \ul{(44) \qquad\qquad}. Pushed by science, or what claims to be science, society is reclassifying what once were considered character flaws or moral failings as personality disorders akin to physical disabilities.

\qquad \ul{(45) \qquad\qquad}.

\qquad Forty-four states have lotteries, 29 have casinos, and most of these states are to varying degrees dependent on -- you might say addicted to -- revenues from wagering. And since the first Internet gambling site was created in 1995, competition for gamblers’ dollars has become intense. The Oct. 28 issue of Newsweek reported that 2 million gamblers patronize 1,800 virtual casinos every week. With \$3.5 billion being lost on Internet wagers this year, gambling has passed pornography as the Web’s most profitable business.

\qquad [A] Although no such evidence was presented, the casino’s marketing department continued to pepper him with mailings. And he entered the casino and used his Fun Card without being detected.

\qquad [B] It is unclear what luring was required, given his compulsive behavior. And in what sense was his will operative?

\qquad [C] By the time he had lost \$5,000 he said to himself that if he could get back to even, he would quit. One night he won \$5,500, but he did not quit.

\qquad [D] Gambling has been a common feature of American life forever, but for a long time it was broadly considered a sin, or a social disease. Now it is a social policy: the most important and aggressive promoter of gambling in America is the government.

\qquad [E] David Williams’s suit should trouble this gambling nation. But don’t bet on it.

\qquad [F] It is worrisome that society is medicalizing more and more behavioral problems, often defining as addictions what earlier, sterner generations explained as weakness of will.

\qquad [G] The anonymous, lonely, undistracted nature of online gambling is especially conducive to compulsive behavior. But even if the government knew how to move against Internet gambling, what would be its grounds for doing so?