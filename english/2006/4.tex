%% Content from http://zhenti.kaoyan.eol.cn/
%% Format by PythonShell
%% 2014-01-07

\begin{center}\textbf{Text 3}\end{center}

\qquad When prehistoric man arrived in new parts of the world, something strange happened to the large animals: They suddenly became extinct. Smaller species survived. The large, slow-growing animals were easy game, and were quickly hunted to extinction. Now something similar could be happening in the oceans.

\qquad That the seas are being overfished has been known for years. What researchers such as Ransom Myers and Boris Worm have shown is just how fast things are changing. They have looked at half a century of data from fisheries around the world. Their methods do not attempt to estimate the actual biomass (the amount of living biological matter) of fish species in particular parts of the ocean, but rather changes in that biomass over time. According to their latest paper published in \emph{Nature}, the biomass of large predators (animals that kill and eat other animals) in a new fishery is reduced on average by 80\% within 15 years of the start of exploitation. In some long-fished areas, it has halved again since then.

\qquad Dr. Worm acknowledges that \ul{these figures are conservative}. One reason for this is that fishing technology has improved. Today's vessels can find their prey using satellites and sonar, which were not available 50 years ago. That means a higher proportion of what is in the sea is being caught, so the real difference between present and past is likely to be worse than the one recorded by changes in catch sizes. In the early days, too, longlines would have been more saturated with fish. Some individuals would therefore not have been caught, since no baited hooks would have been available to trap them, leading to an underestimate of fish stocks in the past. Furthermore, in the early days of longline fishing, a lot of fish were lost to sharks after they had been hooked. That is no longer a problem, because there are fewer sharks around now.

\qquad Dr. Myers and Dr. Worm argue that their work gives a correct baseline, which future management efforts must take into account. They believe the data support an idea current among marine biologists, that of the ``shifting baseline''. The notion is that people have failed to detect the massive changes which have happened in the ocean because they have been looking back only a relatively short time into the past. That matters because theory suggests that the maximum sustainable yield that can be cropped from a fishery comes when the biomass of a target species is about 50\% of its original levels. Most fisheries are well below that, which is a bad way to do business.

\vspace{6pt}

31. The extinction of large prehistoric animals is noted to suggest that\par
	\choice{ large animal were vulnerable to the changing environment}{ small species survived as large animals disappeared}{ large sea animals may face the same threat today}{ slow-growing fish outlive fast-growing ones}

32. We can infer from Dr. Myers and Dr. Worm's paper that\par
	\choice{ the stock of large predators in some old fisheries has reduced by 90\%}{ there are only half as many fisheries as there were 15 years ago}{ the catch sizes in new fisheries are only 20\% of the original amount}{ the number of larger predators dropped faster in new fisheries than in the old}

33. By saying ``these figures are conservative'' (Line 1, paragraph 3), Dr. Worm means that\par
	\choice{ fishing technology has improved rapidly}{ the catch-sizes are actually smaller than recorded}{ the marine biomass has suffered a greater loss}{ the data collected so far are out of date}

34. Dr. Myers and other researchers hold that\par
	\choice{ people should look for a baseline that can work for a longer time}{ fisheries should keep their yields below 50\% of the biomass}{ the ocean biomass should be restored to its original level}{ people should adjust the fishing baseline to the changing situation}

35. The author seems to be mainly concerned with most fisheries'\par
	\choice{ management efficiency}{ biomass level}{ catch-size limits}{ technological application}