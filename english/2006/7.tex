%% Content from http://zhenti.kaoyan.eol.cn/
%% Format by PythonShell
%% 2014-01-07

\qquad Is it true that the American intellectual is rejected and considered of no account in his society? I am going to suggest that it is not true. Father Bruckberger told part of the story when he observed that it is the intellectuals who have rejected America. But they have done more than that. They have grown dissatisfied with the role of intellectual. It is they, not America, who have become anti-intellectual.

\qquad First, the object of our study pleads for definition. What is an intellectual? \ul{(46) I shall define him as an individual who has elected as his primary duty and pleasure in life the activity of thinking in a Socratic way about moral problems.} He explores such problems consciously, articulately, and frankly, first by asking factual questions, then by asking moral questions, finally by suggesting action which seems appropriate in the light of the factual and moral information which he has obtained. \ul{(47) His function is analogous to that of a judge, who must accept the obligation of revealing in as obvious a manner as possible the course of reasoning which led him to his decision.}

\qquad This definition excludes many individuals usually referred to as intellectuals -- the average scientist, for one. \ul{(48) I have excluded him because, while his accomplishments may contribute to the solution of moral problems, he has not been charged with the task of approaching any but the factual aspects of those problems.} Like other human beings, he encounters moral issues even in the everyday performance of his routine duties -- he is not supposed to cook his experiments, manufacture evidence, or doctor his reports. \ul{(49) But his primary task is not to think about the moral code which governs his activity, any more than a businessman is expected to dedicate his energies to an exploration of rules of conduct in business.} During most of his waking life he will take his code for granted, as the businessman takes his ethics.

\qquad The definition also excludes the majority of teachers, despite the fact that teaching has traditionally been the method whereby many intellectuals earn their living. \ul{(50) They may teach very well and more than earn their salaries, but most of them make little or no independent reflections on human problems which involve moral judgment.} This description even fits the majority of eminent scholars. Being learned in some branch of human knowledge is one thing, living in ``public and illustrious thoughts,'' as Emerson would say, is something else.