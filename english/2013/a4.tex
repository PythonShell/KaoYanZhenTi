46. 然而,只要看看无家可归者创造的花园的照片,你就会意识到尽管样式各异,但这些花园除了表达(人类)装饰和创造的欲望之外,更体现了人类其他根本的欲望。

47. 安宁的圣地(体现的)是人类特有的需要,无论怎样疏于雕琢,它仍与遮风挡雨有所不同,后者(反应的)是动物所特有的需要。

48. 这种无家可归者的花园实质上是无定所的花园,它们把“形式”引入城市环境,而城市环境中原本要么没有这种“形式”,要么并没有把它当成“形式”看待。

49. 我们大多数人通常把陷入精神颓丧归咎于某些心理疾病,直到有一天置身花园,才顿觉压抑感神奇地消失了。

50. 虽然有“扩大词义外延”的意味,但正是这种对大自然或隐晦或明晰的参照让用“花园”一词来描述这些人造组合有了充分的根据。