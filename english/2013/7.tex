It is speculated that gardens arise from a basic need in the individuals who made them: the need for creative expression. There is no doubt that gardens evidence an impossible urge to create, express, fashion, and beautify and that self-expression is a basic human urge; \ul{(46) yet when one looks at the photographs of the garden created by the homeless, it strikes one that , for all their diversity of styles, these gardens speak of various other fundamental urges, beyond that of decoration and creative expression.}

One of these urges had to do with creating a state of peace in the midst of turbulence, a “still point of the turning world,” to borrow a phrase from T. S. Eliot. \ul{(47) A sacred place of peace, however crude it may be, is a distinctly human need, as opposed to shelter, which is a distinctly animal need.} This distinction is so much so that where the latter is lacking, as it is for these unlikely gardens, the former becomes all the more urgent. Composure is a state of mind made possible by the structuring of one’s relation to one’s environment. \ul{(48) The gardens of the homeless which are in effect homeless gardens introduce from into an urban environment where it either didn’t exist or was not discernible as such.} In so doing they give composure to a segment of the inarticulate environment in which they take their stand.

Another urge or need that these gardens appear to respond to, or to arise from is so intrinsic that we are barely ever conscious of its abiding claims on us. When we are deprived of green, of plants, of trees, \ul{(49) most of us give into a demoralization of spirit which we usually blame on some psychological conditions, until one day we find ourselves in garden and feel the expression vanish as if by magic.} In most of the homeless gardens of New York City the actual cultivation of plants is unfeasible, yet even so the compositions often seem to represent attempts to call arrangement of materials, an institution of colors, small pool of water, and a frequent presence of petals or leaves as well as of stuffed animals. On display here are various fantasy elements whose reference, at some basic level, seems to be the natural world. \ul{(50) It is this implicit or explicit reference to nature that fully justifies the use of word garden though in a “liberated” sense, to describe these synthetic constructions.} In them we can see basophilic- a yearning for contact with nonhuman life-assuming uncanny representational forms.

