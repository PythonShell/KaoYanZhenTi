The social sciences are flourishing. As of 2005, there were almost half a million professional social scientists from all fields in the world, working both inside and outside academia. According to the World Social Science Report 2010, the number of social-science students worldwide has swollen by about 11\% every year since 2000.

Yet this enormous resource in not contributing enough to today’s global challenges including climate change, security, sustainable development and health. \ul{41 \quad\quad\quad\quad\quad\quad\quad\quad\quad\quad\quad\quad} Humanity has the necessary agro-technological tools to eradicate hunger, from genetically engineered crops to artificial fertilizers . Here, too, the problems are social: the organization and distribution of food, wealth and prosperity.
\ul{42 \quad\quad\quad\quad\quad\quad\quad\quad\quad\quad}This is a shame—the community should be grasping the opportunity to raise its influence in the real world. To paraphrase the great social scientist Joseph Schumpeter: there is no radical innovation without creative destruction.
Today ,the social sciences are largely focused on disciplinary problems and internal scholarly debates, rather than on topics with external impact.

Analyses reveal that the number of papers including the keywords “environmental changed” or “climate change” have increased rapidly since 2004, \ul{43 \quad\quad\quad\quad\quad\quad\quad\quad\quad\quad\quad\quad}

When social scientists do tackle practical issues ,their scope is often local: Belgium is interested mainly in the effects of poverty on Belgium for example .And whether the community’s work contributes much to an overall accumulation of knowledge is doubtful.

The problem is not necessarily the amount of available funding \ul{44 \quad\quad\quad\quad\quad\quad\quad\quad\quad\quad\quad\quad} this is an adequate amount so long as it is aimed in the right direction. Social scientists who complain about a lack of funding should not expect more in today’s economic climate.

The trick is to direct these funds better. The European Union Framework funding programs have long had a category specifically targeted at social scientists. This year, it was proposed that system be changed: Horizon 2020,a new program to be enacted in 2014,would not have such a category ,This has resulted in protests from social scientists. But the intention is not to neglect social science ; rather ,the complete opposite. \ul{45 \quad\quad\quad\quad\quad\quad\quad\quad\quad\quad\quad\quad} That should create more collaborative endeavors and help to develop projects aimed directly at solving global problems.

\vspace{6pt}

[A] It could be that we are evolving two communities of social
scientists: one that is discipline-oriented and publishing in highly
specialized journals, and one that is problem-oriented and publishing
elsewhere, such as policy briefs.

[B] However, the numbers are still small: in 2010, about 1,600 of the 100,000 social-sciences papers published globally included one of these keywords.

[C] the idea is to force social to integrate their work with other categories, including health and demographic change food security, marine research and the bio-economy, clear, efficient energy; and inclusive, innovative and secure societies.

[D] the solution is to change the mindset of the academic community, and what it considers to be its main goal. Global challenges and social innovation ought to receive much more attention from scientists, especially the young ones.

[E] These issues all have root causes in human behavior. All require behavioral change and social innovations, as well as technological 
development. Stemming climate change, for example, is as much about changing consumption patterns and promoting tax acceptance as it is about developing clean energy.

[F] Despite these factors , many social scientists seem reluctant to tackle such problems . And in Europe , some are up in arms over a proposal to drop a specific funding category for social-science research and to integrate it within cross-cutting topics of sustainable development .

[G] During the late 1990s , national spending on social sciences and the humanities as a percentage of all research and development funds-including government, higher education, non-profit and corporate -varied from around 4\% to 25\%; in most European nations , it is about 15\%.
