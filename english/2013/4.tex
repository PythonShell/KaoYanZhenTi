%% Content from http://zhenti.kaoyan.eol.cn/
%% Format by PythonShell
%% 2014-01-08

\begin{center}\textbf{Text 3}\end{center}

\qquad Up until a few decades ago, our visions of the future were largely --- though by no means uniformly --- glowingly positive. Science and technology would cure all the ills of humanity, leading to lives of fulfillment and opportunity for all.

\qquad Now utopia has grown unfashionable, as we have gained a deeper appreciation of the range of threats facing us, from asteroid strike to epidemic flu and to climate change. You might even be tempted to assume that humanity has little future to look forward to.

\qquad But such gloominess is misplaced. The fossil record shows that many species have endured for millions of years --- so why shouldn't we? Take a broader look at our species' place in the universe, and it becomes clear that we have an excellent chance of surviving for tens, if not hundreds, of thousands of years . Look up Homo sapiens in the ``Red List'' of threatened species of the International Union for the Conversation of Nature (IUCN) ,and you will read: ``Listed as Least Concern as the species is very widely distributed, adaptable, currently increasing, and there are no major threats resulting in an overall population decline.''

\qquad So what does our deep future hold? A growing number of researchers and organizations are now thinking seriously about that question. For example, the Long Now Foundation has its flagship project a medical clock that is designed to still be marking time thousands of years hence .

\qquad Perhaps willfully, it may be easier to think about such lengthy timescales than about the more immediate future. The potential evolution of today's technology, and its social consequences, is dazzlingly complicated, and it's perhaps best left to science fiction writers and futurologists to explore the many possibilities we can envisage. That's one reason why we have launched Arc, a new publication dedicated to the near future.

\qquad But take a longer view and there is a surprising amount that we can say with considerable assurance. As so often, the past holds the key to the future: we have now identified enough of the long-term patterns shaping the history of the planet, and our species, to make evidence-based forecasts about the situations in which our descendants will find themselves.

\qquad This long perspective makes the pessimistic view of our prospects seem more likely to be a passing fad. To be sure, the future is not all rosy. But we are now knowledgeable enough to reduce many of the risks that threatened the existence of earlier humans, and to improve the lot of those to come.

\vspace{6pt}

31. Our vision of the future used to be inspired by\par
	\choice{ our desire for lives of fulfillment}{ our faith in science and technology}{ our awareness of potential risks}{ our belief in equal opportunity}

32. The IUCN's ``Red List'' suggest that human being are\par
	\choice{ a sustained species}{ the world's dominant power}{ a threaten to the environment}{ a misplaced race}

33. Which of the following is true according to Paragraph 5?\par
	\choice{ The interest in science fiction is on the rise.}{ Arc helps limit the scope of futurological studies.}{ Technology offers solutions to social problem.}{ Our Immediate future is hard to conceive.}

34. To ensure the future of mankind, it is crucial to\par
	\choice{ adopt an optimistic view of the world}{ draw on our experience from the past}{ explore our planet's abundant resources}{ curb our ambition to reshape history}

35. Which of the following would be the best title for the text?\par
	\choice{ The Ever-bright Prospects of Mankind}{ Science, Technology and Humanity}{ Evolution of the Human Species}{ Uncertainty about Our Future}
