%% Content from http://zhenti.kaoyan.eol.cn/
%% Format by PythonShell
%% 2014-01-08

\qquad People are, on the whole, poor at considering background
information when making individual decisions. At first glance this
might seem like a strength that \underline{\quad 1\quad} the
ability to make judgments which are unbiased by
\underline{\quad 2\quad} factors. But Dr Simonton speculated that
an inability to consider the big \underline{\quad 3\quad} was
leading decision-makers to be biased by the daily samples of
information they were working with. \underline{\quad 4\quad}, he
theorized that a judge \underline{\quad 5\quad} of appearing too
soft \underline{\quad 6\quad}crime might be more likely to send
someone to prison \underline{\quad 7\quad} he had already sentenced
five or six other defendants only to forced community service on
that day.

\qquad To \underline{\quad 8\quad} this idea, they turned their attention
to the university-admissions process. In theory, the \underline{\quad 9\quad} of an
applicant should not depend on the few others \underline{\quad 10\quad} randomly for interview during the same day, but Dr Simonton suspected the truth was \underline{\quad 11\quad}.

\qquad He studied the results of 9,323 MBA interviews \underline{\quad 12\quad} by 31 admissions officers. The interviewers had \underline{\quad 13\quad} applicants on a scale of one to five. This scale \underline{\quad 14\quad} numerous factors into consideration. The scores were \underline{\quad 15\quad} used in conjunction with an applicant's score on the Graduate Management Admission Test, or GMAT, a standardized exam which is \underline{\quad 16\quad} out of 800 points, to make a decision on whether to accept him or her.

\qquad Dr Simonton found if the score of the previous candidate in a daily series of interviewees was 0.75 points or more higher than that of the one \underline{\quad 17\quad} that, then the score for the next applicant would \underline{\quad 18\quad} by an average of 0.075 points. This might sound small, but to \underline{\quad 19\quad} the effects of such a decrease a candidate would need 30 more GMAT points than would otherwise have been \underline{\quad 20\quad}.

\vspace{6pt}

01. \choice{ grants  }{ submits  }{ transmits  }{ delivers}
02. \choice{ minor      }{  objective   }{ crucial       }{ external}
03. \choice{ issue      }{ vision      }{ picture       }{ moment}
04. \choice{ For example  }{ On average  }{ In principle  }{ Above all}
05. \choice{ fond       }{ fearful     }{ capable       }{ thoughtless}
06. \choice{ in         }{ on         }{ to            }{ for}
07. \choice{ if         }{ until       }{ though        }{ unless}
08. \choice{ promote       }{ emphasize   }{ share         }{ test}
09. \choice{ decision   }{ quality     }{ status        }{ success}
10. \choice{ chosen      }{ stupid     }{ found        }{ identified}
11. \choice{ exceptional  }{ defensible  }{ replaceable   }{ otherwise}
12. \choice{ inspired   }{ expressed   }{ conducted     }{ secured}
13. \choice{ assigned   }{ rated       }{ matched       }{ arranged}
14. \choice{ put        }{ got         }{ gave          }{ took}
15. \choice{ instead    }{ then        }{ ever          }{ rather}
16. \choice{ selected   }{ passed      }{ marked        }{ introduced}
17. \choice{ before      }{ after       }{ above         }{ below}
18. \choice{ jump       }{ float       }{ drop     }{ fluctuate}
19. \choice{ achieve    }{ undo        }{ maintain      }{ disregard}
20. \choice{ promising  }{ possible    }{ necessary     }{ helpful}
