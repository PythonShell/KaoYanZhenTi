%% Content from http://zhenti.kaoyan.eol.cn/
%% Format by PythonShell
%% 2014-01-08

\begin{center}\textbf{Text 2}\end{center}

\qquad An old saying has it that half of all advertising budgets are wasted --- the trouble is, no one knows which half. In the internet age, at least in theory, this fraction can be much reduced. By watching what people search for, click on and say online, companies can aim ``behavioral'' ads at those most likely to buy.

\qquad In the past couple of weeks a quarrel has illustrated the value to advertisers of such fine-grained information: Should advertisers assume that people are happy to be tracked and sent behavioral ads? Or should they have explicit permission?

\qquad In December 2010 America's Federal Trade Commission (FTC) proposed adding a ``do not track'' (DNT) option to internet browsers, so that users could tell advertisers that they did not want to be followed. Microsoft's Internet Explorer and Apple's Safari both offer DNT; Google's Chrome is due to do so this year. In February the FTC and Digital Advertising Alliance (DAA) agreed that \ul{the industry} would get cracking on responding to DNT requests.

\qquad On May 31st Microsoft set off the row: It said that Internet Explorer 10, the version due to appear Windows 8, would have DNT as a default.

\qquad Advertisers are horrified. Human nature being what it is, most people stick with default settings. Few switch DNT on now, but if tracking is off it will stay off. Bob Liodice, the chief executive of the Association of National Advertisers, says comsumers will be worse off if the industry cannot collect information about their preference. People will not get fewer ads, he says. ``They'll get less meaningful, less targeted ads.''

\qquad It is not yet clear how advertisers will respond. Getting a DNT signal does not oblige anyone to stop tracking, although some companies have promised to do so. Unable to tell whether someone really objects to behavioral ads or whether they are sticking with Microsoft's default, some may ignore a DNT signal and press on anyway.

\qquad Also unclear is why Microsoft has gone it alone. After all, it has an ad business too, which it says will comply with DNT requests, though it is still working out how. If it is trying to upset Google, which relies almost wholly on advertising, it has chosen an indirect method; there is no guraantee that DNT by default will become the norm. DNT does not seem an obviously huge selling point for Windows 8 --- though the firm has compared some of its other products favorably with Google's on that count before. Brendon Lynch, Microsoft's chief privacy officer, blogged: ``We believe consumers should have more control.'' Could it really be that simple?

\vspace{6pt}

26. It is suggested in paragraph 1 that ``behavioral'' ads help advertisers to\par
	\choice{ provide better online services}{ ease competition among themselves}{ avoid complaints from consumers}{ lower their operational costs}

27. ``The industry'' (Line 6, Para.3) refers to\par
	\choice{ internet browser developers}{ digital information analysis}{ e-commerce conductors}{ online advertisers}

28. Bob Liodice holds that setting DNT as a default\par
	\choice{ many cut the number of junk ads}{ fails to affect the ad industry}{ will not benefit consumers}{ goes against human nature}

29. Which of the following is true according to Paragraph 6?\par
	\choice{ Advertisers are willing to implement DNT}{ DNT may not serve its intended purpose}{ DNT is losing its popularity among consumers}{ Advertisers are obliged to offer behavioral ads}

30. The author's attitude towards what Brendon Lynch said in his blog is one of\par
	\choice{ indulgence}{ understanding}{ appreciation}{ skepticism}
