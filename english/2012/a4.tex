46. 在物理学领域,一种做法把这种寻求大同理论的冲动推向极端,试图寻找包含一切的理论——一个涵括我们所看到一切的生成性公式。

47. 这里,达尔文学说似乎做出了证明,因为如果人类有着共同的起源,那么似乎就有理由认为文化的多样性也可以追溯到更为有限的起源。

48. 从共有特征中滤出独有特征,这使我们得以理解复杂的文化行为是如何产生的,并从进化或认知角度理解什么引导了它的走向。

49. 第二次努力——由乔舒亚 格林堡做出——采用更为经验主义的方法来研究语言的普遍性,确定了多种语言(尤其在语法词序方面)的共有特征,这些特征被认为是代表了由认知限制产生的倾向。

50. 乔姆斯基的语言应该显示出语言变化的模式,这些模式并不受语言谱系或贯穿谱系路径的影响;而格林堡式的普遍性则预言了特定的语法词序关系类型之间所存在的紧密互依性。