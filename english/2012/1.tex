%% Content from http://zhenti.kaoyan.eol.cn/
%% Format by PythonShell
%% 2014-01-08

\qquad The ethical judgments of the Supreme Court justices have become an
important issue recently. The court cannot \underline{\quad 1\quad} its legitimacy as
guardian of the rule of law \underline{\quad 2\quad} justices behave like politicians.
Yet, in several instances, justices acted in ways that \underline{\quad 3\quad} the court's
reputation for being independent and impartial.

\qquad Justice Antonin Scalia, for example, appeared at political events.
That kind of activity makes it less likely that the court's decisions
will be \underline{\quad 4\quad} as impartial judgments. Part of the problem is that the
justices are not \underline{\quad 5\quad} by an ethics code. At the very least, the court
should make itself \underline{\quad 6\quad} to the code of conduct that \underline{\quad 7\quad} to the rest of
the federal judiciary.

\qquad This and other similar cases \underline{\quad 8\quad} the question of whether there is still 
a \underline{\quad 9\quad} between the court and politics.

\qquad The framers of the Constitution envisioned law \underline{\quad 10\quad}  having authority
apart from politics. They gave justices permanent positions \underline{\quad 11\quad}  they 
would be free to \underline{\quad 12\quad}  those in power and have no need to \underline{\quad 13\quad}  political
support. Our legal system was designed to set law apart from politics
precisely because they are so closely \underline{\quad 14\quad}.

\qquad Constitutional law is political because it results from choices rooted
in fundamental social \underline{\quad 15\quad}  like liberty and property. When the court
deals with social policy decisions, the law it \underline{\quad 16\quad}  is inescapably
political-which is why decisions split along ideological lines are so
easily \underline{\quad 17\quad}  as unjust.

\qquad The justices must \underline{\quad 18\quad}  doubts about the court's legitimacy by making
themselves \underline{\quad 19\quad}  to the code of conduct. That would make rulings more
likely to be seen as separate from politics and, \underline{\quad 20\quad}  , convincing as
law.

\vspace{6pt}

01. \choice{ emphasize     }{ maintain     }{ modify       }{ recognize}
02. \choice{ when          }{ lest          }{ before       }{ unless}
03. \choice{ restored      }{ weakened      }{ established  }{ eliminated}
04. \choice{ challenged    }{ compromised   }{ suspected    }{ accepted}
05. \choice{ advanced      }{ caught        }{ bound        }{ founded}
06. \choice{ resistant     }{ subject       }{ immune       }{ prone}
07. \choice{ resorts       }{ sticks        }{ loads        }{ applies}
08. \choice{ evade         }{ raise         }{ deny         }{ settle}
09. \choice{ line          }{ barrier       }{ similarity   }{ conflict}
10. \choice{ by            }{ as            }{ though       }{ towards}
11. \choice{ so            }{ since         }{ provided     }{ though}
12. \choice{ serve         }{ satisfy       }{ upset        }{ replace}
13. \choice{ confirm       }{ express       }{ cultivate    }{ offer}
14. \choice{ guarded       }{ followed      }{ studied      }{ tied}
15. \choice{ concepts      }{ theories      }{ divisions    }{ conceptions}
16. \choice{ excludes      }{ questions     }{ shapes       }{ controls}
17. \choice{ dismissed     }{ released      }{ ranked       }{ distorted}
18. \choice{ suppress      }{ exploit       }{ address      }{ ignore}
19. \choice{ accessible    }{ amiable       }{ agreeable    }{ accountable}
20. \choice{ by all means  }{ at all costs  }{ in a word    }{ as a result}
