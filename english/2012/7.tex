%% Content from http://zhenti.kaoyan.eol.cn/
%% Format by PythonShell
%% 2014-01-08

\qquad Since the days of Aristotle, a search for universal principles has characterized the scientific enterprise. In some ways, this quest for commonalities defines science. Newton's laws of motion and Darwinian evolution each bind a host of different phenomena into a single explicatory frame work.

\qquad \ul{(46) In physics, one approach takes this impulse for unification to its extreme, and seeks a theory of everything---a single generative equation for all we see.} It is becoming less clear, however, that such a theory would be a simplification, given the dimensions and universes that it might entail, nonetheless, unification of sorts remains a major goal.

\qquad This tendency in the natural sciences has long been evident in the social sciences too. \ul{(47) Here, Darwinism seems to offer justification for it all humans share common origins, it seems reasonable to suppose that cultural diversity could also be traced to more constrained beginnings.} Just as the bewildering variety of human courtship rituals might all be considered forms of sexual selection, perhaps the world's languages, music, social and religious customs and even history are governed by universal features. \ul{(48) To filter out what is unique from what is shared might enable us to understand how complex cultural behavior arose and what guides it in evolutionary or cognitive terms.}

\qquad That, at least, is the hope. But a comparative study of linguistic traits published online today supplies a reality check. Russell Gray at the University of Auckland and his colleagues consider the evolution of grammars in the light of two previous attempts to find universality in language.

\qquad The most famous of these efforts was initiated by Noam Chomsky, who suggested that humans are born with an innate language-acquisition capacity that dictates a universal grammar. A few generative rules are then sufficient to unfold the entire fundamental structure of a language, which is why children can learn it so quickly.

\qquad \ul{(49) The second, by Joshua Greenberg, takes a more empirical approach to universality identifying traits (particularly in word order) shared by many language which are considered to represent biases that result from cognitive constraints}

\qquad Gray and his colleagues have put them to the test by examining four family trees that between them represent more than 2,000 languages. \ul{(50) Chomsky's grammar should show patterns of language change that are independent of the family tree or the pathway tracked through it, whereas Greenbergian universality predicts strong co-dependencies between particular types of word-order relations.} Neither of these patterns is borne out by the analysis, suggesting that the structures of the languages are lineage-specific and not governed by universals 
