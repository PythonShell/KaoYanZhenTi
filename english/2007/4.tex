%% Content from http://zhenti.kaoyan.eol.cn/
%% Format by PythonShell
%% 2014-01-08

\begin{center}\textbf{Text 3}\end{center}

\qquad During the past generation, the American middle-class family that once could count on hard work and fair play to keep itself financially secure had been transformed by economic risk and new realities. Now a pink slip, a bad diagnosis, or a disappearing spouse can reduce a family from solidly middle class to newly poor in a few months.

\qquad In just one generation, millions of mothers have gone to work, transforming basic family economics. Scholars, policymakers, and critics of all stripes have debated the social implications of these changes, but few have looked at the side effect: family risk has risen as well. Today's families have budgeted to the limits of their new two-paycheck status. As a result, they have lost the parachute they once had in times of financial setback - a back-up earner (usually Mom) who could go into the workforce if the primary earner got laid off or fell sick. This ``added-worker effect'' could support the safety net offered by unemployment insurance or disability insurance to help families weather bad times. But today, a disruption to family fortunes can no longer be made up with extra income from an otherwise-stay-at-home partner.

\qquad During the same period, families have been asked to absorb much more risk in their retirement income. Steelworkers, airline employees, and now those in the auto industry are joining millions of families who must worry about interest rates, stock market fluctuation, and the harsh reality that they may outlive their retirement money. For much of the past year, President Bush campaigned to move Social Security to a saving-account model, with retirees trading much or all of their guaranteed payments for payments depending on investment returns. For younger families, the picture is not any better. Both the absolute cost of healthcare and the share of it borne by families have risen - and newly fashionable health-savings plans are spreading from legislative halls to Wal-Mart workers, with much higher deductibles and a large new dose of investment risk for families' future healthcare. Even demographics are working against the middle class family, as the odds of having a weak elderly parent - and all the attendant need for physical and financial assistance - have jumped eightfold in just one generation.

\qquad From the middle-class family perspective, much of this, understandably, looks far less like an opportunity to exercise more financial responsibility, and a good deal more like a frightening acceleration of the wholesale shift of financial risk onto their already overburdened shoulders. The financial fallout has begun, and the political fallout may not be far behind.

\vspace{6pt}

31. Today's double-income families are at greater financial risk in that\par
	\choice{ the safety net they used to enjoy has disappeared.}{ their chances of being laid off have greatly increased.}{ they are more vulnerable to changes in family economics.}{ they are deprived of unemployment or disability insurance.}

32. As a result of President Bush's reform, retired people may have\par
	\choice{ a higher sense of security.}{ less secured payments.}{ less chance to invest.}{ a guaranteed future.}

33. According to the author, health-savings plans will\par
	\choice{ help reduce the cost of healthcare.}{ popularize among the middle class.}{ compensate for the reduced pensions.}{ increase the families' investment risk.}

34. It can be inferred from the last paragraph that\par
	\choice{ financial risks tend to outweigh political risks.}{ the middle class may face greater political challenges.}{ financial problems may bring about political problems.}{ financial responsibility is an indicator of political status.}

35. Which of the following is the best title for this text?\par
	\choice{ The Middle Class on the Alert}{ The Middle Class on the Cliff}{ The Middle Class in Conflict}{ The Middle Class in Ruins}