\begin{center}\textbf{Text 2}\end{center}

\qquad For the past several years, the Sunday newspaper supplement Parade has featured a column called “Ask Marilyn.” People are invited to query Marilyn vos Savant, who at age 10 had tested at a mental level of someone about 23 years old; that gave her an IQ of 228 - the highest score ever recorded. IQ tests ask you to complete verbal and visual analogies, to envision paper after it has been folded and cut, and to deduce numerical sequences, among other similar tasks. So it is a bit confusing when vos Savant fields such queries from the average Joe (whose IQ is 100) as, What’s the difference between love and fondness? Or what is the nature of luck and coincidence? It’s not obvious how the capacity to visualize objects and to figure out numerical patterns suits one to answer questions that have eluded some of the best poets and philosophers.

\qquad Clearly, intelligence encompasses more than a score on a test. Just what does it mean to be smart? How much of intelligence can be specified, and how much can we learn about it from neurology, genetics, computer science and other fields?

\qquad The defining term of intelligence in humans still seems to be the IQ score, even though IQ tests are not given as often as they used to be. The test comes primarily in two forms: the Stanford-Binet Intelligence Scale and the Wechsler Intelligence Scales (both come in adult and children’s version). Generally costing several hundred dollars, they are usually given only by psychologists, although variations of them populate bookstores and the World Wide Web. Superhigh scores like vos Savant’s are no longer possible, because scoring is now based on a statistical population distribution among age peers, rather than simply dividing the mental age by the chronological age and multiplying by 100. Other standardized tests, such as the Scholastic Assessment Test (SAT) and the Graduate Record Exam (GRE), capture the main aspects of IQ tests.

\qquad Such standardized tests may not assess all the important elements necessary to succeed in school and in life, argues Robert J. Sternberg. In his article “How Intelligent Is Intelligence Testing?”, Sternberg notes that traditional test best assess analytical and verbal skills but fail to measure creativity and practical knowledge, components also critical to problem solving and life success. Moreover, IQ tests do not necessarily predict so well once populations or situations change. Research has found that IQ predicted leadership skills when the tests were given under low-stress conditions, but under high-stress conditions, IQ was negatively correlated with leadership - that is, it predicted the opposite. Anyone who has toiled through SAT will testify that test-taking skill also matters, whether it’s knowing when to guess or what questions to skip.

26.Which of the following may be required in an intelligence test?\par
	\choice{ Answering philosophical questions.}{ Folding or cutting paper into different shapes.}{ Telling the differences between certain concepts.}{ Choosing words or graphs similar to the given ones.}

27.What can be inferred about intelligence testing from Paragraph 3?\par
	\choice{ People no longer use IQ scores as an indicator of intelligence.}{ More versions of IQ tests are now available on the Internet.}{ The test contents and formats for adults and children may be different.}{ Scientists have defined the important elements of human intelligence.}

28.People nowadays can no longer achieve IQ scores as high as vos Savant’s because\par
	\choice{ the scores are obtained through different computational procedures.}{ creativity rather than analytical skills is emphasized now.}{ vos Savant’s case is an extreme one that will not repeat.}{ the defining characteristic of IQ tests has changed.}

29.We can conclude from the last paragraph that\par
	\choice{ test scores may not be reliable indicators of one’s ability.}{ IQ scores and SAT results are highly correlated.}{ testing involves a lot of guesswork.}{ traditional test are out of date.}

30.What is the author’s attitude towards IQ tests?\par
	\choice{ Supportive.}{ Skeptical.}{ Impartial.}{ Biased.}