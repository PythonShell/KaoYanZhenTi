%% Content from http://zhenti.kaoyan.eol.cn/
%% Format by PythonShell
%% 2014-01-08

\qquad The study of law has been recognized for centuries as a basic intellectual discipline in European universities. However, only in recent years has it become a feature of undergraduate programs in Canadian universities. \ul{(46) Traditionally, legal learning has been viewed in such institutions as the special preserve of lawyers, rather than a necessary part of the intellectual equipment of an educated person.} Happily, the older and more continental view of legal education is establishing itself in a number of Canadian universities and some have even begun to offer undergraduate degrees in law.

\qquad If the study of law is beginning to establish itself as part and parcel of a general education, its aims and methods should appeal directly to journalism educators. Law is a discipline which encourages responsible judgment. On the one hand, it provides opportunities to analyze such ideas as justice, democracy and freedom. \ul{(47) On the other, it links these concepts to everyday realities in a manner which is parallel to the links journalists forge on a daily basis as they cover and comment on the news.} For example, notions of evidence and fact, of basic rights and public interest are at work in the process of journalistic judgment and production just as in courts of law. Sharpening judgment by absorbing and reflecting on law is a desirable component of a journalist's intellectual preparation for his or her career.

\qquad \ul{(48) But the idea that the journalist must understand the law more profoundly than an ordinary citizen rests on an understanding of the established conventions and special responsibilities of the news media.} Politics or, more broadly, the functioning of the state, is a major subject for journalists. The better informed they are about the way the state works, the better their reporting will be. \ul{(49) In fact, it is difficult to see how journalists who do not have a clear grasp of the basic features of the Canadian Constitution can do a competent job on political stories.}

\qquad Furthermore, the legal system and the events which occur within it are primary subjects for journalists. While the quality of legal journalism varies greatly, there is an undue reliance amongst many journalists on interpretations supplied to them by lawyers. \ul{(50) While comment and reaction from lawyers may enhance stories, it is preferable for journalists to rely on their own notions of significance and make their own judgments.} These can only come from a well-grounded understanding of the legal system.