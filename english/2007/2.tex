%% Content from http://zhenti.kaoyan.eol.cn/
%% Format by PythonShell
%% 2014-01-08

\begin{center}\textbf{Text 1}\end{center}

\qquad If you were to examine the birth certificates of every soccer player in 2006's World Cup tournament, you would most likely find a noteworthy quirk: elite soccer players are more likely to have been born in the earlier months of the year than in the later months. If you then examined the European national youth teams that feed the World Cup and professional ranks, you would find this strange phenomenon to be even more pronounced.

\qquad What might account for this strange phenomenon? Here are a few guesses: a) certain astrological signs confer superior soccer skills; b) winter-born babies tend to have higher oxygen capacity, which increases soccer stamina; c) soccer-mad parents are more likely to conceive children in springtime, at the annual peak of soccer \ul{mania}; d) none of the above.

\qquad Anders Ericsson, a 58-year-old psychology professor at Florida State University, says he believes strongly in ``none of the above.'' Ericsson grew up in Sweden, and studied nuclear engineering until he realized he would have more opportunity to conduct his own research if he switched to psychology. His first experiment, nearly 30 years ago, involved memory: training a person to hear and then repeat a random series of numbers. ``With the first subject, after about 20 hours of training, his digit span had risen from 7 to 20,'' Ericsson recalls. ``He kept improving, and after about 200 hours of training he had risen to over 80 numbers.''

\qquad This success, coupled with later research showing that memory itself is not genetically determined, led Ericsson to conclude that the act of memorizing is more of a cognitive exercise than an intuitive one. In other words, whatever inborn differences two people may exhibit in their abilities to memorize, those differences are swamped by how well each person ``encodes'' the information. And the best way to learn how to encode information meaningfully, Ericsson determined, was a process known as deliberate practice. Deliberate practice entails more than simply repeating a task. Rather, it involves setting specific goals, obtaining immediate feedback and concentrating as much on technique as on outcome.

\qquad Ericsson and his colleagues have thus taken to studying expert performers in a wide range of pursuits, including soccer. They gather all the data they can, not just performance statistics and biographical details but also the results of their own laboratory experiments with high achievers. Their work makes a rather startling assertion: the trait we commonly call talent is highly overrated. Or, put another way, expert performers - whether in memory or surgery, ballet or computer programming - are nearly always made, not born.

\vspace{6pt}

21. The birthday phenomenon found among soccer players is mentioned to\par
	\choice{ stress the importance of professional training.}{ spotlight the soccer superstars in the World Cup.}{ introduce the topic of what makes expert performance.}{ explain why some soccer teams play better than others.}

22. The word ``mania'' (Line 4, Paragraph 2) most probably means\par
	\choice{ fun.}{ craze.}{ hysteria.}{ excitement.}

23. According to Ericsson, good memory\par
	\choice{ depends on meaningful processing of information.}{ results from intuitive rather than cognitive exercises.}{ is determined by genetic rather than psychological factors.}{ requires immediate feedback and a high degree of concentration.}

24. Ericsson and his colleagues believe that\par
	\choice{ talent is a dominating factor for professional success.}{ biographical data provide the key to excellent performance.}{ the role of talent tends to be overlooked.}{ high achievers owe their success mostly to nurture.}

25. Which of the following proverbs is closest to the message the text tries to convey?\par
	\choice{ ``Faith will move mountains.''}{ ``One reaps what one sows.''}{ ``Practice makes perfect.''}{ ``Like father, like son.''}