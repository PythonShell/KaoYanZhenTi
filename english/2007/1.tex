\qquad By 1830 the former Spanish and Portuguese colonies had become independent nations. The roughly 20 million   \underline{\quad 1\quad}   of these nations looked   \underline{\quad 2\quad}   to the future. Born in the crisis of the old regime and Iberian Colonialism, many of the leaders of independence   \underline{\quad 3\quad}   the ideas of representative government, careers   \underline{\quad 4\quad}   to talent, freedom of commerce and trade, the   \underline{\quad 5\quad}   to private property, and a belief in the individual as the basis of society, \underline{\quad 6\quad}  there was a belief that the new nations should be sovereign and independent states, large enough to be economically viable and integrated by a   \underline{\quad 7\quad}  set of laws.

\qquad On the issue of   \underline{\quad 8\quad}   of religion and the position of the church,  \underline{\quad 9\quad}  , there was less agreement   \underline{\quad 10\quad}   the leadership. Roman Catholicism had been the state religion and the only one   \underline{\quad 11\quad}  by the Spanish crown,  \underline{\quad 12\quad}  most leaders sought to maintain Catholicism   \underline{\quad 13\quad}   the official religion of the new states, some sought to end the   \underline{\quad 14\quad}  of other faiths. The defense of the Church became a rallying   \underline{\quad 15\quad}   for the conservative forces.

\qquad The ideals of the early leaders of independence were often egalitarian, valuing equality of everything. Bolivar had received aid from Haiti and had  \underline{\quad 16\quad}  in return to abolish slavery in the areas he liberated. By 1854 slavery had been abolished everywhere except Spain's   \underline{\quad 17\quad}   colonies. Early promises to end Indian tribute and taxes on people of mixed origin came much   \underline{\quad 18\quad}   because the new nations still needed the revenue such policies   \underline{\quad 19\quad}   Egalitarian sentiments were often tempered by fears that the mass of the population was   \underline{\quad 20\quad}   self-rule and democracy.

01. \choice{ natives }{ inhabitants }{ peoples}{ individuals}
02. \choice{ confusedly }{ cheerfully }{ worriedly}{ hopefully}
03. \choice{ shared }{ forgot }{ attained}{ rejected}
04. \choice{ related }{ close }{ open}{ devoted}
05. \choice{ access }{ succession }{ right}{ return}
06. \choice{ Presumably }{ Incidentally }{ Obviously}{ Generally}
07. \choice{ unique }{ common }{ particular}{ typical}
08. \choice{ freedom }{ origin }{ impact}{ reform}
09. \choice{ therefore }{ however }{ indeed}{ moreover}
10. \choice{ with }{ about }{ among}{ by}
11. \choice{ allowed }{ preached }{ granted}{ funded}
12. \choice{ Since }{ If }{ Unless}{ While}
13. \choice{ as }{ for }{ under}{ against}
14. \choice{ spread }{ interference }{ exclusion}{ influence}
15. \choice{ support }{ cry }{ plea}{ wish}
16. \choice{ urged }{ intended }{ expected}{ promised}
17. \choice{ controlling }{ former }{ remaining}{ original}
18. \choice{ slower }{ faster }{ easier}{ tougher}
19. \choice{ created }{ produced }{ contributed}{ preferred}
20. \choice{ puzzled by }{ hostile to }{ pessimistic about}{ unprepared for}