%% Content from http://zhenti.kaoyan.eol.cn/
%% Format by PythonShell
%% 2014-01-08

\qquad [A] Set a Good Example for Your Kids

\qquad [B] Build Your Kids' Work Skills

\qquad [C] Place Time Limits on Leisure Activities

\qquad [D] Talk about the Future on a Regular Basis

\qquad [E] Help Kids Develop Coping Strategies

\qquad [F] Help Your Kids Figure Out Who They Are

\qquad [G] Build Your Kids' Sense of Responsibility

\begin{center}\textbf{How Can a Parent Help?}\end{center}

\qquad Mothers and fathers can do a lot to ensure a safe landing in early adulthood for their kids. Even if a job's starting salary seems too small to satisfy an emerging adult's need for rapid content, the transition from school to work can be less of a setback if the start-up adult is ready for the move. Here are a few measures, drawn from my book \emph{Ready or Not, Here Life Comes}, that parents can take to prevent what I call ``work-life unreadiness.''

\qquad \ul{(41) \qquad\qquad}.

\qquad You can start this process when they are 11 or 12. Periodically review their emerging strengths and weaknesses with them and work together on any shortcomings, like difficulty in communicating well or collaborating. Also, identify the kinds of interests they keep coming back to, as these offer clues to the careers that will fit them best.

\qquad \ul{(42) \qquad\qquad}.

\qquad Kids need a range of authentic role models - as opposed to members of their clique, pop stars and vaunted athletes. Have regular dinner-table discussions about people the family knows and how they got where they are. Discuss the joys and downsides of your own career and encourage your kids to form some ideas about their own future. When asked what they want to do, they should be discouraged from saying ``I have no idea.'' They can change their minds 200 times, but having only a foggy view of the future is of little good.

\qquad \ul{(43) \qquad\qquad}.

\qquad Teachers are responsible for teaching kids how to learn; parents should be responsible for teaching them how to work. Assign responsibilities around the house and make sure homework deadlines are met. Encourage teenagers to take a part-time job. Kids need plenty of practice delaying gratification and deploying effective organizational skills, such as managing time and setting priorities.

\qquad \ul{(44) \qquad\qquad}.

\qquad Playing video games encourages immediate content. And hours of watching TV shows with canned laughter only teaches kids to process information in a passive way. At the same time, listening through earphones to the same monotonous beats for long stretches encourages kids to stay inside their bubble instead of pursuing other endeavors. All these activities can prevent the growth of important communication and thinking skills and make it difficult for kids to develop the kind of sustained concentration they will need for most jobs.

\qquad \ul{(45) \qquad\qquad}.

\qquad They should know how to deal with setbacks, stresses and feelings of inadequacy. They should also learn how to solve problems and resolve conflicts, ways to brainstorm and think critically. Discussions at home can help kids practice doing these things and help them apply these skills to everyday life situations.

\qquad What about the son or daughter who is grown but seems to be struggling and wandering aimlessly through early adulthood? Parents still have a major role to play, but now it is more delicate. They have to be careful not to come across as disappointed in their child. They should exhibit strong interest and respect for whatever currently interests their fledging adult (as naive or ill conceived as it may seem) while becoming a partner in exploring options for the future. Most of all, these new adults must feel that they are respected and supported by a family that appreciates them.
