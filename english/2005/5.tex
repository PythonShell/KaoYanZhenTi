%% Content from http://zhenti.kaoyan.eol.cn/
%% Format by PythonShell
%% 2014-01-07

\begin{center}\textbf{Text 4}\end{center}

\qquad Americans no longer expect public figures, whether in speech or in writing, to command the English language with skill and gift. Nor do they aspire to such command themselves. In his latest book, \emph{Doing Our Own Thing: The Degradation of Language and Music and Why We Should Like, Care}, John McWhorter, a linguist and controversialist of mixed liberal and conservative views, sees the triumph of 1960s counter-culture as responsible for the decline of formal English.

\qquad Blaming the permissive 1960s is nothing new, but this is not yet another criticism against the decline in education. Mr. McWhorter's academic speciality is language history and change, and he sees the gradual disappearance of ``whom'', for example, to be natural and no more regrettable than the loss of the case-endings of Old English.

\qquad But the cult of the authentic and the personal, ``doing our own thing'', has spelt the death of formal speech, writing, poetry and music. While even the modestly educated sought an elevated tone when they put pen to paper before the 1960s, even the most well regarded writing since then has sought to capture spoken English on the page. Equally, in poetry, the highly personal, performative genre is the only form that could claim real liveliness. In both oral and written English, \ul{talking} is triumphing over speaking, spontaneity over craft.

\qquad Illustrated with an entertaining array of examples from both high and low culture, the trend that Mr. McWhorter documents is unmistakable. But it is less clear, to take the question of his subtitle, why we should, like, care. As a linguist, he acknowledges that all varieties of human language, including non-standard ones like Black English, can be powerfully expressive -- there exists no language or dialect in the world that cannot convey complex ideas. He is not arguing, as many do, that we can no longer think straight because we do not talk proper.

\qquad Russians have a deep love for their own language and carry large chunks of memorized poetry in their heads, while Italian politicians tend to elaborate speech that would seem old-fashioned to most English-speakers. Mr. McWhorter acknowledges that formal language is not strictly necessary, and proposes no radical education reforms -- he is really grieving over the loss of something beautiful more than useful. We now take our English ``on paper plates instead of china''. A shame, perhaps, but probably an inevitable one.

\vspace{6pt}

36. According to McWhorter, the decline of formal English\par
	\choice{ is inevitable in radical education reforms}{ is but all too natural in language development}{ has caused the controversy over the counter-culture}{ brought about changes in public attitudes in the 1960s}

37. The word ``talking'' (Line 5, Paragraph 3) denotes\par
	\choice{ modesty}{ personality}{ liveliness}{ informality}

38. To which of the following statements would McWhorter most likely agree?\par
	\choice{ Logical thinking is not necessarily related to the way we talk.}{ Black English can be more expressive than standard English.}{ Non-standard varieties of human language are just as entertaining.}{ Of all the varieties, standard English can best convey complex ideas.}

39. The description of Russians' love of memorizing poetry shows the author's\par
	\choice{ interest in their language}{ appreciation of their efforts}{ admiration for their memory}{ contempt for their old-fashionedness}

40. According to the last paragraph, ``paper plates'' is to ``china'' as\par
	\choice{ ``temporary'' is to ``permanent''}{ ``radical'' is to ``conservative''}{ ``functional'' is to ``artistic''}{ ``humble'' is to ``noble''}