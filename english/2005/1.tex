\qquad The human nose is an underrated tool. Humans are often thought to be insensitive smellers compared with animals, \underline{\quad 1\quad} this is largely because, \underline{\quad 2\quad} animals, we stand upright. This means that our noses are \underline{\quad 3\quad} to perceiving those smells which float through the air, \underline{\quad 4\quad} the majority of smells which stick to surfaces. In fact, \underline{\quad 5\quad}, we are extremely sensitive to smells, \underline{\quad 6\quad} we do not generally realize it. Our noses are capable of \underline{\quad 7\quad} human smells even when these are \underline{\quad 8\quad} to far below one part in one million.

\qquad Strangely, some people find that they can smell one type of flower but not another, \underline{\quad 9\quad} others are sensitive to the smells of both flowers. This may be because some people do not have the genes necessary to generate \underline{\quad 10\quad} smell receptors in the nose. These receptors are the cells which sense smells and send \underline{\quad 11\quad} to the brain. However, it has been found that even people insensitive to a certain smell \underline{\quad 12\quad} can suddenly become sensitive to it when \underline{\quad 13\quad} to it often enough.

\qquad The explanation for insensitivity to smell seems to be that brain finds it \underline{\quad 14\quad} to keep all smell receptors working all the time but can \underline{\quad 15\quad} new receptors if necessary. This may \underline{\quad 16\quad} explain why we are not usually sensitive to our own smells -- we simply do not need to be. We are not \underline{\quad 17\quad} of the usual smell of our own house, but we \underline{\quad 18\quad} new smells when we visit someone else's. The brain finds it best to keep smell receptors \underline{\quad 19\quad} for unfamiliar and emergency signals \underline{\quad 20\quad} the smell of smoke, which might indicate the danger of fire.

\vspace{6pt}

01. \choice{ although }{ as }{ but }{ while}
02. \choice{ above }{ unlike }{ excluding }{ besides}
03. \choice{ limited }{ committed }{ dedicated }{ confined}
04. \choice{ catching }{ ignoring }{ missing }{ tracking}
05. \choice{ anyway }{ though }{ instead }{ therefore}
06. \choice{ even if }{ if only }{ only if }{ as if}
07. \choice{ distinguishing }{ discovering }{ determining }{ detecting}
08. \choice{ diluted }{ dissolved }{ dispersed }{ diffused}
09. \choice{ when }{ since }{ for }{ whereas}
10. \choice{ unusual }{ particular }{ unique }{ typical}
11. \choice{ signs }{ stimuli }{ messages }{ impulses}
12. \choice{ at first }{ at all }{ at large }{ at times}
13. \choice{ subjected }{ left }{ drawn }{ exposed}
14. \choice{ ineffective }{ incompetent }{ inefficient }{ insufficient}
15. \choice{ introduce }{ summon }{ trigger }{ create}
16. \choice{ still }{ also }{ otherwise }{ nevertheless}
17. \choice{ sure }{ sick }{ aware }{ tired}
18. \choice{ tolerate }{ repel }{ neglect }{ notice}
19. \choice{ available }{ reliable }{ identifiable }{ suitable}
20. \choice{ similar to }{ such as }{ along with }{ aside from}