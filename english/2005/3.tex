\begin{center}\textbf{Text 2}\end{center}

\qquad Do you remember all those years when scientists argued that smoking would kill us but the doubters insisted that we didn't know for sure? That the evidence was inconclusive, the science uncertain? That the antismoking lobby was out to destroy our way of life and the government should stay out of the way? Lots of Americans bought that nonsense, and over three decades, some 10 million smokers went to early graves.

\qquad There are upsetting parallels today, as scientists in one wave after another try to awaken us to the growing threat of global warming. The latest was a panel from the National Academy of Sciences, enlisted by the White House, to tell us that the Earth's atmosphere is definitely warming and that the problem is largely man-made. The clear message is that we should get moving to protect ourselves. The president of the National Academy, Bruce Alberts, added this key point in the preface to the panel's report: ``Science never has all the answers. But science does provide us with the best available guide to the future, and it is critical that our nation and the world base important policies on the best judgments that science can provide concerning the future consequences of present actions.''

\qquad Just as on smoking, voices now come from many quarters insisting that the science about global warming is incomplete, that it's OK to keep pouring fumes into the air until we know for sure. This is a dangerous game: by the time 100 percent of the evidence is in, it may be too late. With the risks obvious and growing, a prudent people would take out an insurance policy now.

\qquad Fortunately, the White House is starting to pay attention. But it's obvious that a majority of the president's advisers still don't take global warming seriously. Instead of a plan of action, they continue to press for more research -- a classic case of ``\ul{paralysis by analysis}''.

\qquad To serve as responsible stewards of the planet, we must press forward on deeper atmospheric and oceanic research. But research alone is inadequate. If the Administration won't take the legislative initiative, Congress should help to begin fashioning conservation measures. A bill by Democratic Senator Robert Byrd of West Virginia, which would offer financial incentives for private industry, is a promising start. Many see that the country is getting ready to build lots of new power plants to meet our energy needs. If we are ever going to protect the atmosphere, it is crucial that those new plants be environmentally sound.

\vspace{6pt}

26. An argument made by supporters of smoking was that\par
	\choice{ there was no scientific evidence of the correlation between smoking and death}{ the number of early deaths of smokers in the past decades was insignificant}{ people had the freedom to choose their own way of life}{ antismoking people were usually talking nonsense}

27. According to Bruce Alberts, science can serve as\par
	\choice{ a protector}{ a judge}{ a critic}{ a guide}

28. What does the author mean by ``paralysis by analysis'' (Last line, Paragraph 4)?\par
	\choice{ Endless studies kill action.}{ Careful investigation reveals truth.}{ Prudent planning hinders progress.}{ Extensive research helps decision-making.}

29. According to the author, what should the Administration do about global warming?\par
	\choice{ Offer aid to build cleaner power plants.}{ Raise public awareness of conservation.}{ Press for further scientific research.}{ Take some legislative measures.}

30. The author associates the issue of global warming with that of smoking because\par
	\choice{ they both suffered from the government's negligence}{ a lesson from the latter is applicable to the former}{ the outcome of the latter aggravates the former}{ both of them have turned from bad to worse}