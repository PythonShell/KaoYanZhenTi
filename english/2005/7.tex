\qquad It is not easy to talk about the role of the mass media in this overwhelmingly significant phase in European history. History and news become confused, and one's impressions tend to be a mixture of skepticism and optimism. \ul{(46) Television is one of the means by which these feelings are created and conveyed -- and perhaps never before has it served so much to connect different peoples and nations as in the recent events in Europe.} The Europe that is now forming cannot be anything other than its peoples, their cultures and national identities. With this in mind we can begin to analyze the European television scene. \ul{(47) In Europe, as elsewhere, multi-media groups have been increasingly successful: groups which bring together television, radio, newspapers, magazines and publishing houses that work in relation to one another.} One Italian example would be the Berlusconi group, while abroad Maxwell and Murdoch come to mind.

\qquad Clearly, only the biggest and most flexible television companies are going to be able to compete in such a rich and hotly-contested market. \ul{(48) This alone demonstrates that the television business is not an easy world to survive in, a fact underlined by statistics that show that out of eighty European television networks, no less than 50\% took a loss in 1989.}

\qquad Moreover, the integration of the European community will oblige television companies to cooperate more closely in terms of both production and distribution.

\qquad \ul{(49) Creating a ``European identity'' that respects the different cultures and traditions which go to make up the connecting fabric of the Old Continent is no easy task and demands a strategic choice} -- that of producing programs in Europe for Europe. This entails reducing our dependence on the North American market, whose programs relate to experiences and cultural traditions which are different from our own.

\qquad In order to achieve these objectives, we must concentrate more on co-productions, the exchange of news, documentary services and training. This also involves the agreements between European countries for the creation of a European bank for Television Production which, on the model of the European Investments Bank, will handle the finances necessary for production costs. \ul{(50) In dealing with a challenge on such a scale, it is no exaggeration to say, ``United we stand, divided we fall''} -- and if I had to choose a slogan it would be ``Unity in our diversity.'' A unity of objectives that nonetheless respect the varied peculiarities of each country.