%% Content from http://zhenti.kaoyan.eol.cn/
%% Format by PythonShell
%% 2014-01-07

\begin{center}\textbf{Text 1}\end{center}

\qquad Everybody loves a fat pay rise. Yet pleasure at your own can vanish if you learn that a colleague has been given a bigger one. Indeed, if he has a reputation for slacking, you might even be outraged. Such behaviour is regarded as ``all too human'', with the underlying assumption that other animals would not be capable of this finely developed sense of grievance. But a study by Sarah Brosnan and Frans de Waal of Emory University in Atlanta, Georgia, which has just been published in \emph{Nature}, suggests that \ul{it is all too monkey}, as well.

\qquad The researchers studied the behaviour of female brown capuchin monkeys. They look cute. They are good-natured, co-operative creatures, and they share their food readily. Above all, like their female human counterparts, they tend to pay much closer attention to the value of ``goods and services'' than males.

\qquad Such characteristics make them perfect candidates for Dr. Brosnan's and Dr. de Waal's study. The researchers spent two years teaching their monkeys to exchange tokens for food. Normally, the monkeys were happy enough to exchange pieces of rock for slices of cucumber. However, when two monkeys were placed in separate but adjoining chambers, so that each could observe what the other was getting in return for its rock, their behaviour became markedly different.

\qquad In the world of capuchins, grapes are luxury goods (and much preferable to cucumbers). So when one monkey was handed a grape in exchange for her token, the second was reluctant to hand hers over for a mere piece of cucumber. And if one received a grape without having to provide her token in exchange at all, the other either tossed her own token at the researcher or out of the chamber, or refused to accept the slice of cucumber. Indeed, the mere presence of a grape in the other chamber (without an actual monkey to eat it) was enough to induce resentment in a female capuchin.

\qquad The researchers suggest that capuchin monkeys, like humans, are guided by social emotions. In the wild, they are a cooperative, group-living species. Such cooperation is likely to be stable only when each animal feels it is not being cheated. Feelings of righteous indignation, it seems, are not the preserve of people alone. Refusing a lesser reward completely makes these feelings abundantly clear to other members of the group. However, whether such a sense of fairness evolved independently in capuchins and humans, or whether it stems from the common ancestor that the species had 35 million years ago, is, as yet, an unanswered question.

\vspace{6pt}

21. In the opening paragraph, the author introduces his topic by\par
	\choice{ posing a contrast}{ justifying an assumption}{ making a comparison}{ explaining a phenomenon}

22. The statement ``it is all too monkey'' (Last line, Paragraph l) implies that\par
	\choice{ monkeys are also outraged by slack rivals}{ resenting unfairness is also monkeys' nature}{ monkeys, like humans, tend to be jealous of each other}{ no animals other than monkeys can develop such emotions}

23. Female capuchin monkeys were chosen for the research most probably because they are\par
	\choice{ more inclined to weigh what they get}{ attentive to researchers' instructions}{ nice in both appearance and temperament}{ more generous than their male companions}

24. Dr. Brosnan and Dr. de Waal have eventually found in their study that the monkeys\par
	\choice{ prefer grapes to cucumbers}{ can be taught to exchange things}{ will not be cooperative if feeling cheated}{ are unhappy when separated from others}

25. What can we infer from the last paragraph?\par
	\choice{ Monkeys can be trained to develop social emotions.}{ Human indignation evolved from an uncertain source.}{ Animals usually show their feelings openly as humans do.}{ Cooperation among monkeys remains stable only in the wild.}