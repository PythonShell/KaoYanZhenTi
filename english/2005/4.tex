\begin{center}\textbf{Text 3}\end{center}

\qquad Of all the components of a good night's sleep, dreams seem to be least within our control. In dreams, a window opens into a world where logic is suspended and dead people speak. A century ago, Freud formulated his revolutionary theory that dreams were the disguised shadows of our unconscious desires and fears; by the late 1970s, neurologists had switched to thinking of them as just ``mental noise'' -- the random byproducts of the neural-repair work that goes on during sleep. Now researchers suspect that dreams are part of the mind's emotional thermostat, regulating moods while the brain is ``off-line.'' And one leading authority says that these intensely powerful mental events can be not only harnessed but actually brought under conscious control, to help us sleep and feel better, ``It's your dream,'' says Rosalind Cartwright, chair of psychology at Chicago's Medical Center. ``If you don't like it, change it.''

\qquad Evidence from brain imaging supports this view. The brain is as active during REM (rapid eye movement) sleep -- when most vivid dreams occur -- as it is when fully awake, says Dr. Eric Nofzinger at the University of Pittsburgh. But not all parts of the brain are equally involved; the limbic system (the ``emotional brain'') is especially active, while the prefrontal cortex (the center of intellect and reasoning) is relatively quiet. ``We wake up from dreams happy or depressed, and those feelings can stay with us all day.'' says Stanford sleep researcher Dr. William Dement.

\qquad The link between dreams and emotions shows up among the patients in Cartwright's clinic. Most people seem to have more bad dreams early in the night, progressing toward happier ones before awakening, suggesting that they are working through negative feelings generated during the day. Because our conscious mind is occupied with daily life we don't always think about the emotional significance of the day's events -- until, it appears, we begin to dream.

\qquad And this process need not be left to the unconscious. Cartwright believes one can exercise conscious control over recurring bad dreams. As soon as you awaken, identify what is upsetting about the dream. Visualize how you would like it to end instead; the next time it occurs, try to wake up just enough to control its course. With much practice people can learn to, literally, do it in their sleep.

\qquad At the end of the day, there's probably little reason to pay attention to our dreams at all unless they keep us from sleeping or ``we wake up in a panic,'' Cartwright says. Terrorism, economic uncertainties and general feelings of insecurity have increased people's anxiety. Those suffering from persistent nightmares should seek help from a therapist. For the rest of us, the brain has its ways of working through bad feelings. Sleep -- or rather dream -- on it and you'll feel better in the morning.

\vspace{6pt}

31. Researchers have come to believe that dreams\par
	\choice{ can be modified in their courses}{ are susceptible to emotional changes}{ reflect our innermost desires and fears}{ are a random outcome of neural repairs}

32. By referring to the limbic system, the author intends to show\par
	\choice{ its function in our dreams}{ the mechanism of REM sleep}{ the relation of dreams to emotions}{ its difference from the prefrontal cortex}

33. The negative feelings generated during the day tend to\par
	\choice{ aggravate in our unconscious mind}{ develop into happy dreams}{ persist till the time we fall asleep}{ show up in dreams early at night}

34. Cartwright seems to suggest that\par
	\choice{ waking up in time is essential to the ridding of bad dreams}{ visualizing bad dreams helps bring them under control}{ dreams should be left to their natural progression}{ dreaming may not entirely belong to the unconscious}

35. What advice might Cartwright give to those who sometimes have bad dreams?\par
	\choice{ Lead your life as usual.}{ Seek professional help.}{ Exercise conscious control.}{ Avoid anxiety in the daytime.}